\documentclass[helvetica]{seminar} 
\input{xy}
\xyoption{all}
\usepackage{graphicx} 
\usepackage{slidesec} 
\usepackage{url}
\usepackage[framemethod=TikZ]{mdframed}
\usepackage{color}
\usepackage[normalem]{ulem}  

\def\dash---{\unskip\kern.16667em---\penalty\exhyphenpenalty\hskip.16667em\ignorespaces}
\long\def\symbolfootnote[#1]#2{\begingroup%
\def\thefootnote{\fnsymbol{footnote}}\footnote[#1]{#2}\endgroup}

% to fix problems making landscape seminar pdfs
% Letter...
\pdfpagewidth=11truein
\pdfpageheight=8.5truein
\pdfhorigin=1truein     % default value(?), but doesn't work without
\pdfvorigin=1truein     % default value(?), but doesn't work without
% A4
%\pdfpagewidth=297truemm % your milage may vary....
%\pdfpageheight=210truemm
%\pdfhorigin=1truein     % default value(?), but doesn't work without
%\pdfvorigin=1truein     % default value(?), but doesn't work without


\renewcommand{\familydefault}{\sfdefault}  
 
\input{seminar.bug} 
\input{seminar.bg2} % See the Seminar bugs list 
 
\slideframe{none} 
 
 
\usepackage{fancyhdr} 
 
% Headers and footers personalization using the `fancyhdr' package 
\fancyhf{} % Clear all fields 
\renewcommand{\headrulewidth}{0mm} 
\renewcommand{\footrulewidth}{0.1mm} 
 
\fancyfoot[L]{\tiny IETF 99} 
\fancyfoot[C]{\tiny TLS}
\fancyfoot[R]{\tiny \theslide} 
 
 
% To center horizontally the headers and footers (see seminar.bug) 
\renewcommand{\headwidth}{\textwidth} 

% To adjust the frame length to the header and footer ones 
\autoslidemarginstrue 
\pagestyle{fancy} 
 

\newcommand{\heading}[1]{% 
  \begin{center} 
    \large\bf 
    #1 
  \end{center} 
  \vspace{.4 in}} 



\begin{document}

\begin{slide}
\begin{center}
\vspace{.5 in}
\LARGE{{\bf}DTLS 1.3\\{\small \verb^draft-ietf-tls-dtls13-01^}}\\
\vspace{.2in}
\large{
\begin{tabular}{c c c}
\textbf{Eric Rescorla} & Hannes Tschofenig & Nagendra Modadugu \\
Mozilla & ARM & Google \\
\end{tabular}
}
\end{center}
\end{slide}

\centerslidesfalse 

\begin{slide}
\heading{Reminder: ACKs}

\begin{itemize}
\item DTLS historically used an implicit ACK
  \begin{itemize}
  \item Receiving the start of the next flight means the flight was received
  \end{itemize}

\item Simple (but also simpleminded)
  \begin{itemize}
  \item Slightly tricky to implement
  \item Gives limited congestion feedback
  \item Handles single-packet loss badly
  \end{itemize}

\item Interacts badly with some TLS 1.3 features (like NST)
\item Solution: introduce an explicit ACK
\end{itemize}
\end{slide}


\begin{slide}
  \heading{Current proposal: SACK}

  \begin{itemize}
  \item ACKs contain the sequence numbers of received records
    \begin{itemize}
    \item From the current flight only
    \item Senders need to maintain a map from \emph{records} to \emph{handshake messages}
    \item Senders SHOULD NOT retransmit ACKed data and MUST NOT retransmit ACKed flights
    \end{itemize}
  \item Separate record type, not a handshake record
    \begin{itemize}
    \item MUST be sent with epoch $>=$ than what's being ACKed
    \item Sent with the current sending key
    \end{itemize}
  \item Receiving the next flight is an implicit ACK
  \end{itemize}
\end{slide}


\begin{slide}
  \heading{When should receivers ACK}

  \begin{itemize}
  \item When receiving messages that don't have in-handshake responses
  \item When it looks like messages might have gotten lost
    \begin{itemize}
    \item When you get an out-of-order record
    \item When you get a partial record and don't get the rest ``immediately''
    \end{itemize}
  \item Not for non-handshake messages
  \end{itemize}

\end{slide}


\begin{slide}
  \heading{Reduced Headers}

  \begin{itemize}
  \item What can we remove?
    \begin{itemize}
    \item Nonce
    \item Content type and version (hopefully)
    \end{itemize}

  \item Proposal (thanks to MT):
  \end{itemize}

  {\footnotesize
\begin{verbatim}
    struct {
      uint16 epoch_sequence // format = 001eesss ssssssss
      uint16 length;
      opaque encrypted_record[length];
    } DtlsHeader;
\end{verbatim}
}

\end{slide}


\begin{slide}
  \heading{Connection IDs}

  \begin{itemize}
  \item Lack of Connection IDs clearly a problem for NATs/IoT, etc.
  \item Connection IDs are also a clear privacy problem
    \begin{itemize}
    \item Lots of proposals for how to do privacy preserving Conn IDs
    \item ... but they're complicated and none of them seem totally baked
    \item This seems like less of a privacy problem than with browsers (QUIC)     
    \end{itemize}

  \item Proposal: use a fixed connection ID for now
    \begin{itemize}
      \item In an extension
      \item We can always replace it later
    \end{itemize}
  \end{itemize}
\end{slide}


\begin{slide}
  \heading{Concrete proposal}
  
{\footnotesize
\begin{verbatim}
    struct {
       opaque connection_id<0..255>;
    } ConnectionId;

    struct {
      uint16 epoch_sequence // format = 001eesss ssssssss
      opaque connection_id[connection_id_length];
      uint16 length;
      opaque encrypted_record[length];
    } DtlsHeader;
\end{verbatim}
}

\begin{itemize}
\item IDs are used if client offers and server answers
  \begin{itemize}
    \item On all (non-0RTT)? encrypted records
  \end{itemize}
\item Each side \emph{sends} with the other's ID
  \begin{itemize}
    \item Because IDs are unframed, 0-length IDs are just omitted
  \end{itemize}
\end{itemize}
\end{slide}

\begin{slide}
\heading{Other issues?}

\end{slide}

\end{document} 

