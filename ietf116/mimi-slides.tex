\documentclass[helvetica]{beamer}
\usetheme{Boadilla}
\input{xy}
\xyoption{all}
\usepackage{graphicx} 
\usepackage{slidesec} 
\usepackage{url}
\usepackage[framemethod=TikZ]{mdframed}
\usepackage{tikz}
\usetikzlibrary{calc,quotes,arrows.meta}
\usepackage{color}
\usepackage[normalem]{ulem}
\usepackage{verbatim}
\setbeamertemplate{itemize items}[circle]
\def\dash---{\unskip\kern.16667em---\penalty\exhyphenpenalty\hskip.16667em\ignorespaces}
\long\def\symbolfootnote[#1]#2{\begingroup%
\def\thefootnote{\fnsymbol{footnote}}\footnote[#1]{#2}\endgroup}

\title{MIMI Transport Requirements}
\author{Eric Rescorla \\\url{ekr@rtfm.com}}
\date{2022-11-10}

\begin{document}

\begin{frame}
  \titlepage
\end{frame}

\begin{frame}{Abstract Architecture}

  \vspace{.2in}
\begin{center}
  \includegraphics[width=4in]{messaging-architecture}
\end{center}
\end{frame}


\begin{frame}{Protocol Breakdown}

  \vspace{.2in}
\begin{center}
  \includegraphics[width=4in]{messaging-setting}
\end{center}
\end{frame}

\begin{frame}{Question: How much are we defining?}

  \begin{itemize}
  \item A full system obviously needs a client-to-server protocol
    \begin{itemize}
    \item Message protection and content need to be E2E
    \item ... but message transport is not      
    \end{itemize}
  \item Most existing systems (XMPP, SIMPLE, etc. do it all)
  \item Is client $\longleftrightarrow$ server in scope?
  \end{itemize}  
\end{frame}


\begin{frame}{Naming and discovery}

  \begin{itemize}
  \item Two main kinds of existing identifiers
    \begin{itemize}
    \item \emph{System Specific (SSI).} e.g., ``\texttt{1.650.555.1000} on WhatsApp''
      (or maybe \texttt{mimi:16505551000@whatsapp.com})
    \item \emph{System Independent (SII):} e.g., \texttt{1.650.555.1000} or \texttt{ekr}
    \end{itemize}
  \item In general, an SII isn't enough to automatically contact someone
    \begin{itemize}
    \item You don't know what system they are on
    \item The same SII may appear on multiple systems (e.g., phone numbers on WhatsApp + iMessage)
    \end{itemize}
  \item \emph{Discovery} is the process of determining which system(s) an SII appears on
  \end{itemize}
\end{frame}

\begin{frame}{Question: Do we need to support discovery?}
  
  \begin{enumerate}
    \item Only solve for SSIs
    \item Solve for SSIs now and build discovery separately
    \item Integrate discovery and consent (SPIN, draft-rosenberg)
      \begin{itemize}
      \item These designs assume that Alice has \emph{some} out of band channel to contact Bob
      \item What about systems that just use handles?
      \end{itemize}
  \end{enumerate}      
\end{frame}

\begin{frame}{Consent?}
  \includegraphics[width=3in]{archer-spam}
  \begin{itemize}
  \item Alice just send messages to Bob if she has his identifier
    \begin{itemize}
    \item This is a spam vector      
    \end{itemize}
  \item Or does she need to get consent first?
    \begin{itemize}
    \item Typically this consists of sending an \emph{invite}
    \item ... Bob has to accept before seeing Alice's messages      
    \end{itemize}
  \end{itemize}
\end{frame}


\begin{frame}{KeyPackage Availability}
  \begin{itemize}
  \item Sending encrypted messages requires the KeyPackage\footnote{Recall: the KeyPackage contains the public key of the recipient.}
  \item This leaks whether the recipient exists
  \item Potential risk of KeyPackage exhaustion
  \end{itemize}
\end{frame}

\begin{frame}{Question: which modes do we support?}
  \begin{enumerate}
  \item Alice can send messages to Bob immediately    
  \item Alice can send messages to Bob but they're quarantined until Bob accepts
    \begin{itemize}
    \item Potential concerns about excess data on Bob's side      
    \end{itemize}
  \item Alice can't do anything until Bob consents
\end{enumerate}    
\end{frame}


\begin{frame}{Messages and Channels}
  \begin{itemize}
  \item (At least) three modalities
    \begin{itemize}
    \item 1-1 messages
    \item Group messages with $>2$ people
    \item Channels/rooms
    \end{itemize}
    
  \item Some overlap between group messages and channels
    \begin{itemize}
    \item \textbf{Group} messages are defined by the members
    \item Can't add new members (unlike channels!)      
    \end{itemize}
  \item What about multiple group messages (or 1-1 messages) with the same membership?
    \begin{itemize}
    \item This is handled inconsistently by existing messaging services
    \end{itemize}
  \end{itemize}
\end{frame}

\begin{frame}{Question: What models do we support?}

  \begin{enumerate}
  \item Everything's a group (this is what MLS thinks)
    \begin{itemize}
    \item Is this rich enough? What about moderation, etc.?      
    \end{itemize}
  \item Channels are fundamentally different (XMPP, Slack, etc.)
    \begin{itemize}
    \item And maybe we don't need group messages?
    \item Or do channels first and then groups laer.      
    \end{itemize}
  \end{enumerate}
  \begin{itemize}
  \item MLS can support any of these modes
  \end{itemize}
\end{frame}


\begin{frame}{Channel/room Management}

  \begin{itemize}
  \item XMPP (MUC) and Matrix have fairly complicated room management
    \begin{itemize}
    \item Ownership
    \item Moderation
    \item Kick/ban etc
    \item Ask to join chats
    \end{itemize}
  \item A lot of systems don't
  \item This is out of charter scope. Assumption is that this only works on the owning service.
  \end{itemize}

\end{frame}

\begin{frame}{Question: room/channel portability?}

  \begin{itemize}
  \item General assumption seems to be a room/channel lives on one system
    \begin{itemize}
    \item Except for Matrix      
    \end{itemize}
    
  \item Is it possible to move channels between owners?
    \begin{itemize}
    \item For instance, if the last member from the owner leaves
    \item Linearized matrix allows this
    \item XMPP, MTP, etc. don't seem to allow this
    \end{itemize}
  \end{itemize}
\end{frame}

\begin{frame}{MLS channel state versus transport channel state}

  \begin{itemize}
  \item State exists at both levels
    \begin{itemize}
    \item Transport: which people are in the room?
    \item MLS: which keys are in the room?
    \end{itemize}
  \item How tightly in sync are these?
  \item And do they have to be cryptographically bound    
  \end{itemize}
\end{frame}

\begin{frame}{Warmup: Bob gets a new phone}

  \includegraphics[width=3.5in]{MIMI-new-phone}
\end{frame}

\begin{frame}{Charlie added to chat (1)}

  \includegraphics[width=3.5in]{MIMI-new-user-1}
\end{frame}

\begin{frame}{Charlie added to chat (2)}

  \includegraphics[width=3.5in]{MIMI-new-user-2}
\end{frame}

\begin{frame}{Question: Do we need MLS-level access control?}
  \begin{itemize}
  \item MLS just controls which keys are in the group
    \begin{itemize}
    \item But who decides which Commits are accepted by the group members?
    \item For instance, can the owning provider add members 
    \item Or do group members need to approve it?      
    \end{itemize}
    
  \item Some options
    \begin{enumerate}
    \item Do nothing
    \item Specify MLS-level state change authentication
    \item Require MLS-level state change authentication     
    \end{enumerate}
  \end{itemize}
  
\end{frame}

\begin{frame}{Question: Privacy for metadata?}

  \begin{itemize}
  \item MLS mostly protects the content of messages
  \item But what about metadata?
    \begin{itemize}
    \item Who is messaging who      
    \item Channel membership
    \item Contact lists
    \end{itemize}
  \item Are we going to try to do anything here?    
  \end{itemize}
\end{frame}


\end{document}