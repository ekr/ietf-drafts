\documentclass[helvetica]{seminar} 
\input{xy}
\xyoption{all}
\usepackage{graphicx} 
\usepackage{slidesec} 
\usepackage{url}
\usepackage[framemethod=TikZ]{mdframed}
\usepackage{color}
\usepackage[normalem]{ulem}  

\def\dash---{\unskip\kern.16667em---\penalty\exhyphenpenalty\hskip.16667em\ignorespaces}
\long\def\symbolfootnote[#1]#2{\begingroup%
\def\thefootnote{\fnsymbol{footnote}}\footnote[#1]{#2}\endgroup}

% to fix problems making landscape seminar pdfs
% Letter...
\pdfpagewidth=11truein
\pdfpageheight=8.5truein
\pdfhorigin=1truein     % default value(?), but doesn't work without
\pdfvorigin=1truein     % default value(?), but doesn't work without
% A4
%\pdfpagewidth=297truemm % your milage may vary....
%\pdfpageheight=210truemm
%\pdfhorigin=1truein     % default value(?), but doesn't work without
%\pdfvorigin=1truein     % default value(?), but doesn't work without


\renewcommand{\familydefault}{\sfdefault}  
 
\input{seminar.bug} 
\input{seminar.bg2} % See the Seminar bugs list 
 
\slideframe{none} 
 
 
\usepackage{fancyhdr} 
 
% Headers and footers personalization using the `fancyhdr' package 
\fancyhf{} % Clear all fields 
\renewcommand{\headrulewidth}{0mm} 
\renewcommand{\footrulewidth}{0.1mm} 
 
\fancyfoot[L]{\tiny IETF 100} 
\fancyfoot[C]{\tiny TLS}
\fancyfoot[R]{\tiny \theslide} 
 
 
% To center horizontally the headers and footers (see seminar.bug) 
\renewcommand{\headwidth}{\textwidth} 

% To adjust the frame length to the header and footer ones 
\autoslidemarginstrue 
\pagestyle{fancy} 
 

\newcommand{\heading}[1]{% 
  \begin{center} 
    \large\bf 
    #1 
  \end{center} 
  \vspace{.4 in}} 


\begin{document}

\begin{slide}
\begin{center}
\vspace{.5 in}
\LARGE{{\bf}TLS 1.3\\{\small \verb^draft-ietf-tls-tls13-21^}}\\
\vspace{.2in}
\large{
\begin{tabular}{c}
Eric Rescorla\\
Mozilla\\
\url{ekr@rtfm.com}
\end{tabular}
}
\end{center}
\end{slide}

\centerslidesfalse 

\begin{slide}
  \heading{Agenda}

  \begin{itemize}
  \item Middlebox issues (PR\#1091)
  \item \verb^close_notify^ and half-close (PR\#1092)
  \item SNI and resumption(PR\#1080)
  \end{itemize}
\end{slide}


\begin{slide}
  \heading{Middlebox issues}

  \begin{itemize}
  \item Some middleboxes appear to be sad when you negotiate TLS 1.3
  \item Error rates (Firefox Beta versus Cloudflare)
    \begin{itemize}
    \item 2.2\% for TLS 1.2
    \item 3.9\% for TLS 1.3
    \end{itemize}
  \item This means you need fallback to deploy TLS 1.3
  \item Proposal: make TLS 1.3 look like TLS 1.2 resumption
  \end{itemize}
\end{slide}

\begin{slide}
  \heading{Emulate TLS 1.2 resumption part 1: Always}

  \begin{itemize}
  \item Move version negotiation entirely into \verb^supported_versions^
    \begin{itemize}
    \item ServerHello.version == 0x0303 (TLS 1.2)
    \end{itemize}
  \item Restore the missing \verb^session_id^ and compression fields in ServerHello
  \item Change the post-ServerHello record layer version to 0x0303
  \item Merge HRR and ServerHello into a single message with the semantics
    distinguished by a special ServerHello.Random value.
    \item Implementations MUST ignore ChangeCipherSpec during handshake
  \end{itemize}
\end{slide}

\begin{slide}
  \heading{Emulate TLS 1.2 resumption part 2: Compatibility Mode}

  \begin{itemize}
  \item The client sends a fake \verb^session_id^ and the server echoes it
  \item The server sends ChangeCipherSpec messages after
    ServerHello/HelloRetryRequest (so that the middlebox ignores any
    "encrypted" data afterwards), and the client sends ChangeCipherSpec
    after ClientHello.
    ClientHello
    \begin{itemize}
    \item Server's ChangeCipherSpec SHOULD be sent when the client sends
      the fake \verb^session_id^ (not in PR\#1091)
    \end{itemize}
  \end{itemize}
\end{slide}


\begin{slide}
  \heading{Issues Raised}

  \begin{itemize}
  \item Should we \textsf{only} have compatibility mode?
    \begin{itemize}
    \item We don't need this for TLS 1.3/QUIC or DTLS
    \item It's not \emph{entirely} clear we need the client-side CCS
    \item At some point we may be able to stop sending server-side CCS
    \end{itemize}

  \item Should we require the client to enforce CCS cardinality?
    \begin{itemize}
    \item Require CCS be present
    \item Require CCS to appear only once
    \item This complicates the implementation of the receiver
    \end{itemize}
  \end{itemize}
\end{slide}


\begin{slide}
\heading{Interlude: Chrome Data from David Benjamin}

\includegraphics[width=2.4in]{recount}

\vspace{1ex}

Firefox data hopefully coming soon

\end{slide}


\begin{slide}
\heading{Chrome initial draft 18 deployment}

\begin{itemize}
\item No evidence of TLS 1.3 ClientHello intolerance. \verb^supported_versions^ and GREASE did their job.
\item TLS 1.3 ServerHello was a very different story.

\item Successful handshakes to a TLS-1.3-capable service in Chrome beta:
  \begin{itemize}
  \item TLS 1.2 - 98.3\%
  \item Draft 18 - 92.3\%
  \end{itemize}

\item Middleboxes are intolerant to TLS 1.3 ServerHello. This violates TLS versioning rules: ClientHello is invariant, rest is version-specific.
  
\end{itemize}
\end{slide}

\begin{slide}
\heading{Middleboxes}


\begin{itemize}
\item TLS-terminating middleboxes generally work fine with TLS 1.3.
  \begin{itemize}
  \item  "Just" a server and client connected back-to-back. Server half negotiates TLS 1.2, client half only offers what it implements.
  \end{itemize}

\item Other middleboxes process TLS without terminating it. They then try to parse unknown version-specific messages and break.
  
\item This is an oversimplified picture. A lot of middleboxes are a mix of the two strategies.
  
\end{itemize}

\end{slide}


\begin{slide}
\heading{TLS 1.3 variants, round one}

\vspace{-.5in}
\begin{itemize}
\item  "Experiment" $\rightarrow$ PR 1091 without the record-layer version change.
\item  We tested what we could locally, then performed A/B tests in the wild (1-RTT).
\item  Successful handshakes to a TLS-1.3-capable service in Chrome beta:
  \begin{itemize}
   \item  TLS 1.2    - 99.2\%
   \item  Draft 18   - 95.8\%
   \item  PR 1051    - 90.3\%
   \item  Experiment - 98.2\%
   \item  Experiment w/o client session ID - 95.4\%
   \end{itemize}
\item Lots of user reports confirmed problems with each variant, including some for Experiment.
\end{itemize}
\end{slide}


\begin{slide}
\heading{TLS 1.3 variants, round two}

\begin{itemize}
\item Reproduced Experiment problems and changed record version for round 2.
\item Successful handshakes to a TLS-1.3-capable service in Chrome beta:
  \begin{itemize}
  \item TLS 1.2 - 98.6\%
  \item PR 1091 - 98.8\%
  \end{itemize}
\item Corroborated by HTTP-level metrics.

\item No user reports of problems thus far.
\end{itemize}
\end{slide}


\begin{slide}
\heading{\texttt{close\_notify} and half-close (PR\#1092)}

\vspace{-3ex}

\begin{itemize}
\item Right now \verb^close_notify^ is sorta full-close
  \begin{itemize}
  \item Receiver has to flush outstanding untransmitted data
  \item And immediately send \verb^close_notify^
  \end{itemize}

\item Not ideal
  \begin{itemize}
  \item Lots of implementations don't do this
  \item Data may already be in flight
  \item Reasons people may want half-close
  \item Not clear why it's there in the first place

  \end{itemize}

\item Proposal
  \begin{itemize}
  \item Allow implementations to keep sending after receiving \verb^close_notify^
  \item Backward compatible with previous behavior
  \end{itemize}
\end{itemize}
\end{slide}


\begin{slide}
  \heading{SNI and Resumption (PR\#1080)}

  \begin{itemize}
  \item RFC 6066 totally prohibits resuming with different SNIs
  \item Implementations aren't good about following this
  \item Proposal
    \begin{itemize}
    \item Client MUST only resume if SNI is in certificate
    \item Client SHOULD only resume if the SNI is the same
      \begin{itemize}
      \item No reason to think it will work anyway
      \end{itemize}
    \item Leaves the door open for the server to say that you can resume with different SNI
    \end{itemize}
  \item Not entirely clear how to analyze this
    \begin{itemize}
    \item But it looks like we already have these problems with existing implementations and HTTP coalescence
    \end{itemize}
  \end{itemize}
\end{slide}


\begin{slide}
\heading{Next step}

\begin{itemize}
\item Merge outstanding PRs (these and some editorial stuff)
\item Issue -22
\item Targeted WGLC?
\end{itemize}
\end{slide}

\end{document}


