\documentclass[helvetica]{seminar} 
\input{xy}
\xyoption{all}
\usepackage{graphicx} 
\usepackage{slidesec} 
\usepackage{url}
\usepackage[framemethod=TikZ]{mdframed}
\usepackage{color}
\usepackage[normalem]{ulem}  

\def\dash---{\unskip\kern.16667em---\penalty\exhyphenpenalty\hskip.16667em\ignorespaces}
\long\def\symbolfootnote[#1]#2{\begingroup%
\def\thefootnote{\fnsymbol{footnote}}\footnote[#1]{#2}\endgroup}

% to fix problems making landscape seminar pdfs
% Letter...
\pdfpagewidth=11truein
\pdfpageheight=8.5truein
\pdfhorigin=1truein     % default value(?), but doesn't work without
\pdfvorigin=1truein     % default value(?), but doesn't work without
% A4
%\pdfpagewidth=297truemm % your milage may vary....
%\pdfpageheight=210truemm
%\pdfhorigin=1truein     % default value(?), but doesn't work without
%\pdfvorigin=1truein     % default value(?), but doesn't work without


\renewcommand{\familydefault}{\sfdefault}  
 
\input{seminar.bug} 
\input{seminar.bg2} % See the Seminar bugs list 
 
\slideframe{none} 
 
 
\usepackage{fancyhdr} 
 
% Headers and footers personalization using the `fancyhdr' package 
\fancyhf{} % Clear all fields 
\renewcommand{\headrulewidth}{0mm} 
\renewcommand{\footrulewidth}{0.1mm} 
 
\fancyfoot[L]{\tiny IETF 100} 
\fancyfoot[C]{\tiny TLS}
\fancyfoot[R]{\tiny \theslide} 
 
 
% To center horizontally the headers and footers (see seminar.bug) 
\renewcommand{\headwidth}{\textwidth} 

% To adjust the frame length to the header and footer ones 
\autoslidemarginstrue 
\pagestyle{fancy} 
 

\newcommand{\heading}[1]{% 
  \begin{center} 
    \large\bf 
    #1 
  \end{center} 
  \vspace{.4 in}} 


\begin{document}

\begin{slide}
\begin{center}
\vspace{.5 in}
\LARGE{{\bf}DTLS 1.3\\{\small \verb^draft-ietf-tls-dtls13-02^}}\\
\vspace{.2in}
\small{
\begin{tabular}{c c c}
\textbf{Eric Rescorla} & Hannes Tschofenig & Nagendra Modadugu\\
Mozilla& Arm Limited & Google \\ 
\url{ekr@rtfm.com} & \url{hannes.tschofenig@arm.com} & \url{nagendra@cs.stanford.edu} \\
\end{tabular}
}
\end{center}
\end{slide}

\centerslidesfalse 


\begin{slide}
  \heading{Changes since -01}

  \begin{itemize}
  \item Short record headers
  \item Empty ACK and clarified ACK rules
  \item Reintroduce KeyUpdate because it now works with ACKs
  \end{itemize}

\end{slide}

\begin{slide}
  \heading{Short headers 1: Shorten DTLSCiphertext}

\begin{verbatim}
  struct {
      ContentType opaque_type = 23; /* application_data */
      uint32 epoch_and_sequence;
      uint16 length;
      opaque encrypted_record[length];
  } DTLSCiphertext;
\end{verbatim}

  \begin{itemize}
  \item New format for DTLS encrypted traffic
  \item Can be used like DTLS 1.2 DTLSCiphertext
  \item Keyed on version negotiation as expected
  \end{itemize}
\end{slide}


\begin{slide}
\heading{Short headers 2: Special DTLSShortCiphertext}

\begin{verbatim}
    struct {
      uint16 short_epoch_and_sequence;  // 001ESSSS SSSSSSSS
      opaque encrypted_record[remainder_of_datagram];
    } DTLSShortCiphertext;
\end{verbatim}

\begin{itemize}
\item E == truncated epoch
\item S == truncated sequence
\item Can \emph{only} be used
  \begin{itemize}
  \item With 1-RTT data
  \item When you have one record per packet
  \end{itemize}
\end{itemize}
\end{slide}


\begin{slide}
  \heading{Reconstructing the epoch/sequence}

\begin{verbatim}
   Sequence reconstruction (same as QUIC):
     Use full sequence number closest to seq of the
     highest successfully deprotected record.
 
   Epoch:
     If epoch low-order bits match, just decrypt
     If epoch low-order bits match, use the epoch
     which provides the closest reconstructed
     sequence number.
\end{verbatim}
\end{slide}


\begin{slide}
  \heading{Empty Acks}

  \begin{itemize}
  \item Sometimes you can't decrypt part of a flight
    \begin{itemize}
    \item E.g., you get EE before SH
    \end{itemize}
  \item In these cases you can't ACK
    \begin{itemize}
    \item And rely on the retransmit timeout
    \end{itemize}
  \item In this case you should send an empty ACK
    \begin{itemize}
    \item This shortcuts the retransmit
    \end{itemize}
  \end{itemize}
\end{slide}


\begin{slide}
  \heading{KeyUpdate}

  \begin{itemize}
  \item Restored KeyUpdate mechanism
    \begin{itemize}
    \item Works just like TLS 1.3
    \item With ACK, this works properly
    \end{itemize}

  \item When can you send with the new key?
    \begin{itemize}
    \item Currently right away
      \begin{itemize}
      \item What about reordering?
      \item ... trial decryption or drop the packet        
      \end{itemize}

    \item Alternative: can't send until ACKed
      \begin{itemize}
      \item Different than with TLS 1.3
      \item Arguably less complex (though complexity is on updater)
      \end{itemize}
    \end{itemize}
  \end{itemize}

\end{slide}


\begin{slide}
\heading{Remaining Open issues: None!}

\begin{itemize}
\item WGLC?
\end{itemize}
\end{slide}


\end{document}
