\documentclass[helvetica]{seminar} 
\input{xy}
\xyoption{all}
\usepackage{graphicx} 
\usepackage{slidesec} 
\usepackage{url}
\usepackage[framemethod=TikZ]{mdframed}
\usepackage{color}
\usepackage[normalem]{ulem}  

\def\dash---{\unskip\kern.16667em---\penalty\exhyphenpenalty\hskip.16667em\ignorespaces}
\long\def\symbolfootnote[#1]#2{\begingroup%
\def\thefootnote{\fnsymbol{footnote}}\footnote[#1]{#2}\endgroup}

% to fix problems making landscape seminar pdfs
% Letter...
\pdfpagewidth=11truein
\pdfpageheight=8.5truein
\pdfhorigin=1truein     % default value(?), but doesn't work without
\pdfvorigin=1truein     % default value(?), but doesn't work without
% A4
%\pdfpagewidth=297truemm % your milage may vary....
%\pdfpageheight=210truemm
%\pdfhorigin=1truein     % default value(?), but doesn't work without
%\pdfvorigin=1truein     % default value(?), but doesn't work without


\renewcommand{\familydefault}{\sfdefault}  
 
\input{seminar.bug} 
\input{seminar.bg2} % See the Seminar bugs list 
 
\slideframe{none} 
 
 
\usepackage{fancyhdr} 
 
% Headers and footers personalization using the `fancyhdr' package 
\fancyhf{} % Clear all fields 
\renewcommand{\headrulewidth}{0mm} 
\renewcommand{\footrulewidth}{0.1mm} 
 
\fancyfoot[L]{\tiny IETF 100} 
\fancyfoot[C]{\tiny TLS}
\fancyfoot[R]{\tiny \theslide} 
 
 
% To center horizontally the headers and footers (see seminar.bug) 
\renewcommand{\headwidth}{\textwidth} 

% To adjust the frame length to the header and footer ones 
\autoslidemarginstrue 
\pagestyle{fancy} 
 

\newcommand{\heading}[1]{% 
  \begin{center} 
    \large\bf 
    #1 
  \end{center} 
  \vspace{.4 in}} 


\begin{document}

\begin{slide}
\begin{center}
\vspace{.5 in}
\LARGE{{\bf}Connection ID\\{\small \verb^draft-rescorla-tls-connection-id-02^}}\\
\vspace{.2in}
{\small
\begin{tabular}{c c}
\textbf{Eric Rescorla} & Hannes Tschofenig \\
Mozilla& Arm Limited \\ 
  \url{ekr@rtfm.com} & \url{hannes.tschofenig@arm.com} \\
  \\
  Thomas Fossati & Tobias Gondrom \\
  Nokia & Huawei \\
  thomas.fossati@nokia.com & tobias.gondrom@gondrom.org \\
  \end{tabular}

}

\end{center}
\end{slide}

\centerslidesfalse 

\begin{slide}
  \heading{Recap from last time}

  \begin{itemize}
  \item Lack of Connection IDs clearly a problem for NATs/IoT, etc.
  \item Connection IDs are also a clear privacy problem
    \begin{itemize}
    \item Lots of proposals for how to do privacy preserving Conn IDs
    \item ... but they're complicated and none of them seem totally baked
    \end{itemize}

  \item Proposal: use a fixed connection ID for now
    \begin{itemize}
    \item In an extension
    \item We can always replace it later
    \end{itemize}
    
  \item This got pulled out of DTLS and into its own draft
  \end{itemize}
\end{slide}


\begin{slide}
  \heading{Basic idea}
  
\begin{itemize}
\item IDs are used if client offers and server answers
  \begin{itemize}
    \item On all (non-0RTT)? encrypted records
  \end{itemize}
\item Each side \emph{sends} with the other's ID
  \begin{itemize}
    \item Because IDs are unframed, 0-length IDs are just omitted
  \end{itemize}
\end{itemize}
\end{slide}

\begin{slide}
  \heading{DTLS 1.2}

\begin{verbatim}
  struct {
     ContentType type;
     ProtocolVersion version;
     uint16 epoch;
     uint48 sequence_number;
     opaque cid[cid_length];               // New field
     uint16 length;
     select (CipherSpec.cipher_type) {
        case block:  GenericBlockCipher;
        case aead:   GenericAEADCipher;
     } fragment;
  } DTLSCiphertext;
\end{verbatim}
\end{slide}

\begin{slide}
  \heading{DTLS 1.3\symbolfootnote[1]{Not in the draft. Ugh.}}

  \vspace{-5ex}
  \small{
\begin{verbatim}
  struct {
      ContentType opaque_type = 23; /* application_data */
      uint32 epoch_and_sequence;
      opaque cid[cid_length];               // New field
      uint16 length;
      opaque encrypted_record[length];
  } DTLSCiphertext;

  struct {
    uint16 short_epoch_and_sequence;  // 001ESSSS SSSSSSSS
    opaque encrypted_record[remainder_of_datagram];
    opaque cid[cid_length];               // New field
  } DTLSShortCiphertext;


\end{verbatim}
}
\end{slide}


\begin{slide}
  \heading{Connection ID Update (TLS 1.3 only)}

\begin{verbatim}
   enum {
       cid_immediate(0), cid_spare(1), (255)
   } ConnectionIdUsage;

   struct {
       opaque cid<0..2^8-1>;
       ConnectionIdUsage usage;
   } NewConnectionId;
\end{verbatim}

  \begin{itemize}
    \item \verb^cid_immediate^ means ``delete all your older conn ids''
    \item \verb^cid_spare^ means ``add to the valid conn ids''
    \item I am not sure this is ideal
  \end{itemize}

\end{slide}


\begin{slide}
  \heading{Open Issues}

  \begin{itemize}
  \item Do we need a way to tell if a CID is present
    \begin{itemize}
    \item to deal with servers which have both CID and non-CID connections
    \end{itemize}
  \item Do we need CID update for TLS 1.2 (how?)
  \item The record sequence number allows cross-CID linkage
    \begin{itemize}
    \item Solution: adopt the technique we used for QUIC of predictable jumps
    \end{itemize}
  \end{itemize}
\end{slide}


\begin{slide}
\heading{Other issues? WG adoption?}

\end{slide}


\begin{slide}
  \heading{Options for TLS 1.2 Post-Handshake CID Refresh}

  \begin{itemize}
  \item Do nothing
  \item TLS 1.2 renegotiation
  \item Port over TLS 1.3 post-handshake messaging
    \begin{itemize}
    \item I think we'd need to deprecate renegotiation
    \end{itemize}
  \end{itemize}

\end{slide}


\end{document}


