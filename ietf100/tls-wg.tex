\documentclass[helvetica]{seminar} 
\input{xy}
\xyoption{all}
\usepackage{graphicx} 
\usepackage{slidesec} 
\usepackage{url}
\usepackage[framemethod=TikZ]{mdframed}
\usepackage{color}
\usepackage[normalem]{ulem}  

\def\dash---{\unskip\kern.16667em---\penalty\exhyphenpenalty\hskip.16667em\ignorespaces}
\long\def\symbolfootnote[#1]#2{\begingroup%
\def\thefootnote{\fnsymbol{footnote}}\footnote[#1]{#2}\endgroup}

% to fix problems making landscape seminar pdfs
% Letter...
\pdfpagewidth=11truein
\pdfpageheight=8.5truein
\pdfhorigin=1truein     % default value(?), but doesn't work without
\pdfvorigin=1truein     % default value(?), but doesn't work without
% A4
%\pdfpagewidth=297truemm % your milage may vary....
%\pdfpageheight=210truemm
%\pdfhorigin=1truein     % default value(?), but doesn't work without
%\pdfvorigin=1truein     % default value(?), but doesn't work without


\renewcommand{\familydefault}{\sfdefault}  
 
\input{seminar.bug} 
\input{seminar.bg2} % See the Seminar bugs list 
 
\slideframe{none} 
 
 
\usepackage{fancyhdr} 
 
% Headers and footers personalization using the `fancyhdr' package 
\fancyhf{} % Clear all fields 
\renewcommand{\headrulewidth}{0mm} 
\renewcommand{\footrulewidth}{0.1mm} 
 
\fancyfoot[L]{\tiny IETF 100} 
\fancyfoot[C]{\tiny TLS}
\fancyfoot[R]{\tiny \theslide} 
 
 
% To center horizontally the headers and footers (see seminar.bug) 
\renewcommand{\headwidth}{\textwidth} 

% To adjust the frame length to the header and footer ones 
\autoslidemarginstrue 
\pagestyle{fancy} 
 

\newcommand{\heading}[1]{% 
  \begin{center} 
    \large\bf 
    #1 
  \end{center} 
  \vspace{.4 in}} 


\begin{document}

\begin{slide}
\begin{center}
\vspace{.5 in}
\LARGE{{\bf}TLS 1.3\\{\small \verb^draft-ietf-tls-tls13-21^}}\\
\vspace{.2in}
\large{
\begin{tabular}{c}
Eric Rescorla\\
Mozilla\\
\url{ekr@rtfm.com}
\end{tabular}
}
\end{center}
\end{slide}

\centerslidesfalse 

\begin{slide}
  \heading{Agenda}

  \begin{itemize}
  \item Middlebox issues (PR\#1091)
  \item \verb^close_notify^ and half-close (PR\#1092)
  \item SNI and resumption(PR\#1080)
  \end{itemize}
\end{slide}


\begin{slide}
  \heading{Middlebox issues}

  \begin{itemize}
  \item Some middleboxes appear to be sad when you negotiate TLS 1.3
  \item Error rates (Firefox Nightly versus Google)
    \begin{itemize}
    \item X\% for TLS 1.2
    \item X\% for TLS 1.3
    \end{itemize}
  \item This means you need fallback to deploy TLS 1.3
  \item Proposal: make TLS 1.3 look like TLS 1.2 resumption
  \end{itemize}
\end{slide}

\begin{slide}
  \heading{Emulate TLS 1.2 resumption part 1: Always}

  \begin{itemize}
  \item Move version negotiation entirely into \verb^supported_versions^
    \begin{itemize}
    \item ServerHello.version == 0x0303 (TLS 1.2)
    \end{itemize}
  \item Restore the missing \verb^session_id^ and compression fields in ServerHello
  \item Change the post-ServerHello record layer version to 0x0303
  \item Merge HRR and ServerHello into a single message with the semantics
    distinguished by a special ServerHello.Random value.
    \item Implementations MUST ignore ChangeCipherSpec during handshake
  \end{itemize}
\end{slide}

\begin{slide}
  \heading{Emulate TLS 1.2 resumption part 2: Compatibility Mode}

  \begin{itemize}
  \item The client sends a fake \verb^session_id^ and the server echoes it
  \item The server sends ChangeCipherSpec messages after
    ServerHello/HelloRetryRequest (so that the middlebox ignores any
    "encrypted" data afterwards), and the client sends ChangeCipherSpec
    after ClientHello.  (so that the middlebox ignores any "encrypted"
    data afterwards), and the client sends ChangeCipherSpec after
    ClientHello
    \begin{itemize}
    \item Server's ChangeCipherSpec SHOULD be sent when the client sends
      the fake \verb^session_id^ (not in PR\#1091)
    \end{itemize}
  \end{itemize}
\end{slide}


\begin{slide}
  \heading{Issues Raised}

  \begin{itemize}
  \item Should we \textsf{only} have compatibility mode?
    \begin{itemize}
    \item We don't need this for TLS 1.3/QUIC or DTLS
    \item It's not \emph{entirely} clear we need the client-side CCS
    \item At some point we may be able to stop sending server-side CCS
    \end{itemize}

  \item Should we require the client to enforce CCS cardinality?
    \begin{itemize}
    \item Require CCS be present
    \item Require CCS to appear only once
    \item This complicates the implementation of the receiver
    \end{itemize}
  \end{itemize}
\end{slide}


\begin{slide}
\heading{Interlude: Chrome from David Benjamin}

\end{slide}

\begin{slide}
\heading{Preliminary Firefox Data}

\end{slide}


\begin{slide}
\heading{\texttt{close\_notify} and half-close (PR\#1092)}

\begin{itemize}
\item Right now \verb^close_notify^ is sorta full-close
  \begin{itemize}
  \item Receiver has to flush outstanding untransmitted data
  \item And immediately send \verb^close_notify^
  \end{itemize}

\item Not ideal
  \begin{itemize}
  \item Lots of implementations don't do this
  \item Data may already be in flight
  \item Reasons people may want half-close
  \item Not clear why it's there in the first place

  \end{itemize}

\item Proposal
  \begin{itemize}
  \item Allow implementations to keep sending after receiving \verb^close_notify^
  \item Backward compatible with previous behavior
  \end{itemize}
\end{itemize}
\end{slide}


\begin{slide}
  \heading{SNI and Resumption (PR\#1080)}

  \begin{itemize}
  \item RFC 6066 totally prohibits resuming with different SNIs
  \item Implementations aren't good about following this
  \item Proposal
    \begin{itemize}
    \item Client MUST only resume if SNI is in certificate
    \item Client SHOULD only resume if the SNI is the same
      \begin{itemize}
      \item No reason to think it will work anyway
      \end{itemize}
    \item Leaves the door open for the server to say that you can resume with different SNI
    \end{itemize}
  \item Not entirely clear how to analyze this
    \begin{itemize}
    \item But it looks like we already have these problems with existing implementations and HTTP coalescence
    \end{itemize}
  \end{itemize}
\end{slide}


\centerslidestrue

\begin{slide}
\begin{center}
\vspace{.5 in}
\LARGE{{\bf}DTLS 1.3\\{\small \verb^draft-ietf-tls-dtls13-02^}}\\
\vspace{.2in}
\small{
\begin{tabular}{c c c}
\textbf{Eric Rescorla} & Hannes Tschofenig & Nagendra Modadugu\\
Mozilla& Arm Limited & Google \\ 
\url{ekr@rtfm.com} & \url{hannes.tschofenig@arm.com} & \url{nagendra@cs.stanford.edu} \\
\end{tabular}
}
\end{center}
\end{slide}

\centerslidesfalse 


\begin{slide}
  \heading{Changes since -01}

  \begin{itemize}
  \item Short record headers
  \item Empty ACK and clarified ACK rules
  \item Reintroduce KeyUpdate because it now works with ACKs
  \end{itemize}

\end{slide}

\begin{slide}
  \heading{Short headers 1: Shorten DTLSCiphertext}

\begin{verbatim}
  struct {
      ContentType opaque_type = 23; /* application_data */
      uint32 epoch_and_sequence;
      uint16 length;
      opaque encrypted_record[length];
  } DTLSCiphertext;
\end{verbatim}

  \begin{itemize}
  \item New format for DTLS encrypted traffic
  \item Can be used like DTLS 1.2 DTLSCiphertext
  \item Keyed on version negotiation as expected
  \end{itemize}
\end{slide}


\begin{slide}
\heading{Short headers 2: Special DTLSShortCiphertext}

\begin{verbatim}
    struct {
      uint16 short_epoch_and_sequence;  // 001ESSSS SSSSSSSS
      opaque encrypted_record[remainder_of_datagram];
    } DTLSShortCiphertext;
\end{verbatim}

\begin{itemize}
\item E == truncated epoch
\item S == truncated sequence
\item Can \emph{only} be used
  \begin{itemize}
  \item With 1-RTT data
  \item When you have one record per packet
  \end{itemize}
\end{itemize}
\end{slide}


\begin{slide}
  \heading{Reconstructing the epoch/sequence}

\begin{verbatim}
   Sequence reconstruction (same as QUIC):
     Use full sequence number closest to seq of the
     highest successfully deprotected record.
 
   Epoch:
     If epoch low-order bits match, just decrypt
     If epoch low-order bits match, use the epoch
     which provides the closest reconstructed
     sequence number.
\end{verbatim}
\end{slide}


\begin{slide}
  \heading{Empty Acks}

  \begin{itemize}
  \item Sometimes you can't decrypt part of a flight
    \begin{itemize}
    \item E.g., you get EE before SH
    \end{itemize}
  \item In these cases you can't ACK
    \begin{itemize}
    \item And rely on the retransmit timeout
    \end{itemize}
  \item In this case you should send an empty ACK
    \begin{itemize}
    \item This shortcuts the retransmit
    \end{itemize}
  \end{itemize}
\end{slide}


\begin{slide}
  \heading{KeyUpdate}

  \begin{itemize}
  \item Restored KeyUpdate mechanism
    \begin{itemize}
    \item Works just like TLS 1.3
    \item With ACK, this works properly
    \end{itemize}

  \item When can you send with the new key?
    \begin{itemize}
    \item Currently right away
      \begin{itemize}
      \item What about reordering?
      \item ... trial decryption or drop the packet        
      \end{itemize}

    \item Alternative: can't send until ACKed
      \begin{itemize}
      \item Different than with TLS 1.3
      \item Arguably less complex (though complexity is on updater)
      \end{itemize}
    \end{itemize}
  \end{itemize}

\end{slide}


\begin{slide}
\heading{Remaining Open issues: None!}

\begin{itemize}
\item WGLC?
\end{itemize}
\end{slide}



\centerslidestrue

\begin{slide}
\begin{center}
\vspace{.5 in}
\LARGE{{\bf}Connection ID\\{\small \verb^draft-rescorla-tls-connection-id-02^}}\\
\vspace{.2in}
{\small
\begin{tabular}{c c}
\textbf{Eric Rescorla} & Hannes Tschofenig \\
Mozilla& Arm Limited \\ 
  \url{ekr@rtfm.com} & \url{hannes.tschofenig@arm.com} \\
  \\
  Thomas Fossati & Tobias Gondrom \\
  Nokia & Huawei \\
  thomas.fossati@nokia.com & tobias.gondrom@gondrom.org \\
  \end{tabular}

}

\end{center}
\end{slide}

\centerslidesfalse 

\begin{slide}
  \heading{Recap from last time}

  \begin{itemize}
  \item Lack of Connection IDs clearly a problem for NATs/IoT, etc.
  \item Connection IDs are also a clear privacy problem
    \begin{itemize}
    \item Lots of proposals for how to do privacy preserving Conn IDs
    \item ... but they're complicated and none of them seem totally baked
    \item This seems like less of a privacy problem than with browsers (QUIC)     
    \end{itemize}

  \item Proposal: use a fixed connection ID for now
    \begin{itemize}
    \item In an extension
    \item We can always replace it later
    \end{itemize}
    
  \item This got pulled out of DTLS and into its own draft
  \end{itemize}
\end{slide}


\begin{slide}
  \heading{Basic idea}
  
\begin{itemize}
\item IDs are used if client offers and server answers
  \begin{itemize}
    \item On all (non-0RTT)? encrypted records
  \end{itemize}
\item Each side \emph{sends} with the other's ID
  \begin{itemize}
    \item Because IDs are unframed, 0-length IDs are just omitted
  \end{itemize}
\end{itemize}
\end{slide}

\begin{slide}
  \heading{DTLS 1.2}

\begin{verbatim}
  struct {
     ContentType type;
     ProtocolVersion version;
     uint16 epoch;
     uint48 sequence_number;
     opaque cid[cid_length];               // New field
     uint16 length;
     select (CipherSpec.cipher_type) {
        case block:  GenericBlockCipher;
        case aead:   GenericAEADCipher;
     } fragment;
  } DTLSCiphertext;
\end{verbatim}
\end{slide}

\begin{slide}
  \heading{DTLS 1.3\symbolfootnote[1]{Not in the draft. Ugh.}}

  \vspace{-5ex}
  \small{
\begin{verbatim}
  struct {
      ContentType opaque_type = 23; /* application_data */
      uint32 epoch_and_sequence;
      opaque cid[cid_length];               // New field
      uint16 length;
      opaque encrypted_record[length];
  } DTLSCiphertext;

  struct {
    uint16 short_epoch_and_sequence;  // 001ESSSS SSSSSSSS
    opaque encrypted_record[remainder_of_datagram];
    opaque cid[cid_length];               // New field
  } DTLSShortCiphertext;


\end{verbatim}
}
\end{slide}


\begin{slide}
  \heading{Connection ID Update (TLS 1.3 only)}

\begin{verbatim}
   enum {
       cid_immediate(0), cid_spare(1), (255)
   } ConnectionIdUsage;

   struct {
       opaque cid<0..2^8-1>;
       ConnectionIdUsage usage;
   } NewConnectionId;
\end{verbatim}

  \begin{itemize}
    \item \verb^cid_immediate^ means ``delete all your older conn ids''
    \item \verb^cid_spare^ means ``add to the valid conn ids''
    \item I am not sure this is ideal
  \end{itemize}

\end{slide}



\begin{slide}
  \heading{Open Issues}

  \begin{itemize}
  \item Do we need a way to tell if a CID is present
    \begin{itemize}
    \item to deal with servers which have both CID and non-CID connections
    \end{itemize}
  \item Do we need CID update for TLS 1.2 (how?)
  \item The record sequence number allows cross-CID linkage
    \begin{itemize}
    \item Solution: adopt the technique we used for QUIC of predictable jumps
    \end{itemize}
  \end{itemize}
\end{slide}


\begin{slide}
\heading{Other issues? WG adoption?}

\end{slide}

\end{document}


