\documentclass[helvetica]{seminar} 
\input{xy}
\xyoption{all}
\usepackage{graphicx} 
\usepackage{slidesec} 
\usepackage{url}
\usepackage[framemethod=TikZ]{mdframed}
\usepackage{color}
\usepackage[normalem]{ulem}  

\def\dash---{\unskip\kern.16667em---\penalty\exhyphenpenalty\hskip.16667em\ignorespaces}
\long\def\symbolfootnote[#1]#2{\begingroup%
\def\thefootnote{\fnsymbol{footnote}}\footnote[#1]{#2}\endgroup}

% to fix problems making landscape seminar pdfs
% Letter...
\pdfpagewidth=11truein
\pdfpageheight=8.5truein
\pdfhorigin=1truein     % default value(?), but doesn't work without
\pdfvorigin=1truein     % default value(?), but doesn't work without
% A4
%\pdfpagewidth=297truemm % your milage may vary....
%\pdfpageheight=210truemm
%\pdfhorigin=1truein     % default value(?), but doesn't work without
%\pdfvorigin=1truein     % default value(?), but doesn't work without


\renewcommand{\familydefault}{\sfdefault}  
 
\input{seminar.bug} 
\input{seminar.bg2} % See the Seminar bugs list 
 
\slideframe{none} 
 
 
\usepackage{fancyhdr} 
 
% Headers and footers personalization using the `fancyhdr' package 
\fancyhf{} % Clear all fields 
\renewcommand{\headrulewidth}{0mm} 
\renewcommand{\footrulewidth}{0.1mm} 
 
\fancyfoot[L]{\tiny IETF 101} 
\fancyfoot[C]{\tiny TLS}
\fancyfoot[R]{\tiny \theslide} 
 
 
% To center horizontally the headers and footers (see seminar.bug) 
\renewcommand{\headwidth}{\textwidth} 

% To adjust the frame length to the header and footer ones 
\autoslidemarginstrue 
\pagestyle{fancy} 
 

\newcommand{\heading}[1]{% 
  \begin{center} 
    \large\bf 
    #1 
  \end{center} 
  \vspace{.4 in}} 


\begin{document}

\begin{slide}
\begin{center}
\vspace{.5 in}
\LARGE{{\bf}Connection ID\\{\small \verb^draft-rescorla-tls-connection-id-02^}}\\
\vspace{.2in}
{\small
\begin{tabular}{c c}
\textbf{Eric Rescorla} & Hannes Tschofenig \\
Mozilla& Arm Limited \\ 
  \url{ekr@rtfm.com} & \url{hannes.tschofenig@arm.com} \\
  \\
  Thomas Fossati & Tobias Gondrom \\
  Nokia & Huawei \\
  thomas.fossati@nokia.com & tobias.gondrom@gondrom.org \\
  \end{tabular}

}

\end{center}
\end{slide}

\centerslidesfalse 

\begin{slide}
  \heading{Reminder: IETF 101}

  \begin{itemize}
  \item Explicitly mark the presence of CID
    \begin{itemize}
    \item CID length is implicit
    \end{itemize}
  \item Split work into DTLS 1.2 and DTLS 1.3
    \begin{itemize}
    \item Conn-ID document becomes DTLS 1.2 only
    \item Add conn ID to DTLS 1.3 main draft
    \item Both use the ``connection\_id'' extension
    \end{itemize}
  \end{itemize}
\end{slide}

\begin{slide}
  \heading{DTLS 1.2: CID ``present'' marking}

  \begin{itemize}
  \item Two options presented
    \begin{itemize}
    \item Bit in the length field
    \item Shadow content type values
    \end{itemize}

  \item Concern about compat of length field $\rightarrow$ shadow content types
  \item Spec defines four new types (alert, handshake, application, heartbeat)
    \begin{itemize}
    \item This does seem like a lot
    \end{itemize}
  \end{itemize}
\end{slide}

\begin{slide}
  \heading{Alternate Design (Thomson)}

  \begin{itemize}
  \item Allocate one extra code point (``encrypted content type + CID'')
    \begin{itemize}
    \item This would use TLS 1.3-style content type encryption
    \item + Free padding
    \item So no need for more code points
    \end{itemize}    
  \end{itemize}
\end{slide}

\begin{slide}
  \heading{Alternate Alternate Design}

\begin{verbatim}
   struct {
        ContentType type = 25; // new type valie
        ProtocolVersion version;
        uint16 epoch;
        uint48 sequence_number;
        ContentType true_content_type;
        opaque cid[cid_length];               // New field
        uint16 length;
        select (CipherSpec.cipher_type) {
            case block:  GenericBlockCipher;
            case aead:   GenericAEADCipher;
        } fragment;
   } DTLSCiphertext;
\end{verbatim}  
\end{slide}

\begin{slide}
  \heading{Other ideas?}
\end{slide}

\begin{slide}
\heading{DTLS 1.3: Unified Packet Format}

{\scriptsize
\begin{verbatim}
  0 1 2 3 4 5 6 7
 +-+-+-+-+-+-+-+-+
 |0|0|1|C|L|X|X|X|
 +-+-+-+-+-+-+-+-+
 |Ep.| 14 bit    |   Legend:
 +-+-+           |
 |Sequence Number|   Ep. - Epoch
 +-+-+-+-+-+-+-+-+   C   - CID present
 | Connection ID |   L   - Length present
 | (if any,      |   X   - Reserved
 /  length as    /
 |  negotiated)  |
 +-+-+-+-+-+-+-+-+
 | 16 bit Length |
 | (if present)  |
 +-+-+-+-+-+-+-+-+
\end{verbatim}
}
\end{slide}

\begin{slide}
\heading{Examples}

\vspace{-.3in}
{\scriptsize
\begin{verbatim}
   0 1 2 3 4 5 6 7        0 1 2 3 4 5 6 7
  +-+-+-+-+-+-+-+-+     +-+-+-+-+-+-+-+-+
  |0|0|1|C|L|X|X|X|     |0|0|1|0|0|X|X|X|
  +-+-+-+-+-+-+-+-+     +-+-+-+-+-+-+-+-+
  |E|E| 14 bit    |     |E|E| 14 bit    |
  +-+-+           |     +-+-+           |
  |Sequence Number|     |Sequence Number|
  +-+-+-+-+-+-+-+-+     +-+-+-+-+-+-+-+-+
  |               |     |               |
  / Connection ID /     |   Encrypted   |
  |               |     /   Record      /
  +-+-+-+-+-+-+-+-+     |               |
  |   16 bit      |     +-+-+-+-+-+-+-+-+
  |   Length      |
  +-+-+-+-+-+-+-+-+
  |               |
  |  Encrypted    |
  /  Record       /
  |               |
  +-+-+-+-+-+-+-+-+
\end{verbatim}
  }
\end{slide}

\end{document}


