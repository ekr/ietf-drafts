\documentclass[helvetica]{seminar} 
\input{xy}
\xyoption{all}
\usepackage{graphicx} 
\usepackage{slidesec} 
\usepackage{url}
\usepackage[framemethod=TikZ]{mdframed}
\usepackage{color}
\usepackage[normalem]{ulem}  

\def\dash---{\unskip\kern.16667em---\penalty\exhyphenpenalty\hskip.16667em\ignorespaces}
\long\def\symbolfootnote[#1]#2{\begingroup%
\def\thefootnote{\fnsymbol{footnote}}\footnote[#1]{#2}\endgroup}

% to fix problems making landscape seminar pdfs
% Letter...
\pdfpagewidth=11truein
\pdfpageheight=8.5truein
\pdfhorigin=1truein     % default value(?), but doesn't work without
\pdfvorigin=1truein     % default value(?), but doesn't work without
% A4
%\pdfpagewidth=297truemm % your milage may vary....
%\pdfpageheight=210truemm
%\pdfhorigin=1truein     % default value(?), but doesn't work without
%\pdfvorigin=1truein     % default value(?), but doesn't work without


\renewcommand{\familydefault}{\sfdefault}  
 
\input{seminar.bug} 
\input{seminar.bg2} % See the Seminar bugs list 
 
\slideframe{none} 
 
 
\usepackage{fancyhdr} 
 
% Headers and footers personalization using the `fancyhdr' package 
\fancyhf{} % Clear all fields 
\renewcommand{\headrulewidth}{0mm} 
\renewcommand{\footrulewidth}{0.1mm} 
 
\fancyfoot[L]{\tiny IETF 98} 
\fancyfoot[C]{\tiny TLS}
\fancyfoot[R]{\tiny \theslide} 
 
 
% To center horizontally the headers and footers (see seminar.bug) 
\renewcommand{\headwidth}{\textwidth} 

% To adjust the frame length to the header and footer ones 
\autoslidemarginstrue 
\pagestyle{fancy} 
 

\newcommand{\heading}[1]{% 
  \begin{center} 
    \large\bf 
    #1 
  \end{center} 
  \vspace{.4 in}} 



\begin{document}

\begin{slide}
\begin{center}
\vspace{.5 in}
\LARGE{{\bf}DTLS 1.3\\{\small \verb^draft-rescorla-tls-dtls13-01^}}\\
\vspace{.2in}
\large{
\begin{tabular}{c c c}
\textbf{Eric Rescorla} & Hannes Tschofenig & Nagendra Modadugu \\
Mozilla & ARM & Google \\
\end{tabular}
}
\end{center}
\end{slide}

\centerslidesfalse 

\begin{slide}
\heading{Overview}

\begin{itemize}
\item DTLS version of TLS 1.3
\item Still presented as a delta from TLS 1.3
\item Some improvements/cleanup
\item Partly informed by early implementation experience
\end{itemize}

\end{slide}

\begin{slide}
\heading{Document Status}

\begin{itemize}
\item Individual submission
\item Currently in call for acceptance
\item Here to talk about the draft...  
\end{itemize}
\end{slide}


\begin{slide}
\heading{Issue\#2: ACKs}

\begin{itemize}
\item DTLS historically used an implicit ACK
  \begin{itemize}
  \item Receiving the start of the next flight means the flight was received
  \end{itemize}

\item Simple (but also simpleminded)
  \begin{itemize}
  \item Slightly tricky to implement
  \item Gives limited congestion feedback
  \item Handles single-packet loss badly
  \end{itemize}

\item Interacts badly with some TLS 1.3 features (like NST)
\item Solution: introduce an explicit ACK
\end{itemize}
\end{slide}


\begin{slide}
\heading{Where should we ACK?}

\begin{itemize}
\item Every flight
\item Just at the end of things that aren't explicitly acknowledged
  \begin{itemize}
  \item Client Finished
  \item NewSessionTicket
  \end{itemize}

\item Proposal: allow ACKs at any time
  \begin{itemize}
  \item This allows partial retransmit of flights (if we SACK)
  \item Also just means one trigger for state machine evolution
  \end{itemize}
\end{itemize}
\end{slide}


\begin{slide}
\heading{Strawman ACK format (not what's in the draft)}

{\small
\begin{verbatim}
struct AcknowledgedMessage {
   uint16 message_seq;
   uint32 timestamp; 
};

struct {
   AcknowledgedMessage messages<2..2^16-2>;
} DtlsAck;
\end{verbatim}
}

\begin{itemize}
\item This is a compromise between ``lots of data'' and ``convenient''
\item We could also include the DTLS records for more path feedback
\end{itemize}

\end{slide}

\begin{slide}
\heading{What epoch should ACKs be encrypted under?}

\begin{itemize}
\item Issue here is key transitions
\item E.g., ACK of the client Finished
  \begin{itemize}
  \item Natural to match the epoch of what you're ACKing
  \item But this may mean you have two keys
  \end{itemize}
\end{itemize}
\end{slide}

\begin{slide}
\heading{Key Update}

\begin{itemize}
\item Key Update in TLS 1.3 is unreliable
  \begin{itemize}
  \item This means new epoch records may appear before KeyUpdate
  \end{itemize}
\item Current draft just omits KeyUpdate
  \begin{itemize}
  \item KeyUpdate from one side triggers another
  \item Only one unacknowledged KeyUpdate allowed outstanding
  \item Can't unilaterally update
  \end{itemize}

\item Potential alternative design
  \begin{itemize}
  \item Send KeyUpdate (using the ACK for reliability)
  \item Still have to process out-of-order records
  \end{itemize}
\end{itemize}
\end{slide}

\begin{slide}
\heading{Shrinking the Packet Header}

\begin{itemize}
\item DTLS packet header is very large
\end{itemize}

{\small
\begin{verbatim}
 struct {
     ContentType opaque_type = 23; /* application_data */
     ProtocolVersion legacy_record_version = {254,253); // DTLSv1.2
     uint16 epoch;                         // DTLS-related field
     uint48 sequence_number;               // DTLS-related field
     uint16 length;
     ...
\end{verbatim}
}

\begin{itemize}
\item Would be nice to make it smaller
  \begin{itemize}
  \item Give us room for connection ID...
  \end{itemize}
\end{itemize}

\end{slide}

\begin{slide}
\heading{A shorter header (due to MT)}

\begin{verbatim}
001eesss ssssssss

Where ee = epoch modulo 4 and ss..ss = sequence number
modulo 2048
\end{verbatim}

\begin{itemize}
\item Why two bits for the epoch?
\item What about long header/short header as in QUIC draft?
\end{itemize}

\end{slide}


\begin{slide}
\heading{Connection ID: Problem and Solution}
\begin{itemize} 
\item Demultiplexing based on 5 tuple
\item If NAT binding expires server cannot find security context.
\item Happens if IoT devices enter a sleep cycle.
\item Solution: Add additional identifier to record layer header.
\end{itemize}
\end{slide}

\begin{slide}
\heading{Connection ID: Design Decisions}

\vspace{-3ex}
\begin{itemize} 
\item Should there be a negotation?
\item What messages should use it?  
  \begin{itemize} 
  \item Everything except ClientHello (for backwards compatibility)?
  \item Messages protected using keys derived from a handshake\_traffic\_secret 
  \end{itemize} 

\item Unlinkability property desirable to some: possible approaches

  \begin{itemize} 
  \item Static identifier similar to IPsec SPI
  \item Hash chain/Counter-based approach
  \item Token bucket with receiver-side refresh
  \end{itemize} 

\item Somehow you need to demux these
  \begin{itemize} 
  \item Either outside DTLS
  \item Or change the stack somehow
  \end{itemize} 
\end{itemize} 
\end{slide}

\begin{slide}
\heading{Hash chains}

\begin{itemize}
\item Peers exchange $K_u$
  \begin{itemize}
  \item Packet $i$ uses conn id $E(K_u, i)$ (or $trunc(H^i(K))$
  \end{itemize}

\item Expensive to process with reordering
  \begin{itemize}
  \item Either precompute a bunch of values (memory cost)
  \item Or you have to compute packet $i$ for all $K_u$ when you get an out-of-order packet
  \end{itemize}
\end{itemize}
\end{slide}


\begin{slide}
\heading{Token Buckets}

\begin{itemize}
\item Server has a single static key $K$
\item Server gives client $n$ tokens $T_0, T_1, ... T_n$
  \begin{itemize}
  \item $T_i = E(K, u || i)$
  \end{itemize}
\item Client uses a fresh token for each packet
\item Server replenishes tokens in response to packets
\item Consumes a lot of bandwidth
  \begin{itemize}
  \item Each token is sent twice
  \item Tokens have a minimum size
  \end{itemize}

\end{itemize}

\end{slide}

\begin{slide}
\heading{Handshake Message Transcript}

\begin{itemize}
\item The TLS and DTLS transcripts are different
\item Both include the message header
  \begin{itemize}
  \item But headers are different
  \item DTLS includes a (synthetic) DTLhandshake message header
  \end{itemize}
\item We could just do the TLS message header
  \begin{itemize}
  \item Cross-version consistency between cross-protocol consistency
  \end{itemize}
\end{itemize}


\end{slide}


\begin{slide}
\heading{Other issues?}

\end{slide}

\end{document} 

