\documentclass[helvetica]{seminar} 
\input{xy}
\xyoption{all}
\usepackage{graphicx} 
\usepackage{slidesec} 
\usepackage{url}

\long\def\symbolfootnote[#1]#2{\begingroup%
\def\thefootnote{\fnsymbol{footnote}}\footnote[#1]{#2}\endgroup}

% to fix problems making landscape seminar pdfs
% Letter...
\pdfpagewidth=11truein
\pdfpageheight=8.5truein
\pdfhorigin=1truein     % default value(?), but doesn't work without
\pdfvorigin=1truein     % default value(?), but doesn't work without
% A4
%\pdfpagewidth=297truemm % your milage may vary....
%\pdfpageheight=210truemm
%\pdfhorigin=1truein     % default value(?), but doesn't work without
%\pdfvorigin=1truein     % default value(?), but doesn't work without



\renewcommand{\familydefault}{\sfdefault}  
 
\input{seminar.bug} 
\input{seminar.bg2} % See the Seminar bugs list 
 
\slideframe{none} 
 
 
\usepackage{fancyhdr} 
 
% Headers and footers personalization using the `fancyhdr' package 
\fancyhf{} % Clear all fields 
\renewcommand{\headrulewidth}{0mm} 
\renewcommand{\footrulewidth}{0.1mm} 
 
\fancyfoot[L]{\tiny W3C TPAC 2011} 
\fancyfoot[C]{\tiny RTC-Web Security Issues}
\fancyfoot[R]{\tiny \theslide} 
 
 
% To center horizontally the headers and footers (see seminar.bug) 
\renewcommand{\headwidth}{\textwidth} 

% To adjust the frame length to the header and footer ones 
\autoslidemarginstrue 
\pagestyle{fancy} 
 

\newcommand{\heading}[1]{% 
  \begin{center} 
    \large\bf 
    #1 
  \end{center} 
  \vspace{.4 in}} 

\begin{document}
\begin{slide}
\begin{center}
\vspace{1 in}
\LARGE{{\bf}RTC-Web Security Model Overview}\\
\vspace{.2in}
\large{{W3C TPAC 2011}} \\
\vspace{3em}
\large{
\begin{tabular}{c}
Eric Rescorla \\
\url{ekr@rtfm.com}
\end{tabular}
}
\end{center}

\end{slide}


\centerslidesfalse 



\begin{slide}
\heading{Context}

\begin{itemize}
\item IETF RTC-Web has been developing a security model
\item This is an attempt to summarize the current state
\end{itemize}

\end{slide}


\begin{slide}
\heading{The Browser Threat Model}

\textbf{Core Web Security Guarantee}: ``users can safely visit arbitrary web sites
and execute scripts provided by those sites.''\cite{huang-cache}

\vspace{1em}

\begin{itemize}
\item This includes sites which are hosting malicious scripts!
\item Basic Web security technique is isolation/sandboxing
  \begin{itemize}
  \item Protect your computer from malicious scripts
  \item Protect content from site A from content hosted at site B
  \item Protect site A from content hosted at site B
  \end{itemize}
\item In this case we're primarily concerned with JavaScript running in the browser
\end{itemize}

\textbf{The browser acts as a trusted computing base for the site}

\end{slide}



\begin{slide}
\heading{The Internet Threat Model}

``You hand the packets to the attacker to deliver.'' -- Steve Bellovin

\begin{itemize}
\item Endpoints are secure
\item An attack has complete control of the network...
\begin{itemize}
\item Can read any packet you send
\item Can modify any packet in transit
\item Can forge a packet with any source address
\end{itemize}
\item Cryptographic techniques are the major security measure
\end{itemize}
\end{slide}



\begin{slide}
\heading{So what's our threat model?}

\begin{itemize}
\item Solid protection under the browser threat model
  \begin{itemize}
  \item Safe to visit any Web site
  \item Can safely authorize A and not B
  \end{itemize}
\item Do the best we can under the Internet threat model
  \begin{itemize}
  \item Protect against network attackers if people use HTTPS
  \item Provide strong media security for RTCWeb-RTCWeb calls
  \end{itemize}
\end{itemize}
\end{slide}


\begin{slide}
\heading{List of Issues to Consider}

\begin{itemize}
\item Access to local devices
\item Consent to communications
\item Communications security
\end{itemize}
              
\end{slide}


\begin{slide}
\heading{Access to Local Devices}

\vspace{-.2in}

\begin{itemize}
\item Making phone (and video) calls requires that your voice be transmitted to other side
  \begin{itemize}
  \item But the \emph{other side} is controlled by some site you visit
  \item What if you visit \url{http://bugmyphone.example.com}?
  \end{itemize}

\item Somehow we need to get the user's consent
  \begin{itemize}
  \item But to what?
  \item And when?
  \item Users routinely click through warning dialogs when presenting ``in-flow''
  \end{itemize}
\end{itemize}
\end{slide}


\begin{slide}
\heading{Permissions Models}

\begin{itemize}
\item Short-term permissions
\begin{itemize}
\item Allow this single call
\item E.g., for a customer service call
\end{itemize}

\item Long-term permissions
\begin{itemize}
\item Allow this site to make calls whenever it wants
\item E.g., for a social networking or calling site
\item Is this going to happen? [Terriberry]
\end{itemize}

\item Per-peer permissions
\begin{itemize}
\item Allow calls to \url{bob@example.com}
\item Requires strong peer identity
\end{itemize}
\end{itemize}
\end{slide}




\begin{slide}
\heading{Ad Hoc Calling from Embedded Advertisements}

\vspace{-.3in}
\begin{minipage}[b]{1.8in}
\includegraphics[width=1.8in]{ford-example}
\center{Option A: Ad in an IFRAME}
\end{minipage}
\begin{minipage}[b]{1.8in}
\includegraphics[width=1.8in]{ford-example2}
\center{Option B: Injected ad}
\end{minipage}
\end{slide}


\begin{slide}
\heading{User expectations}

\begin{itemize}
\item When I place this call I'm talking to Ford
\item Not giving Ford long-term access to my camera and microphone
\item[]
\item But I'm on Slashdot...
\begin{itemize}
\item Do I think Slashdot has endorsed this?
\end{itemize}
\end{itemize}
\end{slide}


\begin{slide}
\heading{UI for Short-Term Permissions}

\begin{itemize}
\item Generally necessary at the time of call
\item Needs an immediate call-to-action
\begin{itemize}
\item E.g., a ``check-my-hair'' self video image with an OK button
\item This must be in the browser chrome to prevent click-jacking
\end{itemize}
\item Permissions should only be good for this call
\item Non-maskable indicator of call status in browser chrome
\item Also a non-maskable call termination button
\end{itemize}


\end{slide}


\begin{slide}
\heading{API impact of short-term permissions}

\begin{itemize}
\item One natural design is to show ``self'' picture a la Facetime
  \begin{itemize}
  \item Here's your video, do you want to set up the call
  \end{itemize}

\item But this implies \emph{some level} of device access prior to permissions grant
  \begin{itemize}
  \item Step 1: display video to user with call start button
  \item Step 2: start call
  \end{itemize}

\item Out of scope?
\end{itemize}
\end{slide}



\begin{slide}
\heading{What about the site I'm visiting?}

\begin{itemize}
\item Adam Barth: the user thinks he's on Slashdot
  \begin{itemize}
  \item Even though Slashdot neither placed the ad nor is the called party
  \item Only vaguely conscious of ad networks
  \end{itemize}

\item Should the top-level site get to have an opinion?
  \begin{itemize}
  \item Protect the user?
  \item Protect its reputation?
  \item What about privacy?
  \end{itemize}
\end{itemize}

\end{slide}


\begin{slide}
\heading{Long-term permissions: Calling Services}

\begin{itemize}
\item I have an account on SocialWeb
\begin{itemize}
\item ... ``friends'' with a bunch of my real-world friends
\item Want to call one of my friends
\end{itemize}
\end{itemize}

\includegraphics[width=2in]{cullen1}
\end{slide}


\begin{slide}
\heading{UI for Long-Term Permissions}

\begin{itemize}
\item Probably a bad idea to have an intrusive dialog
\begin{itemize}
\item Want to avoid ``click here to screw yourself''
\item Possible: door hanger style UI ``some features on this site may not work''
\end{itemize}

\item Do we want this at all?
  \begin{itemize}
  \item Tension between convenience and security
    \begin{itemize}
    \item Calling sites likely want a clean UI
    \item But this means that the calling site has a lot of power
    \end{itemize}

  \item Is this a permissions model the browsers want?
  \end{itemize}
\end{itemize}
\end{slide}



\begin{slide}
\heading{Peer Identity-based Permissions}

\begin{itemize}
\item Not clear what level of attention to demand
\item General idea
  \begin{itemize}
  \item Do you want to talk to \url{bob@example.com}
  \item Easier if you can integrate with address book
  \end{itemize}

\item Orthogonal to site-based permissions?
\end{itemize}

\end{slide}

\begin{slide}
\heading{Partial Digression: Network Attackers}

\vspace{-.25in}
\begin{itemize}
\item Assumption: I've authorized PokerWeb
\item I'm in an Internet Cafe and visit any URL
  \begin{itemize}
  \item Attacker injects IFRAME pretending to be PokerWeb
  \item But calls go to him
  \end{itemize}
\end{itemize}

\includegraphics[width=2in]{injected-iframe}

\begin{itemize}
\item Result: attacker has bugged your computer
\end{itemize}
\end{slide}


\begin{slide}
\heading{User Expectations}

\begin{itemize}
\item It's safe to authorize PokerWeb and then surf the Internet
  \begin{itemize}
  \item Without being bugged
  \end{itemize}

\item Including on insecure networks
  \begin{itemize}
  \item This may include your home network
  \end{itemize}

\item Unfortunately, this is not true here...
\end{itemize}
\end{slide}



\begin{slide}
\heading{Origin and HTTP/HTTPS}

\begin{itemize}
\item Basic problem here is HTTP-based origins
  \begin{itemize}
  \item Question: should RTC-Web be available over HTTP
  \end{itemize}

\item What about mixed content?
  \begin{itemize}
  \item Appears to be HTTPS but for our purposes is HTTP
  \item Treat as HTTP?
  \item Forbid RTC-Web entirely over mixed content?
  \end{itemize}
\end{itemize}
\end{slide}



\begin{slide}
\heading{Consent for real-time peer-to-peer communication}

\begin{itemize}
\item Need to able to send data between two browsers
  \begin{itemize}
  \item Unless you want to relay everything
  \end{itemize}

\item But this is unsafe (and violates SOP)
  \begin{itemize}
  \item Not OK to let browsers send TCP and UDP to arbitrary locations
  \end{itemize}

\item General principle: verify consent
  \begin{itemize}
  \item Before sending traffic from a script to recipient, verify recipient
    wants to receive it from the sender
  \item Familiar paradigm from CORS~\cite{cors} and WebSockets\cite{hybi}
  \end{itemize}
\end{itemize}

\end{slide}

\begin{slide}
\heading{How to verify communications consent for RTC-Web}

\vspace{-.2in}
\begin{itemize}
\item Can't trust the server (see above)
  \begin{itemize}
  \item Needs to be enforced by the browser
  \end{itemize}

\item Browser does a handshake with target peer to verify connectivity
\end{itemize}

\vspace{-.25in}
$$
\xymatrix@R=.2in@C=1.3in{
  \txt{\textbf{Alice}} & \txt{\textbf{Server}} & \txt{\textbf{Bob}} \\
  & \ar[l]_{\txt{Connect to Bob}}  \ar[r]^{\txt{Connect to Alice}} &\\
  \ar@{<->}[rr]^{\txt{Handshake}} & & \\
  \ar@{<->}[rr]^{\txt{Media traffic}} & &\\
}`
$$

\begin{itemize}
\item This should look familiar from ICE~\cite{rfc5245}
\end{itemize}
\end{slide}



\begin{slide}
\heading{Implementing Communications Consent Securely}

\vspace{-.25in}
\begin{itemize}
\item Remember: we don't trust the JS
\item[]
\item Restrict pre-handshake communications
\begin{itemize}
\item Restrict communications to an endpoint until handshake completes
\item Minimize application control of ICE packets (extensions, etc.)
\item Rate-limit ICE checks
\end{itemize}
\item Browser \textbf{must not} let application see STUN transaction ID
\begin{itemize}
\item Prevents forgery of STUN responses by the server
\end{itemize}
\item What about cross-protocol attacks?
\begin{itemize}
\item Not really an issue for UDP, especially with DTLS
\item TCP \textbf{must} use masking
\end{itemize}

\end{itemize}

\end{slide}


\begin{slide}
\heading{What about communications security?}

\begin{itemize}
\item We've already addressed this in the context of SIP
  \begin{itemize}
  \item Things aren't that different here--all the usual protocols work
  \end{itemize}

\item This should be mostly invisible with current style API
  \begin{itemize}
  \item The relevant messaging is embedded in the SDP
  \item Possibly need API controls to set up security parameters
  \item ... cipher suites, etc.
  \end{itemize}

\item Security indicators need to be in the browser chrome
  \begin{itemize}
  \item So the it can't be changed by the JS
  \end{itemize}

\end{itemize}

\end{slide}



\begin{slide}
\heading{Questions?}

\end{slide}



\begin{slide}

\scriptsize{
\bibliographystyle{alpha}
\bibliography{ws}
}
\end{slide}






\end{document}