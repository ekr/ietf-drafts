\documentclass[helvetica]{seminar} 
\input{xy}
\xyoption{all}
\usepackage{graphicx} 
\usepackage{slidesec} 
\usepackage{url}
\usepackage[framemethod=TikZ]{mdframed}
\usepackage{color}
\usepackage[normalem]{ulem}

\long\def\symbolfootnote[#1]#2{\begingroup%
\def\thefootnote{\fnsymbol{footnote}}\footnote[#1]{#2}\endgroup}

% to fix problems making landscape seminar pdfs
% Letter...
\pdfpagewidth=11truein
\pdfpageheight=8.5truein
\pdfhorigin=1truein     % default value(?), but doesn't work without
\pdfvorigin=1truein     % default value(?), but doesn't work without
% A4
%\pdfpagewidth=297truemm % your milage may vary....
%\pdfpageheight=210truemm
%\pdfhorigin=1truein     % default value(?), but doesn't work without
%\pdfvorigin=1truein     % default value(?), but doesn't work without


\renewcommand{\familydefault}{\sfdefault}  
 
\input{seminar.bug} 
\input{seminar.bg2} % See the Seminar bugs list 
 
\slideframe{none} 
 
 
\usepackage{fancyhdr} 
 
% Headers and footers personalization using the `fancyhdr' package 
\fancyhf{} % Clear all fields 
\renewcommand{\headrulewidth}{0mm} 
\renewcommand{\footrulewidth}{0.1mm} 
 
\fancyfoot[L]{\tiny IETF 92} 
\fancyfoot[C]{\tiny TLS}
\fancyfoot[R]{\tiny \theslide} 
 
 
% To center horizontally the headers and footers (see seminar.bug) 
\renewcommand{\headwidth}{\textwidth} 

% To adjust the frame length to the header and footer ones 
\autoslidemarginstrue 
\pagestyle{fancy} 
 

\newcommand{\heading}[1]{% 
  \begin{center} 
    \large\bf 
    #1 
  \end{center} 
  \vspace{.4 in}} 



\begin{document}

\begin{slide}
\begin{center}
\vspace{.5 in}
\LARGE{{\bf}TCP Use TLS Option}\\
\vspace{.2in}
\large{
\begin{tabular}{c}
Eric Rescorla\\
Mozilla\\
\url{ekr@rtfm.com}
\end{tabular}
}
\end{center}

\end{slide}

\centerslidesfalse 

\begin{slide}
\heading{Background: TLS over TCP}

\begin{itemize}
\item TLS over TCP is ubiquitous
\begin{itemize}
\item Probably the most deployed Internet security protocol – Widely implemented
\item Heavily analyzed and reasonably well understood
\end{itemize}
\item Hard to coordinate
\begin{itemize}
\item  Servers which are expecting application data choke on TLS \verb^ClientHello^
\end{itemize}
\end{itemize}

\end{slide}

\begin{slide}
\heading{Some Existing Coordination techniques}

\begin{itemize}
\item External signal to the client (e.g., \verb^https:^)
\item Separate ports
\item Manual config
\item DNS signaling
\item Extend the application layer protocol (STARTTLS)
\item None of these lend themselves to opportunistic deployment
\end{itemize}

\end{slide}


\begin{slide}
\heading{Problem Statement}

Add the minimum necessary machinery to TCP to let it opportunistically negotiate TLS when both sides want to.
\end{slide}


\begin{slide}
\heading{TLS TCP Option}

\xymatrix@R=.2in@C=2.5in{
Alice & Bob \\
\ar[r]^{\txt{\texttt{SYN + TCP-TLS}}} & \\
& \ar[l]_{\txt{\texttt{SYN/ACK + TCP-TLS}}} \\
\ar[r]^{\txt{\texttt{ACK}}} & \\
\ar@{<->}[r]^{\txt{\texttt{TLS Handshake}}} & \\
\ar@{<->}[r]^{\txt{\texttt{Application Data (over TLS)}}} & \\
}

\end{slide}


\begin{slide}
\heading{Bob Doesn't Support TLS}

\xymatrix@R=.2in@C=2.5in{
Alice & Bob \\
\ar[r]^{\txt{\texttt{SYN + TCP-TLS}}} & \\
& \ar[l]_{\txt{\texttt{SYN/ACK}}} \\
\ar[r]^{\txt{\texttt{ACK}}} & \\
\ar@{<->}[r]^{\txt{\texttt{Application Data}}} & \\
}

\end{slide}


\begin{slide}
\heading{What do we need to signal?}

\begin{itemize}
\item That I want to do TLS
  \begin{itemize}
    \item Signaled by option present
  \end{itemize}

\item TLS roles (client vs. server)
\item Obvious for non simultaneous open case
  \begin{itemize}
  \item Let's ignore simultaneous open (or do an optional tiebreaker)
  \end{itemize}

\end{itemize}
\end{slide}

\begin{slide}
\heading{Minimal Option}

\begin{verbatim}
           +------------+------------+
           |  Kind=XX   | Length = 2 |
           +------------+------------+
\end{verbatim}
\end{slide}


\begin{slide}
\heading{Comparison to Integrated Designs (e.g. tcpcrypt)}

\begin{itemize}
\item Advantages
  \begin{itemize}
  \item Easy to specify and implement
  \item Leverage the work that has alredy gone into TLS 
    \begin{itemize}
    \item Looks like existing TLS over TCP on the wire
    \end{itemize}
\end{itemize}

\item Disadvantages
\begin{itemize}
\item Imports TLS history; may want to profile
\item Less optimized, especially when you want to do anti-DoS
\item TLS records can span segment boundaries 
  \begin{itemize}
  \item Easy to manage with attention to MTU
  \end{itemize}
\end{itemize}
\end{itemize}
\end{slide}

\begin{slide}
\heading{End of Connection (not in draft)}

\begin{itemize}
\item TLS already has a connection close (\verb^close_notify^)
\item Half-closed state not supported
  \begin{itemize}
  \item Could modify TLS if needed
  \end{itemize}
\end{itemize}
\end{slide}


\begin{slide}
\heading{Setup latency (detail, TLS 1.3, no data in SYN)}


\xymatrix@R=.2in@C=2.5in{
Alice & Bob \\
\ar[r]^{\txt{\texttt{SYN + TCP-TLS}}} & \\
& \ar[l]_{\txt{\texttt{SYN/ACK + TCP-TLS}}} \\
\ar[r]^{\txt{\texttt{ACK}}} & \\
\ar[r]^{\txt{\texttt{ClientHello, ClientKeyShare}}} & \\
& \ar[l]_{\txt{\texttt{ServerHello, ServerKeyShare, ..., Finished}}} \\
& \ar[l]_{\color{red}\txt{\texttt{Application Data}}} \\
\ar[r]^{\txt{\texttt{Finished}}} & \\
\ar@{<->}[r]^{\color{red}\txt{\texttt{Application Data}}} & \\
}
\end{slide}


\begin{slide}
\heading{Setup latency (detail, TLS 1.3, TFO or data in SYN)}

\xymatrix@R=.2in@C=2.5in{
Alice & Bob \\
\ar[r]^{\txt{\texttt{SYN + TCP-TLS}}} & \\
\ar[r]^{\txt{\texttt{ClientHello, ClientKeyShare}}} & \\
& \ar[l]_{\txt{\texttt{SYN/ACK + TCP-TLS}}} \\
& \ar[l]_{\txt{\texttt{ServerHello, ServerKeyShare, ..., Finished}}} \\
& \ar[l]_{\color{red}\txt{\texttt{Application Data}}} \\
\ar[r]^{\txt{\texttt{ACK}}} & \\
\ar[r]^{\txt{\texttt{Finished}}} & \\
\ar@{<->}[r]^{\color{red}\txt{\texttt{Application Data}}} & \\
}
\end{slide}


\begin{slide}
\heading{TLS Complexity/Profiling}

\begin{itemize}
\item TLS is \sout{complicated}powerful
  \begin{itemize}
    \item Though TLS 1.3 is removing a lot of stuff
  \end{itemize}
\item The necessary subset for this is not that complicated
\item And it's a pretty obvious subsetting exercise
\end{itemize}
\end{slide}


\begin{slide}
\heading{Questions?}

\end{slide}

\begin{slide}
\heading{Backup Slide: Handling Simultaneous Open}

\begin{verbatim}
           +------------+------------+------------+------------+
           |  Kind=XX   | Length = 8 |        Tiebreaker       |
           +------------+------------+------------+------------+
           |                   Tiebreaker                      |
           +---------------------------------------------------+
\end{verbatim}
\end{slide}

\end{document}
