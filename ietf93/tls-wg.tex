\documentclass[helvetica]{seminar} 
\input{xy}
\xyoption{all}
\usepackage{graphicx} 
\usepackage{slidesec} 
\usepackage{url}
\usepackage[framemethod=TikZ]{mdframed}
\usepackage{color}

\long\def\symbolfootnote[#1]#2{\begingroup%
\def\thefootnote{\fnsymbol{footnote}}\footnote[#1]{#2}\endgroup}

% to fix problems making landscape seminar pdfs
% Letter...
\pdfpagewidth=11truein
\pdfpageheight=8.5truein
\pdfhorigin=1truein     % default value(?), but doesn't work without
\pdfvorigin=1truein     % default value(?), but doesn't work without
% A4
%\pdfpagewidth=297truemm % your milage may vary....
%\pdfpageheight=210truemm
%\pdfhorigin=1truein     % default value(?), but doesn't work without
%\pdfvorigin=1truein     % default value(?), but doesn't work without


\renewcommand{\familydefault}{\sfdefault}  
 
\input{seminar.bug} 
\input{seminar.bg2} % See the Seminar bugs list 
 
\slideframe{none} 
 
 
\usepackage{fancyhdr} 
 
% Headers and footers personalization using the `fancyhdr' package 
\fancyhf{} % Clear all fields 
\renewcommand{\headrulewidth}{0mm} 
\renewcommand{\footrulewidth}{0.1mm} 
 
\fancyfoot[L]{\tiny IETF 93} 
\fancyfoot[C]{\tiny TLS}
\fancyfoot[R]{\tiny \theslide} 
 
 
% To center horizontally the headers and footers (see seminar.bug) 
\renewcommand{\headwidth}{\textwidth} 

% To adjust the frame length to the header and footer ones 
\autoslidemarginstrue 
\pagestyle{fancy} 
 

\newcommand{\heading}[1]{% 
  \begin{center} 
    \large\bf 
    #1 
  \end{center} 
  \vspace{.4 in}} 



\begin{document}

\begin{slide}
\begin{center}
\vspace{.5 in}
\LARGE{{\bf}TLS 1.3 Status}\\
\vspace{.2in}
\large{
\begin{tabular}{c}
Eric Rescorla\\
Mozilla\\
\url{ekr@rtfm.com}
\end{tabular}
}
\end{center}

\end{slide}

\centerslidesfalse 

\begin{slide}
\heading{Overview of Changes Since IETF 92 (Major)}

\begin{itemize}
\item Integrate DH-based handshake (per WG discussion in Dallas)
\item Add initial cut at 0-RTT support
\item HKDF-based key derivation (per WG discussion in Dallas)
\item Moved ClientKeyShare into an extension
\item Added support for PSK
\item Removed resumption and merged ticket support with PSK
\end{itemize}
\end{slide}

\begin{slide}
\heading{Overview of Changes Since IETF 92 (Minor)}

\begin{itemize}
\item Prohibit RC4 negotiation
\item Froze record-layer header
\item Context field for signatures
\item Replaced explicit IV with sequence number + mask
\end{itemize}
\end{slide}



\begin{slide}
\heading{Open Issues Preview}

\begin{itemize}
\item Indicating known configurations
\item 0-RTT w/ PSK
\item Interaction of 0-RTT and authentication
\item 0-RTT rejection handling
\item PSK resumption restrictions
\item Traffic key generation
\end{itemize}

\end{slide}


\begin{slide}
\heading{DH-Based Handshake (Review)}

\begin{itemize}
\item Server has a semi-static DH key (just like 1-RTT)
\item Probably really has long-term signing key
  \begin{itemize}
  \item Used to sign the semi-static key
  \item Agreement at previous IETFs to use online-only signing
  \end{itemize}

\item Common key exchange computations between all modes
\end{itemize}
\end{slide}


\begin{slide}
\heading{Key Computation Inputs}

\begin{footnotesize}
\begin{verbatim}
     Key Exchange            Static Secret (SS)    Ephemeral Secret (ES)
     ------------            ------------------    ---------------------
     (EC)DHE                   Client ephemeral         Client ephemeral
     (full handshake)       w/ server ephemeral      w/ server ephemeral

     (EC)DHE                   Client ephemeral         Client ephemeral
     (w/ known_configuration)      w/ Known Key      w/ server ephemeral

     PSK                         Pre-Shared Key           Pre-shared key

     PSK + (EC)DHE               Pre-Shared Key         Client ephemeral
\end{verbatim}
\end{footnotesize}
\end{slide}


\begin{slide}
\heading{Key Computations}

\begin{center}
\includegraphics[width=3in]{tls-key-derive}
\end{center}

\end{slide}


\begin{slide}
\heading{Two New Mechanisms}

\begin{itemize}
\item Server configurations and known configuration
  \begin{itemize}
  \item Server publishes a configuration to the client in handshake $n$
  \item Client reuses that configuration in handshake $n+1$
  \end{itemize}

\item Early data indication
  \begin{itemize}
  \item Client indicates that he wants to do 0-RTT (client auth, data, both)
  \item Server accepts or rejects
  \end{itemize}
\end{itemize}

\end{slide}
\begin{slide}
\heading{Example: Initial Handshake}

\vspace{-2ex}
\begin{footnotesize}
\begin{verbatim}
     ClientHello
       + ClientKeyShare        -------->
                                                     ServerHello
                                                 ServerKeyShare*
                                           {EncryptedExtensions}
                                          {ServerConfiguration*}
                                                  {Certificate*}
                                           {CertificateRequest*}
                                            {CertificateVerify*}
                               <--------              {Finished}
     {Certificate*}
     {CertificateVerify*}
     {Finished}                -------->
     [Application Data]        <------->      [Application Data]
\end{verbatim}
\end{footnotesize}
\end{slide}


\begin{slide}
\heading{Known Configuration}

\begin{footnotesize}
\begin{verbatim}
         struct {
             opaque configuration_id<1..2^16-1>;
             uint32 expiration_date;
             NamedGroup group;
             opaque server_key<1..2^16-1>;
             Boolean early_data_allowed;
         } ServerConfiguration;
\end{verbatim}
\end{footnotesize}

\begin{itemize}
\item The client's reuse of the configuration implicitly resurrects the previous state
  \begin{itemize}
  \item See open issues
  \end{itemize}
\end{itemize}

\end{slide}



\begin{slide}
\heading{Example: 0-RTT Handshake (w/o new configuration)}

\vspace{-4ex}
\begin{footnotesize}
\begin{verbatim}
       ClientHello
         + ClientKeyShare
         + KnownConfiguration
         + EarlyDataIndication
       (Certificate*)
       (CertificateVerify*)
       (Application Data)                       -------->
                                                       ServerHello
                                              + KnownConfiguration
                                             + EarlyDataIndication
                                                    ServerKeyShare
                                 <--------              {Finished}
       {Finished}                -------->

       [Application Data]        <------->      [Application Data]
\end{verbatim}
\end{footnotesize}
\end{slide}


\begin{slide}
\heading{Early Data Indication}


\begin{footnotesize}
\begin{verbatim}
         enum { early_handshake(1), early_data(2),
                early_handshake_and_data(3), (255) } EarlyDataType;

         struct {
           select (Role) {
             case client:
               opaque context<0..255>;
               EarlyDataType type;
             case server:
               struct {};
           }
         } EarlyDataIndication;
\end{verbatim}
\end{footnotesize}
\end{slide}


\begin{slide}
\heading{What do failed 0-RTT handshakes look like?}

\begin{itemize}
\item Server doesn't respond with an EarlyDataIndication
  \begin{itemize}
  \item System falls back to 1-RTT
  \item All of the early data is just ignored
  \end{itemize}

\item This is kind of clunky
  \begin{itemize}
  \item Early handshake messages have a different content type
  \item What about encrypted content types
  \end{itemize}

\item Analysis needed that ignoring early data is OK
  \begin{itemize}
  \item ... currently underway
  \end{itemize}
\end{itemize}

\end{slide}



\begin{slide}
\heading{Managing semi-static keys (I)}

\begin{itemize}
\item Need two keys
  \begin{itemize}
  \item Ephemeral (for PFS)
  \item Semi-static (cached server 1-RTT, 0-RTT)
  \end{itemize}

\item Various options for making these work together
  \begin{itemize}
  \item Always use a single semi-static key -- suboptimal performance
  \item Have the server supply a separate key -- odd when you refresh keys
  \end{itemize}
\end{itemize}
\end{slide}

\begin{slide}
\heading{Managing semi-static keys (II)}
\begin{itemize}
\item Current draft state
  \begin{itemize}
  \item First handshake looks like draft-06
    \begin{itemize}
    \item Can supply a \verb^ServerConfiguration^
    \end{itemize}

  \item Subsequent handshakes can reuse \verb^ServerConfiguration^
    \begin{itemize}
      \item But need to sign if they want to provide one
    \end{itemize}
  \end{itemize}
\item More modes than we would really like (but best perf profile)
\end{itemize}
\end{slide}


\begin{slide}
\heading{Example: 0-RTT Handshake w/ new configuration}

\vspace{-4ex}
\begin{footnotesize}
\begin{verbatim}
       ClientHello
         + ClientKeyShare
         + KnownConfiguration
         + EarlyDataIndication
       (Certificate*)
       (CertificateVerify*)
       (Application Data)        -------->
                                                       ServerHello
                                              + KnownConfiguration
                                             + EarlyDataIndication
                                                    ServerKeyShare
                                            {ServerConfiguration*} <- 
                                                    {Certificate*} <- SEE HERE
                                              {CertificateVerify*} <-
                                 <--------              {Finished}
       {Finished}                -------->

       [Application Data]        <------->      [Application Data]
\end{verbatim}
\end{footnotesize}
\end{slide}


\begin{slide}
\heading{Pre-Shared Keys}

\begin{itemize}
\item TLS 1.2 had PSK
  \begin{itemize}
  \item But we kind of broke it
  \end{itemize}

\item draft-07 brings it back
  \begin{itemize}
  \item But I did get rid of identity hint...
  \end{itemize}
\end{itemize}
\end{slide}


\begin{slide}
\heading{Example: Pure PSK Handshake}

\begin{footnotesize}
\begin{verbatim}
          ClientHello
            + ClientKeyShare,
              PreSharedKeyExtension -------->
                                                          ServerHello
                                               +PreSharedKeyExtension
                                    <--------              {Finished}
          {Certificate*}
          {Finished}                -------->
          [Application Data]        <------->      [Application Data]
\end{verbatim}
\end{footnotesize}

\begin{itemize}
\item Can also do this with DHE-PSK
\end{itemize}
\end{slide}


\begin{slide}
\heading{PreSharedKey Extension}

\begin{footnotesize}
\begin{verbatim}
         opaque psk_identity<0..2^16-1>;

         struct {
           select (Role) {
             case client:
               psk_identity identities<0..2^16-1>;

             case server:
               psk_identity identity;
         } PreSharedKeyExtension;
\end{verbatim}
\end{footnotesize}
\end{slide}


\begin{slide}
\heading{PSK For Resumption}

\begin{itemize}
\item Resumption and PSK are very similar
  \begin{itemize}
  \item Let's make them identical
  \end{itemize}

\item Basic idea
  \begin{itemize}
  \item Server gives client a PSK label
  \item PSK is derived from initial handshake (resumption master secret)
  \end{itemize}
\end{itemize}
\end{slide}


\begin{slide}
\heading{Example: Establishing a PSK for resumption}

\begin{footnotesize}
\begin{verbatim}
          ClientHello
            + ClientKeyShare       -------->
                                                          ServerHello
                                                       ServerKeyShare
                                                {EncryptedExtensions}
                                                {ServerConfiguration*}
                                                       {Certificate*}
                                                {CertificateRequest*}
                                                 {CertificateVerify*}
                                    <--------              {Finished}
          {Certificate*}
          {CertificateVerify*}
          {Finished}                -------->
                                    <--------      [NewSessionTicket] <- SEE HERE
          [Application Data]        <------->      [Application Data]
\end{verbatim}
\end{footnotesize}
\end{slide}


\begin{slide}
\heading{AEAD IV}

\begin{itemize}
\item TLS 1.2 (well, GCM) uses a partially explicit IV
  \begin{itemize}
  \item This chews up bandwidth
  \end{itemize}
  
\item Consensus to remove explicit IV
  \begin{itemize}
  \item And reuse sequence number
  \item Brian Smith raised concerns about every connection using the same nonce sequence
  \end{itemize}
\end{itemize}
\end{slide}


\begin{slide}
\heading{draft-07 design for AEAD IV}

\begin{itemize}
\item $iv\_length = max(8, N\_MAX)$
\item Generate per-session mask of length $iv\_length$
\item Left-pad RSN with 0s to $iv\_length$
\item XOR RSN with mask to produce per-record nonce
\end{itemize}
\end{slide}


\begin{slide}
\heading{ClientKeyShare Extension}

\begin{itemize}
\item This used to be a separate message
  \begin{itemize}
  \item That just made life complicated
  \end{itemize}

\item It's now an extension
\item Nothing else has changed
\end{itemize}
\end{slide}


\begin{slide}
\heading{Indicating Known Configurations}

\begin{itemize}
\item Current design has client just indicate configuration ID
  \begin{itemize}
  \item This means that the server needs to memorize each crypto configuration (ugh)
  \end{itemize}

\item Proposed redesign
  \begin{itemize}
  \item Client indicates configuration ID \emph{and} cryptographic configuration
    \begin{itemize}
    \item Cipher suites and cryptographic extensions
    \item MUST replicate the server's selection from a previous handshake
    \end{itemize}
  \item Server verifies client's \verb^ClientHello^
    \begin{itemize}
    \item Checks that configuration ID is valid
    \item Verifies that client's parameters are what it would negotiate
    \end{itemize}
  \end{itemize}
\end{itemize}
\end{slide}


\begin{slide}
\heading{Strawman}

\begin{footnotesize}
\begin{verbatim}
         struct {
           select (Role) {
             case client:
               opaque identifier<0..2^16-1>;
               CipherSuite cipher_suite;            <- SEE HERE
               Extension extensions<0..2^16-1>;     <- SEE HERE       

             case server:
               struct {};
           }
         } KnownConfigurationExtension
\end{verbatim}
\end{footnotesize}
\end{slide}

\begin{slide}
\heading{Analysis}

\begin{itemize}
\item Pros
  \begin{itemize}
  \item Server doesn't need to keep per-connection state
  \item Neatly solves PSK (and any other key negotiation mechanism)
  \item Explicit state is explicit
  \end{itemize}
\item Cons
  \begin{itemize}
  \item Server has to compare client's offer
  \item Very modest wire bloat
  \end{itemize}

\item Note: we could have the server not echo the parameters in ServerHello
  \begin{itemize}
  \item But I'd rather keep things consistent
  \end{itemize}
\end{itemize}
\end{slide}


\begin{slide}
\heading{0-RTT Rejection Handling (I)}

\begin{itemize}
\item Currently it's all or nothing
  \begin{itemize}
  \item Server can't accept 0-RTT client auth but not 0-RTT data
  \item ... maybe it should be able to express its preferences in ServerConfiguration
  \end{itemize}

\item This seems easiest
\item Proposed resolution: Server gets to indicate what it wants in ServerConfiguration
\end{itemize}
\end{slide}


\begin{slide}
\heading{0-RTT Rejection Handling (II)}

\vspace{-2ex}
\begin{itemize}
\item How do you distinguish client's early data (which you want to discard) from the client's second flight (which you want to process)
\item Current algorithm uses content type
  \begin{itemize}
  \item Early handshake data has \verb^early_handshake^
  \item Early data has \verb^application_data^ type
  \item The next thing you want to process has \verb^handshake^ type
  \item Just skip to the next \verb^handshake^ message
  \end{itemize}
\item This isn't maximally elegant
  \begin{itemize}
  \item And will fail with encrypted content types (there you need trial decryption)
  \item Other ideas welcome
  \end{itemize}
\end{itemize}
\end{slide}


\begin{slide}
\heading{0-RTT Rejection Handling (III)}

\begin{itemize}
\item What is included in handshake hash?
  \begin{itemize}
  \item Handshake hash generally includes \emph{plaintext}
  \item ... but in rejection cases, you probably don't have decryption cases
  \end{itemize}

\item Present draft just ignores this data with rejection
\item Alternative: include \emph{ciphertext}
\item Proposal: keep with current version pending analysis
\end{itemize}
\end{slide}


\begin{slide}
\heading{0-RTT and Authentication}

\begin{itemize}
\item There isn't any per-connection data from the server to sign
  \begin{itemize}
  \item Client provides all the freshness\symbolfootnote[1]{
      Insert caveats about issues with 0-RTT anti-replay}
  \end{itemize}
\item What context does the client have to sign?
  \begin{itemize}
  \item It should include server identity
  \end{itemize}
\end{itemize}

\begin{footnotesize}
\begin{verbatim}
      handshake_hash = Hash(
                            Hash(handshake_messages) ||
                            Hash(configuration)
                           )

      configuration = ServerConfiguration || Certificate                           
\end{verbatim}
\end{footnotesize}
\end{slide}


\begin{slide}
\heading{PSK Resumption Restrictions?}

\begin{itemize}
\item Resumption required that you use the same ciphers
  \begin{itemize}
  \item But if you make resumption PSK then you could in principle negotiate a new cipher
  \end{itemize}

\item Should we require servers to pick the same symmetric cipher?
\item This would be somewhat easier if we had a la carte negotiation
\end{itemize}
\end{slide}


\begin{slide}
\heading{Traffic Key Generation}

\begin{itemize}
\item Presently we generate a \verb^key_block^
\item ... and then slice and dice
\item Generating independent keys with a context input would be more HSM-friendly
\item Expected context
  \begin{itemize}
  \item Key length
  \item Usage
  \item Algorithm (ugh)
  \end{itemize}
\item Should we do this?
\end{itemize}
\end{slide}


\begin{itemize}
\item This slide intentionally left blank
\end{itemize}

\end{document}  
