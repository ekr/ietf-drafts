\documentclass[helvetica]{seminar} 
\centerslidesfalse 
\input{xy}
\xyoption{all}
\usepackage{graphicx} 
\usepackage{slidesec} 
\usepackage{url} 

\long\def\symbolfootnote[#1]#2{\begingroup%
\def\thefootnote{\fnsymbol{footnote}}\footnote[#1]{#2}\endgroup}

% to fix problems making landscape seminar pdfs
% Letter...
\pdfpagewidth=11truein
\pdfpageheight=8.5truein
\pdfhorigin=1truein     % default value(?), but doesn't work without
\pdfvorigin=1truein     % default value(?), but doesn't work without
% A4
%\pdfpagewidth=297truemm % your milage may vary....
%\pdfpageheight=210truemm
%\pdfhorigin=1truein     % default value(?), but doesn't work without
%\pdfvorigin=1truein     % default value(?), but doesn't work without



\renewcommand{\familydefault}{\sfdefault}  
 
\input{seminar.bug} 
\input{seminar.bg2} % See the Seminar bugs list 
 
\slideframe{none} 
 
 
\usepackage{fancyhdr} 
 
% Headers and footers personalization using the `fancyhdr' package 
\fancyhf{} % Clear all fields 
\renewcommand{\headrulewidth}{0mm} 
\renewcommand{\footrulewidth}{0.1mm} 
 
\fancyfoot[L]{\tiny IETF 70} 
\fancyfoot[C]{\tiny Secure Peer-ID Assignment in P2PSIP } 
\fancyfoot[R]{\tiny \theslide} 
 
 
% To center horizontally the headers and footers (see seminar.bug) 
\renewcommand{\headwidth}{\textwidth} 

% don't center vertically
\centerslidesfalse 
% To adjust the frame length to the header and footer ones 
\autoslidemarginstrue 
\pagestyle{fancy} 
 

\newcommand{\heading}[1]{% 
  \begin{center} 
    \large\bf 
    #1 
  \end{center} 
  \vspace{.4 in}} 
\begin{document}

\begin{slide}
\begin{center}
\LARGE{{\bf}Secure Peer-ID Assignment in P2PSIP}\\

\vspace{.6 in}
\large{Eric Rescorla}\\
\large{Network Resonance}\\
\large{\url{ekr@networkresonance.com}}

\end{center}
\end{slide}



\begin{slide}
\heading{Background}

\begin{itemize}
\item Need to be able to authenticate that a peer has a given ID
\begin{itemize}
\item Otherwise a variety of routing attacks are possible
\end{itemize}
\item In practice, this means cryptography
\begin{itemize}
\item Need to bind peer-id $X$ to its public key
\item But how?
\end{itemize}

\end{itemize}
\end{slide}


\begin{slide}
\heading{Cryptographically Generated Peer-Ids}

\begin{itemize}
\item Peer generates a random key pair $K_{pub}, K_{priv}$
\begin{itemize}
\item $I = SHA1(K_{pub})$
\item This gives you a ``random'' peer-id
\begin{itemize}
\item Because of the SHA-1
\end{itemize}
\end{itemize}
\item How does authentication work?
\begin{itemize}
\item Peer signs something with $K_{priv}$ and sends $signature, K_{pub}, I$
\item Relying party verifies signature and that  $I = SHA1(K_{pub})$
\end{itemize}
\item This is the technique used in HIP
\end{itemize}
\end{slide}


\begin{slide}
\heading{Chosen Location Attacks}

\begin{itemize}
\item Attacker wants to get between $X$ and $predecessor(X)$
\begin{itemize}
\item A random node-id has a $1/N$ chance of being in $(pred(X),X)$
\begin{itemize}
\item Where $N$ is the number of nodes in the overlay
\item The size of the hashspace is irrelevant
\end{itemize}
\end{itemize}
\item An attacker can succeed in average $N/2$ trials
\begin{itemize}
\item This is an offline attack
\end{itemize}
\item Two basic countermeasures
\begin{itemize}
\item Slow down the search (but keep it offline)
\item Make it an online attack
\end{itemize}
\end{itemize}
\end{slide}


\begin{slide}
\heading{Proof of Work}

\vspace{-.3 in}
\begin{itemize}
\item Idea: make generating candidates expensive
\begin{itemize}
\item Example: partial preimage
\begin{itemize}
\item $PeerId = SHA1(X)$
\item Bottom $n$ bits of $PeerId$ must be zero
\begin{itemize}
\item Need to try average $2^{n-1}$ $X$ values to get a valid $PeerId$
\end{itemize}
\item This increases search cost by $2^{n-1}$
\end{itemize}
\end{itemize}
\item The puzzle must be tied to the peer-id
\begin{itemize}
\item Otherwise the attacker can solve the puzzle once and then generate many peer-ids
\item This is why CAPTCHAs are hard to deploy here
\end{itemize}
\item This only works well when the attacker isn't powerful
\begin{itemize}
\item ... by comparison to the average user
\item Not true with botnets
\end{itemize}
\end{itemize}
\end{slide}


\begin{slide}
\heading{Invitations}

\begin{itemize}
\item What if an existing peer asks you to join [MI07]
\begin{itemize}
\item You start as a client with peer-id X
\begin{itemize}
\item But you can't attack anyone since you're not a peer
\end{itemize}
\item The responsible peer invites you to become a peer
\begin{itemize}
\item Splits his zone of responsibility with you
\end{itemize}
\end{itemize}
\item Not clear how this helps
\begin{itemize}
\item Attacker chooses his peer-id
\begin{itemize}
\item Joins the overlay
\item Waits to be invited as a client
\end{itemize}
\item Also, how do you cryptographically bind key to peer-id
\end{itemize}
\end{itemize}

\end{slide}



\begin{slide}
\heading{Central Enrollment Server}

\begin{itemize}
\item We have a central server
\begin{itemize}
\item Joining peer contacts the server with his public key
\item Server validates peer somehow
\item Server issues a certificate with a random peer-id
\end{itemize}
\item This makes the attack online
\begin{itemize}
\item Even if no authentication is performed, you need a lot of queries to the server
\item If you have user authentication, then you only get one query
\end{itemize}
\end{itemize}

\end{slide}


\end{document}