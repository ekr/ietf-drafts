\documentclass[helvetica]{seminar} 
\input{xy}
\xyoption{all}
\usepackage{graphicx} 
\usepackage{slidesec} 
\usepackage{url}

\long\def\symbolfootnote[#1]#2{\begingroup%
\def\thefootnote{\fnsymbol{footnote}}\footnote[#1]{#2}\endgroup}

% to fix problems making landscape seminar pdfs
% Letter...
\pdfpagewidth=11truein
\pdfpageheight=8.5truein
\pdfhorigin=1truein     % default value(?), but doesn't work without
\pdfvorigin=1truein     % default value(?), but doesn't work without
% A4
%\pdfpagewidth=297truemm % your milage may vary....
%\pdfpageheight=210truemm
%\pdfhorigin=1truein     % default value(?), but doesn't work without
%\pdfvorigin=1truein     % default value(?), but doesn't work without



\renewcommand{\familydefault}{\sfdefault}  
 
\input{seminar.bug} 
\input{seminar.bg2} % See the Seminar bugs list 
 
\slideframe{none} 
 
 
\usepackage{fancyhdr} 
 
% Headers and footers personalization using the `fancyhdr' package 
\fancyhf{} % Clear all fields 
\renewcommand{\headrulewidth}{0mm} 
\renewcommand{\footrulewidth}{0.1mm} 
 
\fancyfoot[L]{\tiny IETF 79} 
\fancyfoot[C]{\tiny P2PSIP WG}
\fancyfoot[R]{\tiny \theslide} 
 
 
% To center horizontally the headers and footers (see seminar.bug) 
\renewcommand{\headwidth}{\textwidth} 

% To adjust the frame length to the header and footer ones 
\autoslidemarginstrue 
\pagestyle{fancy} 
 

\newcommand{\heading}[1]{% 
  \begin{center} 
    \large\bf 
    #1 
  \end{center} 
  \vspace{.4 in}} 

\begin{document}
\begin{slide}
\begin{center}
\vspace{1 in}
\LARGE{{\bf}RELOAD Status/Open Issues}\\
\large{\url{draft-ietf-p2psip-base-11}}\\
\large{{IETF 79}} \\
\vspace{3em}
\large{Eric Rescorla}\\
\large{\url{ekr@rtfm.com}}
\end{center}
\end{slide}


\centerslidesfalse 


\begin{slide}
\heading{Overall Status}

\begin{itemize}
\item draft -10 (Aug 3), draft-11 (Oct 12)
\begin{itemize}
\item Resolved most known open issues  
\item Thanks to Eric Burger for a detailed review
\end{itemize}
\item Second WGLC ended November 4
\begin{itemize}
\item Some minor new issues raised
\end{itemize}
\item General plan
\begin{itemize}
\item Resolve remaining issues here
\item Confirm on the list
\item Generate a finished draft by December 10
\end{itemize}
\end{itemize}
\end{slide}


\begin{slide}
\heading{Variable-length node-ids}

\begin{itemize}
\item Enacts WG consensus
\item Fixed per overlay
\item Range of 16-20 bytes
\item Set in configuration document
\end{itemize}

\end{slide}



\begin{slide}
\heading{Non-TLS security modes}

\begin{itemize}
\item Enacts WG consensus: (D)TLS for now with room for other
  prototocols in future
\item Requirements for future link protocols in \S 5.6.1:
\begin{itemize}
\item Endpoint authentication
\item Traffic origin authentication and integrity
\item Traffic confidentiality
\end{itemize}
\item Set in configuration document
\end{itemize}

\end{slide}


\begin{slide}
\heading{Direct Response Routing}

\begin{itemize}
\item Permitted on a single overlay basis
\item Set in configuration document
\end{itemize}
\end{slide}



\begin{slide}
\heading{Minor Changes}

\begin{itemize}
\item Provided a definition of \verb^AppAttachReq^ and \verb^AppAttachAns^ in \S 5.5.2.1 and 5.5.2.2.
\item no ICE $\rightarrow$ NoICE
\item Added a \verb^send_update^ flag to \verb^AttachReqAns^ to facilitate
  requests for immediate updates
\end{itemize}
\end{slide}

\begin{slide}
\heading{Minor Changes: RFC 2119 issues}

\begin{itemize}
\item Removed MUST-level requirement for generation counter on
  opaque \verb^Destination^ values as unenforceable [Eric Burger]
\item Made setting \verb^FORWARD_CRITICAL^ and \verb^DESTINATION_CRITICAL^
  MUST-level with \verb^DirectResponseForwardingOption^. (interop
  requirement)
\item Recipients now MAY process messages with unknown
  non-critical extensions (was SHOULD) [Eric Burger]
\item Clarified what the MUST requirement is for processing \verb^Attach^
  (you can refuse and throw an error) [Eric Burger]
\item Strengthened requirements on which STUN servers to use
  (MUST use one from the same group) in \S 5.5.1.4.
\end{itemize}
\end{slide}



\begin{slide}
\heading{Known Uncontroversial TODOs}

\begin{itemize}
\item Add padding to \verb^PING^ to facilitate MTU discovery
\item Rewrite/clarify leap-second text in \S 5.5.3.2
\end{itemize}
\end{slide}


\begin{slide}
\heading{ICE: Nomination Level}

\begin{itemize}
\item \S 5.5.1.10.2 formerly required regular nomination
\begin{itemize}
\item Regular nomination is quite a bit slower than aggressive
\item There are already a lot of round-trips
\end{itemize}
\item Original rationale was to ensure consistent state
\begin{itemize}
\item Don't believe this is needed: ICE naturally converges
\end{itemize}
\end{itemize}
\vspace{1em}
\textbf{Proposed Resolution:} Leave as-is in the draft
\end{slide}


\begin{slide}
\heading{Mandatory to Implement Signature/Hash Algorithms}

\begin{itemize}
\item None specified
\item Need some for interop
\end{itemize}

\vspace{1em}
\textbf{Proposed Resolution:} RSA with SHA-256
\end{slide}




\begin{slide}
\heading{Direct Response Routing and ICE}

\begin{itemize}
\item Specified in \S 5.3.2.4
\end{itemize}


{\footnotesize
\begin{verbatim}
   This option can only be used if the direct-return-response-permitted
   flag in the configuration for the overlay is set to TRUE.  The
   RESPONSE_COPY flag SHOULD be set to false while the FORWARD_CRITICAL
   and DESTINATION_CRITICAL MUST be set to true.  When a node that
   supports this forwarding options receives a request with it, it acts
   as if it had send an Attach request to the the requesting_node and it
   had received the connection_information in the answer.  This causes
   it to form a new connection directly to that node. 
\end{verbatim}
}

\begin{itemize}
\item This doesn't work with ICE because the sender of the request
  doesn't have your information
\end{itemize}

\vspace{1em}
\textbf{Proposed Resolution:} DRR can only be used with No-ICE
\end{slide}


\begin{slide}
\heading{Node-Ids in \texttt{JOIN/LEAVE}}

\begin{itemize}
\item Currently \verb^JoinReq^ and \verb^LeaveReq^ have the joining 
  Node-Id
\end{itemize}

{\footnotesize
\begin{verbatim}
  struct {
     NodeId                joining_peer_id;
     opaque                overlay_specific_data<0..2^16-1>;
  } JoinReq;
\end{verbatim}
}

\begin{itemize}
\item This is unnecessary because the Node-Id is provided by the security block.
\item Just one more thing to check
\end{itemize}

\vspace{1em}
\textbf{Proposed Resolution:} It's annoying but harmless, so in the interest
of compatibility leave it in but clarify that a check is required.
\end{slide}



\begin{slide}
\heading{Specifying Counter Values for \texttt{NODE-MULTIPLE}}


\vspace{-2em}
\S 6.3.4:

{\footnotesize
\begin{verbatim}
   In the NODE-MULTIPLE policy, a given value MUST be written (or
   overwritten) if and only if the request is signed with a key
   associated with a certificate containing a Node-ID such that
   H(Node-ID || i) is equal to the Resource-ID for some small integer
   value of i.  When this policy is in use, the maximum value of i MUST
   be specified in the kind definition.
\end{verbatim}
}

\begin{itemize}
\item \verb^i^ is not carried on the wire anywhere
\item Maximum value is specified in the configuration document
\item Possible approaches
\begin{itemize}
\item Verifier iterates through \verb^i^ values (not that slow but annoying)
\item Add syntax to carry \verb^i^ (kind of a gross special case)
\end{itemize}
\end{itemize}

\vspace{1em}
\textbf{Proposed Resolution:} Verifier iterates (with regrets)
\end{slide}


\begin{slide}
\heading{Pings while Joining (\S 9.4)}

\begin{itemize}
\item Current procedure requires sending Pings to populate the table (step 2)
\item These are unnecessary since Attach automatically discovers the right node
\end{itemize}

\vspace{1em}
\textbf{Proposed Resolution:} Remove Pings as proposed on-list by BBL (Nov 1)
\end{slide}



\begin{slide}
\heading{Join race condition I (Michael Chen)}

\begin{itemize}
\item \S 9.4:
\begin{itemize}
\item Step 7: routing table from AP $\rightarrow$ JP
\item Step 8: routing table from AP $\rightarrow$ NP
\end{itemize}
\item In some cases (e.g., Chord predecessors) this may cause
  simultaneous connects between JP and it's new neighbors
\end{itemize}

\vspace{1em}
\textbf{Proposed Resolution:} Tiebreaker when multiple connections are established
between a pair of nodes. Terminate the connection \emph{originating} from the
smaller Node-Id seems like a natural choice.

\end{slide}


\begin{slide}
\heading{Join Attach timing (Michael Chen)}

\begin{itemize}
\item Proposal is to skip step 3 in which JP sends Attaches to 
  its expected nodes.
\item Argument for this is that the logic is simpler since no need
  to do incremental probing.
\item Argument against is that it then takes longer to get fully 
  established. Client has multiple ways to get AP's routing table
  which would allow unified logic for the neighbor set.
\end{itemize}

\vspace{1em}
\textbf{Proposed Resolution:} Leave as-is but add discussion of the
option to get AP's routing table rather than probe.
\end{slide}


\end{document}