\documentclass[helvetica]{seminar} 
\input{xy}
\xyoption{all}
\usepackage{graphicx} 
\usepackage{slidesec} 
\usepackage{url}
\usepackage[framemethod=TikZ]{mdframed}
\usepackage{color}

\long\def\symbolfootnote[#1]#2{\begingroup%
\def\thefootnote{\fnsymbol{footnote}}\footnote[#1]{#2}\endgroup}

% to fix problems making landscape seminar pdfs
% Letter...
\pdfpagewidth=11truein
\pdfpageheight=8.5truein
\pdfhorigin=1truein     % default value(?), but doesn't work without
\pdfvorigin=1truein     % default value(?), but doesn't work without
% A4
%\pdfpagewidth=297truemm % your milage may vary....
%\pdfpageheight=210truemm
%\pdfhorigin=1truein     % default value(?), but doesn't work without
%\pdfvorigin=1truein     % default value(?), but doesn't work without


\renewcommand{\familydefault}{\sfdefault}  
 
\input{seminar.bug} 
\input{seminar.bg2} % See the Seminar bugs list 
 
\slideframe{none} 
 
 
\usepackage{fancyhdr} 
 
% Headers and footers personalization using the `fancyhdr' package 
\fancyhf{} % Clear all fields 
\renewcommand{\headrulewidth}{0mm} 
\renewcommand{\footrulewidth}{0.1mm} 
 
\fancyfoot[L]{\tiny IETF 94} 
\fancyfoot[C]{\tiny TLS}
\fancyfoot[R]{\tiny \theslide} 
 
 
% To center horizontally the headers and footers (see seminar.bug) 
\renewcommand{\headwidth}{\textwidth} 

% To adjust the frame length to the header and footer ones 
\autoslidemarginstrue 
\pagestyle{fancy} 
 

\newcommand{\heading}[1]{% 
  \begin{center} 
    \large\bf 
    #1 
  \end{center} 
  \vspace{.4 in}} 



\begin{document}

\begin{slide}
\begin{center}
\vspace{.5 in}
\LARGE{{\bf}TLS 1.3 Status\\draft-10}\\
\vspace{.2in}
\large{
\begin{tabular}{c}
Eric Rescorla\\
Mozilla\\
\url{ekr@rtfm.com}
\end{tabular}
}
\end{center}

\end{slide}

\centerslidesfalse 


\begin{slide}
\heading{Overview}

\begin{itemize}
\item Changes since IETF 93 (Prague)
\item Client authentication (PR\#316)
\item HelloRetryRequest (Issues \#104, \#185)
\item 0-RTT framing (\#311, \#295)
\item Re-key (\#4, \#125)
\item Exporters (\#282)
\end{itemize}

\end{slide}

\begin{slide}
\heading{Changes Since IETF 93 (II)}

\begin{itemize}
\item Always require digital signatures from the server with public-key cipher suites
  \begin{itemize}
  \item ...even with 0-RTT
  \end{itemize}

\item Relaxed certificate selection rules *
\item Deprecated a lot of algorithms *
\item Encrypted content type *
\item Built-in record padding *
\item Improved CertificateRequest syntax
\end{itemize}
\end{slide}


\begin{slide}
\heading{Changes Since IETF 93 (II)}

\begin{itemize}
\item Update key schedule
\item Added MTI algorithms
\item Reduced maximum record expansion
\item Extensionsify ServerKeyShare
\item Assorted editorial stuff
\end{itemize}
\end{slide}


\begin{slide}
\heading{Relaxed Certificate Selection Rules}

\begin{itemize}
\item TLS 1.2 requires that certificates appear in order
  \begin{itemize}
  \item Many servers don't do this
    \begin{itemize}
    \item Not always possible
    \end{itemize}
  \item Many clients try to construct the path anyway
  \item Updated draft to encourage but not require this
  \end{itemize}

\item TLS 1.2 required that server certificates conform to SignatureAlgorithms
  \begin{itemize}
  \item But what if the only cert you have doesn't match?
  \item Draft now allows you to send it in that case 
    \begin{itemize}
    \item ...but only if you have to
    \end{itemize}
  \end{itemize}
\end{itemize}
\end{slide}


\begin{slide}
\heading{Deprecated Algorithms} 

\begin{itemize}
\item Forbid MD5 (and SHA-224)
\item Forbid SHA-1 in CertificateVerify
\item Removed DSA
\item Switched to PSS (more on this later)
\item Removed a lot of old EC groups
\end{itemize}

\end{slide}

\begin{slide}
\heading{Encrypted Content Type and Padding}

\begin{footnotesize}
\begin{verbatim}
   struct {
       ContentType opaque_type = application_data(23); /* see fragment.type */
       ProtocolVersion record_version = { 3, 1 };    /* TLS v1.x */
       uint16 length;
       aead-ciphered struct {
          opaque content[TLSPlaintext.length];
          ContentType type;
          uint8 zeros[length_of_padding];
       } fragment;
   } TLSCiphertext;
\end{verbatim}
\end{footnotesize}

\begin{itemize}
\item This allows padding
\item But doesn't require it
\item Receiver behaves the same either way
\end{itemize}
\end{slide}


\begin{slide}
\heading{Improved CertificateRequest Syntax (Popov)}

\vspace{-3ex}
\begin{footnotesize}
\begin{verbatim}
      struct {
          opaque certificate_extension_oid<1..2^8-1>;
          opaque certificate_extension_values<0..2^16-1>;
      } CertificateExtension;

      struct {
          SignatureAndHashAlgorithm
            supported_signature_algorithms<2..2^16-2>;
          DistinguishedName certificate_authorities<0..2^16-1>;
          CertificateExtension certificate_extensions<0..2^16-1>;
      } CertificateRequest;
\end{verbatim}
\end{footnotesize}

\begin{itemize}
\item Extensions correspond to X.509v3 extensions in the EE certificate
\item Each extension has its own matching rule
  \begin{itemize}
  \item KeyUsage and EKU defined in this document
  \end{itemize}
\item Client can ignore any unrecognized extensions
\end{itemize}

\end{slide}




\end{document}  
