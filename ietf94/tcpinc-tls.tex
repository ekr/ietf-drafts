\documentclass[helvetica]{seminar} 
\input{xy}
\xyoption{all}
\usepackage{graphicx} 
\usepackage{slidesec} 
\usepackage{url}
\usepackage[framemethod=TikZ]{mdframed}
\usepackage{color}

\long\def\symbolfootnote[#1]#2{\begingroup%
\def\thefootnote{\fnsymbol{footnote}}\footnote[#1]{#2}\endgroup}

% to fix problems making landscape seminar pdfs
% Letter...
\pdfpagewidth=11truein
\pdfpageheight=8.5truein
\pdfhorigin=1truein     % default value(?), but doesn't work without
\pdfvorigin=1truein     % default value(?), but doesn't work without
% A4
%\pdfpagewidth=297truemm % your milage may vary....
%\pdfpageheight=210truemm
%\pdfhorigin=1truein     % default value(?), but doesn't work without
%\pdfvorigin=1truein     % default value(?), but doesn't work without


\renewcommand{\familydefault}{\sfdefault}  
 
\input{seminar.bug} 
\input{seminar.bg2} % See the Seminar bugs list 
 
\slideframe{none} 
 
 
\usepackage{fancyhdr} 
 
% Headers and footers personalization using the `fancyhdr' package 
\fancyhf{} % Clear all fields 
\renewcommand{\headrulewidth}{0mm} 
\renewcommand{\footrulewidth}{0.1mm} 
 
\fancyfoot[L]{\tiny IETF 94} 
\fancyfoot[C]{\tiny TCPINC Use TLS}
\fancyfoot[R]{\tiny \theslide} 
 
 
% To center horizontally the headers and footers (see seminar.bug) 
\renewcommand{\headwidth}{\textwidth} 

% To adjust the frame length to the header and footer ones 
\autoslidemarginstrue 
\pagestyle{fancy} 
 

\newcommand{\heading}[1]{% 
  \begin{center} 
    \large\bf 
    #1 
  \end{center} 
  \vspace{.4 in}} 



\begin{document}

\begin{slide}
\begin{center}
\vspace{.5 in}
\LARGE{{\bf}TCP-use-TLS}\\
\vspace{.2in}
\large{
\begin{tabular}{c}
Eric Rescorla\\
Mozilla\\
\url{ekr@rtfm.com}
\end{tabular}
}
\end{center}

\end{slide}
\centerslidesfalse 

\begin{slide}
\heading{Basic idea}

\begin{itemize}
\item Use ENO to negotiate use of the \_spec\_ defined in this draft
\item Run TLS as usual over TCP once ENO negotiates it
\item Profile TLS down to a small subset of TLS 1.3
  \begin{itemize}
  \item But also allow TLS 1.2 (see below)
  \end{itemize}
\item Intuition: if you ignore ENO extensions this looks just like TLS-over-TCP
\end{itemize}

\end{slide}


\begin{slide}
\heading{Minimal SYN Suboption}

\begin{verbatim}
    bit   7   6   5   4   3   2   1   0
        +---+---+---+---+---+---+---+---+
        | 0 |           TBD             |
        +---+---+---+---+---+---+---+---+
\end{verbatim}
\end{slide}

\begin{slide}
\heading{SYN/ACK Suboptions}

\begin{verbatim}
One-byte   +--------+
1-RTT      |   TBD  |
Only       +--------+  


Variable   +--------+---------//----------+
1-RTT or   |1| TBD  |       Nonce         |
0-RTT      +--------+---------//----------+
\end{verbatim}

\begin{itemize}
\item Warning: nonce thing may be half-baked; more on this later
\end{itemize}
\end{slide}


\begin{slide}
\heading{TLS 1.3 Profile Overview}

\begin{itemize}
\item ECDHE only (MUST P256, SHOULD X25519)
\item Limited set of symmetric ciphers
\item Support for raw public keys (avoid need for X.509 certs)
\item No client authentication
\item No resumption
\end{itemize}
\end{slide}

\begin{slide}
\heading{TLS 1.3 Handshake in TCPINC (1-RTT)}

\begin{scriptsize}
\begin{verbatim}
     SYN + TCP-ENO [TLS]       ------->
                               <-------  SYN/ACK + TCP-ENO [TLS]
     ACK + TCP-ENO             ------->
     ClientHello
       + ClientKeyShare
       + TCPENOTranscript      ------->
                                                     ServerHello
                                                  ServerKeyShare
                                           {EncryptedExtensions}
                                          {ServerConfiguration*}
                                                   {Certificate}
                                             {CertificateVerify}
                               <--------              {Finished}
                               <--------      [Application Data]
     {Finished}                -------->
     [Application Data]        <------->      [Application Data]
\end{verbatim}
\end{scriptsize}
\end{slide}


\begin{slide}
\heading{0-RTT}

\begin{itemize}
\item In initial handshake, server provides long-term ECDHE key in ServerConfiguration
\item In subsequent handshake, client can encrypt using that key
  \begin{itemize}
  \item Obviously this doesn't provide PFS
  \item There are replay issues in stock TLS 1.3
  \item ... but here we have a nonce
  \end{itemize}
\item 0-RTT currently required to be manually configured
\end{itemize}

\end{slide}



\begin{slide}
\heading{TLS 1.3 Handshake in TCPINC (0-RTT)}

\vspace{-5ex}
\begin{scriptsize}
\begin{verbatim}
     SYN + TCP-ENO [TLS]       ------->
                               <-------        SYN/ACK + TCP-ENO
                                                   [TLS + Nonce]
     ACK + TCP-ENO             ------->
     ClientHello
       + ClientKeyShare
       + EarlyDataIndication
       + TCPENOTranscript
     (EncryptedExtensions)
     (Finished)
     (Application Data)        -------->
                                                     ServerHello
                                           + EarlyDataIndication
                                                  ServerKeyShare
                                           {EncryptedExtensions}
                                          {ServerConfiguration*}
                                                   {Certificate}
                                             {CertificateVerify}
                               <--------              {Finished}
                               <--------      [Application Data]
     {Finished}                -------->
     [Application Data]        <------->      [Application Data]
\end{verbatim}
\end{scriptsize}
\end{slide}


\begin{slide}
\heading{Session IDs}

\begin{itemize}
\item Computed as TLS Exporter [RFC 5705]
\item Label is TBD
\item TCP-ENO transcript bound in via TCPENOTranscript extension
\end{itemize}
\end{slide}


\begin{slide}
\heading{Open issues}

\begin{itemize}
\item Ability to negotiate TLS 1.2 as well
  \begin{itemize}
  \item This is useful, though perhaps not for TCPINC use case
  \item Some suggestions to define two code points
  \end{itemize}

\item Should the client be able to say it wants 0-RTT?
  \begin{itemize}
  \item How would this help?
  \end{itemize}

\item Interaction with TFO
\end{itemize}
\end{slide}

\begin{slide}
\heading{Questions?}

\end{slide}
\end{document} 
                