\documentclass[helvetica]{seminar} 
\input{xy}
\xyoption{all}
\usepackage{graphicx} 
\usepackage{slidesec} 
\usepackage{url}
\usepackage{multicol}
\usepackage[framemethod=TikZ]{mdframed}

\long\def\symbolfootnote[#1]#2{\begingroup%
\def\thefootnote{\fnsymbol{footnote}}\footnote[#1]{#2}\endgroup}

% to fix problems making landscape seminar pdfs
% Letter...
\pdfpagewidth=11truein
\pdfpageheight=8.5truein
\pdfhorigin=1truein     % default value(?), but doesn't work without
\pdfvorigin=1truein     % default value(?), but doesn't work without
% A4
%\pdfpagewidth=297truemm % your milage may vary....
%\pdfpageheight=210truemm
%\pdfhorigin=1truein     % default value(?), but doesn't work without
%\pdfvorigin=1truein     % default value(?), but doesn't work without



\renewcommand{\familydefault}{\sfdefault}  
 
\input{seminar.bug} 
\input{seminar.bg2} % See the Seminar bugs list 
 
\slideframe{none} 
 
 
\usepackage{fancyhdr} 
 
% Headers and footers personalization using the `fancyhdr' package 
\fancyhf{} % Clear all fields 
\renewcommand{\headrulewidth}{0mm} 
\renewcommand{\footrulewidth}{0.1mm} 
 
\fancyfoot[L]{\tiny TPAC 2013} 
\fancyfoot[C]{\tiny Identity; Security; API}
\fancyfoot[R]{\tiny \theslide} 
 
 
% To center horizontally the headers and footers (see seminar.bug) 
\renewcommand{\headwidth}{\textwidth} 

% To adjust the frame length to the header and footer ones 
\autoslidemarginstrue 
\pagestyle{fancy} 
 

\newcommand{\heading}[1]{% 
  \begin{center} 
    \large\bf 
    #1 
  \end{center} 
  \vspace{.4 in}} 

\begin{document}

\begin{slide}
\begin{center}
\vspace{1 in}
\vspace{.2in}
\large{{Identity, Security, etc. API Issues}} \\
\vspace{3em}
\begin{tabular}{c}
Eric Rescorla \\
\url{ekr@rtfm.com}
\end{tabular}
}
\end{center}

\end{slide}


\centerslidesfalse 


\begin{slide}
\heading{Overview of Topics}

\begin{itemize}
\item DTLS
\begin{itemize}
  \item Controlling my own DTLS key
  \item Examining remote DTLS parameters
\end{itemize}

\item Identity
  \begin{itemize}
  \item Examining my own identity
  \item Channel between chrome and content
  \item \verb^noaccess^ and \verb^peerIdentity^ constraints
  \end{itemize}
\end{itemize}
\end{slide}


\begin{slide}
\heading{DTLS Key Control Requirements}

\begin{itemize}
\item Keys are scoped to origin
\item Be able to use the same key repeatedly
  \begin{itemize}
  \item Avoid repeatedly generating keys
  \item Enable key continuity/auditing
  \end{itemize}

\item Be able to use multiple distinct keys
\item Be able to generate a temporary key
\item[]
\item Application needs to be able to control this
\end{itemize}
\end{slide}


\begin{slide}
\heading{DtlsKeyName Constraint}

\vspace{-.25in}
\begin{verbatim}
{
  mandatory : [
   {
     DtlsKeyName : "ekr@example.com"
   }
  ] 
}
\end{verbatim}

\begin{itemize}
\item DTLS Keys are stored under \verb^DtlsKeyName^ value $D$
\item If no key exists with name $D$ it is made and stored
\item If key exists with name $D$ that key is reused
\item ``falsy'' (\verb^false^, \verb^null^, ...) \verb^DtlsKeyName^ values never match anything
  \begin{itemize}
    \item ... this means make a fresh key pair for this call
  \end{itemize}
\end{itemize}
\end{slide}


\begin{slide}
\heading{DTLS Key Control using WebCrypto}

\begin{itemize}
\item JS creates a key using WebCrypto
  \begin{itemize}
  \item \verb^pc.setDtlsKey()^ API call to impose the key
  \end{itemize}

\item JS is responsible for figuring out what keys to use
  \begin{itemize}
    \item Keys can be stored using usual WebCrypto mechanisms (\verb^wrap()^, \verb^unwrap()^, etc.)
  \end{itemize}
\end{itemize}
\end{slide}

\begin{slide}
\heading{WebCrypto Example}

\begin{multicols}{2}
{\tiny
\begin{verbatim}
function new_key(label){ 
  // Algorithm Object
  var algorithmKeyGen = {
    name: "RSASSA-PKCS1-v1_5",
    // RsaKeyGenParams
    modulusLength: 2048,
    publicExponent: new Uint8Array([0x01, 0x00, 0x01]),
  };
  
  window.crypto.subtle.generateKey(
     algorithmKeyGen,
     false, ["peerconnection"]).then(
    function(key) {
      index.put(key, label);
      pc.setDtlsKey(key);
    }
  );
};

function set_key(label) {
  var req = index.get(label);
  req.onsuccess(
    function() {
      if (req.result === undefined) {
        new_key(label);
      }
      else {
        pc.setDtlsKey(req.result);
      }
    }
  );
}

set_key("ekr-key");



\end{verbatim}
}
\end{multicols}
\end{slide}


\begin{slide}
\heading{What about the other side's public key}

\begin{itemize}
\item Would be nice to know the other side's public key
  \begin{itemize}
    \item For key continuity
  \end{itemize}

\item We {Justin, Martin, EKR} went back and forth on this
  \begin{itemize}
  \item And decided that less is more
  \end{itemize}

\item Proposal: a binary version of the other side's keys
\end{itemize}

\end{slide}


\begin{slide}
\heading{New API}

\begin{itemize}
\item \verb^pc.remoteCertificates^ contains a list of other side's certificate chain
  \begin{itemize}
    \item As ArrayBuffer
  \end{itemize}

\item The raw certificate can just be used as a lookup key
  \begin{itemize}
    \item ... or parsed with WebCrypto (when available)
  \end{itemize}

\item No claims about the browser's opinion of the certificates
\end{itemize}
\end{slide}


\begin{slide}
\heading{Recap: remote identity}

\begin{itemize}
\item Remote identity is directly observable
\end{itemize}

\begin{verbatim}
dictionary RTCIdentityAssertion {
    DOMString idp;
    DOMString name;
    // Extensible
};
\end{verbatim}

\begin{itemize}
\item Stored as \verb^pc.peerIdentity^
\end{itemize}

\end{slide}


\begin{slide}
\heading{What about my own identity?}

\begin{itemize}
\item Would be nice to be able to observe this
\item We have \verb^pc.onidentityresult^ to notify when assertion obtained
  \begin{itemize}
    \item It doesn't have a defined argument (``TODO'')
  \end{itemize}
\end{itemize}
\end{slide}


\begin{slide}
\heading{Proposal}

\begin{itemize}
\item \verb^onidentityresult^ takes a \verb^RTCIdentity^ argument corresponding to the obtained identity
\item Rename \verb^peerIdentity^ to \verb^remoteIdentity^ to match \verb^remoteDescription^
\item \verb^localIdentity^ contains my own identity (can be \verb^null^)
\end{itemize}
\end{slide}

\begin{slide}
\heading{Message Channel between chrome and content}

\begin{quote}
\emph{``The context must have a MessageChannel named window.TBD which is "entangled" to the RTCPeerConnection and is unique to that subcontext. This channel is used for messaging between the RTCPeerConnection and the IdP. All messages sent via this channel are strings, specifically the JSONified versions of JavaScript structs.''}\end{quote}

\begin{itemize}
\item This works fine in current Firefox implementation (landing soon)
\item What should ``TBD'' be?
\begin{itemize}
\item Proposal: \verb^identityMessageChannel^ (but I don't care)
\end{itemize}
\end{itemize}
\end{slide}

\begin{slide}
\heading{\texttt{noaccess} and \texttt{peerIdentity}}

\begin{itemize}
\item Current status
  \begin{itemize}
    \item Can't attach MediaStream to inappropriate sinks
    \item ... generates errors
  \end{itemize}

\item New proposal from Martin
  \begin{itemize}
    \item Can attach anything anywhere
    \item But unauthorized sinks just get silence/black
      \begin{itemize}
        \item Rules for ``authorized'' remain the same
      \end{itemize}
    \item Need API flag for ``authorized''; propose read-only \verb^.accessible^
  \end{itemize}
\end{itemize}

\end{slide}

\begin{slide}
\heading{Modified Permissions Model}


\begin{itemize}
\item Allow JS to get a \verb^noaccess^ stream w/o any permissons
  \begin{itemize}
  \item Can map into video/audio tag
  \item Usable for ``hair check''
  \end{itemize}

\item Permissions check when constraints change
  \begin{itemize}
  \item This means we need a \verb^.onaccessiblechange^ event
  \end{itemize}

\item This is generally more flexible
  \begin{itemize}
    \item But arguably more creepy
  \end{itemize}
\end{itemize}
\end{slide}



\centerslidestrue


\begin{slide}
\begin{center}
\vspace{1 in}
\vspace{.2in}
\large{{What happens when you run out of resources?}} \\
\vspace{3em}
\begin{tabular}{c}
Eric Rescorla \\
\url{ekr@rtfm.com}
\end{tabular}
}
\end{center}

\centerslidesfalse 

\begin{slide}
\heading{Possible cases (mostly hardware limitations)}

\begin{itemize}
\item Fixed number of HW decoders
\item Fixed total number of HW resources (e.g., macroblocks)
\item Total CPU limitations
\end{itemize}
\end{slide}

\begin{slide}
\heading{When do you know?}

\begin{itemize}
\item \verb^AddStream()^ (for encoders, but not sure)
\item \verb^CreateOffer(), CreateAnswer()^ (can I create a valid offer with the known resources)
\item \verb^SetRemote()^ (did the other side ask to send me more than I can process)
\item When media starts to come in (if I over-allocate)???
\end{itemize}
\end{slide}

\begin{slide}
\heading{Proposed processing model}

\begin{itemize}
\item Iterate over tracks \emph{in order}
\item Add any track which can be added
\begin{itemize}
\item Assume maximum possible resource consumption for that track
\end{itemize}
\item Skip any track which cannot
\end{itemize}
\end{slide}

\begin{slide}
\heading{Example 1}

\begin{itemize}
\item I can receive one HD stream and one SD stream
\item Other side offers HD1, HD2, SD
  \begin{itemize}
    \item I accept HD1, SD
  \end{itemize}

\item Other side offers HD1, SD, HD2
  \begin{itemize}
    \item I accept HD1, SD
  \end{itemize}
\end{itemize}
\end{slide}


\begin{slide}
\heading{Example 2}

\begin{itemize}
\item I can receive one HD stream \emph{or} two SD streams
\item Other side offers HD, SD1, SD2
  \begin{itemize}
    \item I accept HD1
  \end{itemize}

\item Other side offers SD1, HD, SD2
  \begin{itemize}
    \item I accept SD1, SD2
  \end{itemize}
\end{itemize}

\end{slide}


\begin{slide}
\heading{How do I find out what happened?}

\begin{itemize}
\item For \verb^CreateOffer()^, 
  \begin{itemize}
  \item Just negotiate the maximum possible
    \item Maybe can introspect with Stefan's doohickey
  \end{itemize}

\item For \verb^SetRemoteDescription()^ there is no notification
  \begin{itemize}
    \item Just negotiate the maximum possible
    \item Not possible to introspect into what didn't get accepted
  \end{itemize}
\end{itemize}

\end{slide}

\end{document}
