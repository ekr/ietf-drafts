\documentclass[helvetica]{seminar} 
\input{xy}
\xyoption{all}
\usepackage{graphicx} 
\usepackage{slidesec} 
\usepackage{url}
\usepackage[framemethod=TikZ]{mdframed}

\long\def\symbolfootnote[#1]#2{\begingroup%
\def\thefootnote{\fnsymbol{footnote}}\footnote[#1]{#2}\endgroup}

% to fix problems making landscape seminar pdfs
% Letter...
\pdfpagewidth=11truein
\pdfpageheight=8.5truein
\pdfhorigin=1truein     % default value(?), but doesn't work without
\pdfvorigin=1truein     % default value(?), but doesn't work without
% A4
%\pdfpagewidth=297truemm % your milage may vary....
%\pdfpageheight=210truemm
%\pdfhorigin=1truein     % default value(?), but doesn't work without
%\pdfvorigin=1truein     % default value(?), but doesn't work without



\renewcommand{\familydefault}{\sfdefault}  
 
\input{seminar.bug} 
\input{seminar.bg2} % See the Seminar bugs list 
 
\slideframe{none} 
 
 
\usepackage{fancyhdr} 
 
% Headers and footers personalization using the `fancyhdr' package 
\fancyhf{} % Clear all fields 
\renewcommand{\headrulewidth}{0mm} 
\renewcommand{\footrulewidth}{0.1mm} 
 
\fancyfoot[L]{\tiny TPAC 2013} 
\fancyfoot[C]{\tiny Identity; Security; API}
\fancyfoot[R]{\tiny \theslide} 
 
 
% To center horizontally the headers and footers (see seminar.bug) 
\renewcommand{\headwidth}{\textwidth} 

% To adjust the frame length to the header and footer ones 
\autoslidemarginstrue 
\pagestyle{fancy} 
 

\newcommand{\heading}[1]{% 
  \begin{center} 
    \large\bf 
    #1 
  \end{center} 
  \vspace{.4 in}} 

\begin{document}

\begin{slide}
\begin{center}
\vspace{1 in}
\vspace{.2in}
\large{{Identity, Security, etc. API Issues}} \\
\vspace{3em}
\begin{tabular}{c}
Eric Rescorla \\
\url{ekr@rtfm.com}
\end{tabular}
}
\end{center}

\end{slide}


\centerslidesfalse 


\begin{slide}
\heading{Overview of Topics}

\begin{itemize}
\item DTLS
\begin{itemize}
  \item Controlling my own DTLS key
  \item Examining remote DTLS parameters
\end{itemize}

\item Identity
  \begin{itemize}
  \item Examining my own identity
  \item Channel between Chrome and content
  \end{itemize}
\end{itemize}
\end{slide}


\begin{slide}
\heading{DTLS Key Control Requirements}

\begin{itemize}
\item Keys are scoped to origin
\item Be able to use the same key repeatedly
  \begin{itemize}
  \item Avoid repeatedly generating keys
  \item Enable key continuity/auditing
  \end{itemize}

\item Be able to use multiple distinct keys
\item Be able to generate a temporary key
\item[]
\item Application needs to be able to control this
\end{itemize}
\end{slide}


\begin{slide}
\heading{DtlsIdentity Constraint}

\vspace{-.25in}
\begin{verbatim}
{
  mandatory : [
   {
     DtlsIdentity : "ekr@example.com"
   }
  ] 
}
\end{verbatim}

\begin{itemize}
\item DTLS Keys are stored under \verb^DtlsIdentity^ value $D$
\item If no key exists with name $D$ it is made and stored
\item If key exists with name $D$ that key is reused
\item ``falsy'' (\verb^false^, \verb^null^, ...) \verb^DtlsIdentity^ values never match anything
  \begin{itemize}
    \item ... this means make a fresh key pair for this call
  \end{itemize}
\end{itemize}
\end{slide}


\begin{slide}
\heading{Alternative Design: use WebCrypto}

\begin{itemize}
\item JS creates a WebCrypto key
  \begin{itemize}
  \item \verb^pc.setDtlsKey()^ API call to impose the key
  \item JS is responsible for figuring out what keys to use
  \end{itemize}

\item Problem: private key needs to be unavailable to JS
  \begin{itemize}
    \item Otherwise Identity isn't secure
  \end{itemize}

\item WebCrypto keys can be marked unexportable
  \begin{itemize}
  \item But this doesn't mean an unexportable key was never known
  \end{itemize}

\item This is going to need a bunch of WebCrypto bookkeeping that doesn't exist 
yet
\begin{itemize}
\item Has this private key \emph{ever} been available to the JS
\end{itemize}
\end{itemize}

\end{slide}


\begin{slide}
\heading{What about the other side's public key}

\begin{itemize}
\item Would be nice to know the other side's public key
  \begin{itemize}
    \item For key continuity
  \end{itemize}

\item We {Justin, Martin, EKR} went back and forth on this
  \begin{itemize}
  \item And decided that less is more
  \end{itemize}

\item Proposal: a binary version of the other side's keys
\end{itemize}

\end{slide}


\begin{slide}
\heading{New API}

\begin{itemize}
\item \verb^pc.remoteCertificates^ contains a list of other side's certificates
  \begin{itemize}
    \item As base64-encoded (?) blobs
  \end{itemize}

\item The raw certificate can just be used as a lookup key
  \begin{itemize}
    \item ... or parsed with WebCrypto
  \end{itemize}

\item No claims about the browser's opinion of the certificates
\end{itemize}
\end{slide}


\begin{slide}
\heading{Recap: remote identity}

\begin{itemize}
\item Remote identity is directly observable
\end{itemize}

\begin{verbatim}
dictionary RTCIdentityAssertion {
    DOMString idp;
    DOMString name;
};
\end{verbatim}

\begin{itemize}
\item Stored as \verb^pc.peerIdentity^
\end{itemize}

\end{slide}


\begin{slide}
\heading{What about my own identity?}

\begin{itemize}
\item Would be nice to be able to observe this
\item We have \verb^pc.onidentityresult^ to notify when assertion obtained
  \begin{itemize}
    \item It doesn't have a defined argument (``TODO'')
  \end{itemize}
\end{itemize}
\end{slide}


\begin{slide}
\heading{Proposal}

\begin{itemize}
\item \verb^onidentityresult^ takes a \verb^RTCIdentity^ argument corresponding to the obtained identity
\item Rename \verb^peerIdentity^ to \verb^remoteIdentity^ to match \verb^remoteDescription^
\item \verb^localIdentity^ contains my own identity (can be \verb^null^)
\end{itemize}
\end{slide}

\begin{slide}
\heading{Message Channel between chrome and content}

\begin{quote}
\emph{``The context must have a MessageChannel named window.TBD which is "entangled" to the RTCPeerConnection and is unique to that subcontext. This channel is used for messaging between the RTCPeerConnection and the IdP. All messages sent via this channel are strings, specifically the JSONified versions of JavaScript structs.''}\end{quote}

\begin{itemize}
\item This works fine in current Firefox implementation (landing soon)
\item What should ``TBD'' be?
\begin{itemize}
\item Proposal: \verb^rtcwebMessageChannel^ (but I don't care)
\end{itemize}
\end{itemize}

\end{slide}

\end{document}
