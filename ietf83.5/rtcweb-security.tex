\documentclass[helvetica]{seminar} 
\input{xy}
\xyoption{all}
\usepackage{graphicx} 
\usepackage{slidesec} 
\usepackage{url}
\usepackage[normalem]{ulem}
\usepackage[usenames]{color}
\usepackage{algpseudocode}

\long\def\symbolfootnote[#1]#2{\begingroup%
\def\thefootnote{\fnsymbol{footnote}}\footnote[#1]{#2}\endgroup}

% to fix problems making landscape seminar pdfs
% Letter...
\pdfpagewidth=11truein
\pdfpageheight=8.5truein
\pdfhorigin=1truein     % default value(?), but doesn't work without
\pdfvorigin=1truein     % default value(?), but doesn't work without
% A4
%\pdfpagewidth=297truemm % your milage may vary....
%\pdfpageheight=210truemm
%\pdfhorigin=1truein     % default value(?), but doesn't work without
%\pdfvorigin=1truein     % default value(?), but doesn't work without



\renewcommand{\familydefault}{\sfdefault}  
 
\input{seminar.bug} 
\input{seminar.bg2} % See the Seminar bugs list 
 
\slideframe{none} 
 
 
\usepackage{fancyhdr} 
 
% Headers and footers personalization using the `fancyhdr' package 
\fancyhf{} % Clear all fields 
\renewcommand{\headrulewidth}{0mm} 
\renewcommand{\footrulewidth}{0.1mm} 
 
\fancyfoot[L]{\tiny IETF 83} 
\fancyfoot[C]{\tiny RTCWeb Security}
\fancyfoot[R]{\tiny \theslide} 
 
 
% To center horizontally the headers and footers (see seminar.bug) 
\renewcommand{\headwidth}{\textwidth} 

% To adjust the frame length to the header and footer ones 
\autoslidemarginstrue 
\pagestyle{fancy} 
 

\newcommand{\heading}[1]{% 
  \begin{center} 
    \large\bf 
    #1 
  \end{center} 
  \vspace{.4 in}} 

\begin{document}

\centerslidestrue

\begin{slide}
\begin{center}
\vspace{1 in}
\LARGE{{\bf}RTCWEB Security}\\
\vspace{.2in}
\large{{IETF 83.5}} \\
\vspace{3em}
\large{
\begin{tabular}{c}
Eric Rescorla \\
\url{ekr@rtfm.com}
\end{tabular}
}
\end{center}
\end{slide}

\centerslidesfalse


\begin{slide}
\heading{Overview}

\begin{itemize}
\item Communications consent and DoS prevention
  \begin{itemize}
  \item Consent freshness--what is required?
  \item Simulated forking
  \item Traffic rate limiting
  \end{itemize}

\item Identity Protocol Details

\item Resolution of issues raised in reviews (Thomson, Druta)
\end{itemize}

\end{slide}


\begin{slide}
\heading{Communications Consent Overview}

\begin{itemize}
\item Consensus to use ICE for initial consent
  \begin{itemize}
  \item Sufficient for prevention of cross-protocol attack
  \item Not so great protection against packet-based DoS
  \end{itemize}

\item Open issues
  \begin{itemize}
  \item What is required for continuing consent?
  \item Should we limit sender rate?
  \item What about simulated forking?
  \end{itemize}
\end{itemize}
\end{slide}


\begin{slide}
\heading{Consent Freshness}

\begin{itemize}
\item As specified, ICE only checks at start of session
\item Keepalives just keep the NAT binding open
  \begin{itemize}
  \item But aren't confirmed
  \item Or authenticated
  \end{itemize}

\item What if I no longer wish to receive traffic?
  \begin{itemize}
  \item General agreement, some sort of keepalive
  \item Check every $X$ seconds
  \item If I don't receive a keepalive after $Y$ seconds must stop transmitting
    \begin{itemize}
    \item Can re-start ICE if needed
    \end{itemize}
  \end{itemize}
\end{itemize}

\end{slide}



\begin{slide}
\heading{Concrete Proposal: \texttt{draft-muthu-behave-consent-freshness-00}}

\vspace{-.25in}
\begin{itemize}
\item Defines a new ICE method, \verb^Consent^
  \begin{itemize}
  \item Simple request/response
  \item No username/password or message integrity
  \end{itemize}

\item Why not just use STUN binding Request?
  \begin{itemize}
  \item Binding requests require integrity checks
  \end{itemize}
\end{itemize}

\begin{quote}
``One of the reasons for ICE choosing STUN Binding indications for
keepalives is because Binding indication allows integrity to be
disabled, allowing for better performance.  This is useful for large-
scale endpoints, such as PSTN gateways and SBCs as described in
Appendix B section B.10 of the ICE specification.''
\end{quote}

\end{slide}


\begin{slide}

\heading{\texttt{draft-mutha} Consent Algorithm}

{\scriptsize
\begin{algorithmic}[1]
  \Repeat  
     \State $T \gets now()$
     \State $Failures \gets 0$
     \Repeat
     \State $T2 \gets now()$
     \State Start STUN Consent transaction
     \If{Success}
        \State go to 16
     \EndIf
     \State $Failures \gets Failures + 1$
     \If{$Failures == 3$ or $T_M < (now() - T)$}
         \State Stop transmitting; exit
     \EndIf
     \State Wait until $T2 + T_c$
     \Until{False}
     \State Wait till $T + T_c$
  \Until{Call ends.}
\end{algorithmic}
}

\begin{itemize}
\item Proposed values: $T_m = 30, T_c = 15$
\end{itemize}

\end{slide}


\begin{slide}
\heading{Duration of Unwanted Traffic}

\begin{itemize}
\item Assume we have initial consent and then receiver goes offline
  \begin{itemize}
  \item On average next check will be at $T_c/2$ (worst case $T_c$)
  \item Takes around $T_m$ to fail
    \begin{itemize}
    \item Complicated interaction of $T_c$, $T_m$ and ICE RTO
    \item Worst-case is about $2T_m$
    \end{itemize}

\end{itemize}

\item With RFC 5389 parameters, a transaction fails after $39500$ ms
  \begin{itemize}
  \item Expected duration of unwanted traffic is $47$ s
  \item This seems awful long
  \end{itemize}

\item We probably to shorten these parameters
\end{itemize}
\end{slide}


\begin{slide}
\heading{Is a MAC needed?}

\begin{itemize}
\item ICE Binding Requests are authenticated with a MAC
  \begin{itemize}
  \item Based on ufrag and password
  \end{itemize}

\item \verb^Consent^ as currently specified does not have a MAC
  \begin{itemize}
  \item All security is from the 96-bit STUN transaction ID
  \item ... must be pseudorandomly generated
  \item This is plenty of security against an off-path attacker
  \end{itemize}

\item An on-path attacker can simulate consent even if the victim is not responding
  \begin{itemize}
  \item MAC requires attacker to have username and password as well
  \end{itemize}

\item Not clear if there is a concrete attack that requires MAC
\end{itemize}
\end{slide}


\begin{slide}
\heading{DoS via Excessive Traffic}

{\centering
\noindent\includegraphics[width=4in]{excessive-traffic.png}
}
\end{slide}






\begin{slide}
\heading{Why does this work?}

\begin{itemize}
\item ICE only verifies connectivity
  \begin{itemize}
  \item But anything can be sent over that channel
  \end{itemize}


\item SDP parameters are under the control of the attacker
  \begin{itemize}
  \item And that is what controls bit rate
  \end{itemize}
\end{itemize}
\end{slide}


\begin{slide}
\heading{No user consent required (?)}

\begin{itemize}
\item We just need a source of high bandwidth data
  \begin{itemize}
  \item We (probably) can't use a datachannel because it's congestion controlled
  \item And the sequence numbers are unpredictable (allegedly)
  \end{itemize}

\item But media probably is not

\item It's not video that requires user consent
  \begin{itemize}
  \item ... but access to the camera
  \end{itemize}

\item Set up a bogus MediaStream blob that generates continuous patterns
  \begin{itemize}
  \item Use it to source the data
  \item This shouldn't trigger consent dialogs
  \end{itemize}
\end{itemize}
\end{slide}


\begin{slide}
\heading{How serious is this issue?}

\begin{itemize}
  \item Basically another version of voice hammer
  \item Short duration
    \begin{itemize}
    \item If we have consent freshness then $< 1$ minute
    \end{itemize}

  \item But very scalable
    \begin{itemize}
    \item I can mount this using an ad network or any other traffic fishing system
    \item No user consent required
    \item Not self-throttling (unlike HTTP-based attacks)
    \end{itemize}
\end{itemize}
\end{slide}


\begin{slide}
\heading{... But}

\begin{itemize}
\item ICE implementations already need to implement Binding Requests
\end{itemize}

\begin{quote}
``Though Binding Indications are used for keepalives,
an agent MUST be prepared to receive a connectivity check as well.
If a connectivity check is received, a response is generated as
discussed in [RFC5389], but there is no impact on ICE processing
otherwise.'' [RFC 5245; Section 10]
\end{quote}

\begin{itemize}
\item So Binding Requests will work better with non-RTCWEB equipment
\end{itemize}
\end{slide}



\begin{slide}
\heading{A related problem: liveness testing}

\begin{itemize}
\item Applications want to detect connection failure
  \begin{itemize}
  \item This needs to happen on a shorter time scale than consent
  \item How much dead air will people tolerate? ($< 5 seconds$)
  \end{itemize}

\item Proposal: configurable minimal \emph{received} traffic spacing
  \begin{itemize}
  \item If no packet is received in that time, send a Binding Request
  \item Application failure signaled on Binding Request failure
  \end{itemize}
\end{itemize}
\end{slide}


\begin{slide}
\heading{Combined Consent/Liveness Proposal I:\\Shorten STUN timers}

  \begin{itemize}
  \item Proposal: $Rc = 5$, $Rm = 4$. Use measured RTO, minimum $200 ms$
  \item Example: Packets transmitted at $0, 200, 600, 1400, 3000$; transaction fails at $4600ms$
  \item Rationale
    \begin{itemize}
    \item If our RTT is $> 5s$, that's not going to be a very good user experience
    \end{itemize}
  \end{itemize}

\end{slide}


\begin{slide}
\heading{Combined Consent/Liveness Proposal II:\\Both checks done via binding requests}

  \begin{itemize}
  \item Consent timer: $T_{c}$ (default = $20s$, no more than $30s$)
  \item Packet receipt timer: $T_{r}$ (no less than $500 ms$); configurable
  \item When either timer expires start a STUN transaction
  \item When a STUN transaction succeeds, re-start both timers
  \item When a STUN transaction fails
    \begin{itemize}
    \item If transaction was started by $T_c$, stop sending, abort
    \item ... else, notify application of failure, but keep sending
    \end{itemize}

  \item $T_r$ is also reset by receiving any packet from the other side
  \end{itemize}
\end{slide}


\begin{slide}
\heading{What API points do we need?}

\begin{itemize}
\item Ability to set keepalive frequency (individually on each stream?)
\item A consent transaction has failed and so I am not transmitting on stream $M$
\item A liveness check has failed on stream $M$
\end{itemize}
\end{slide}


\begin{slide}
\heading{Proposed API}

\begin{verbatim}
// Do a liveness check every 500ms and call callback if it fails
pc.setLivenessCheck(500, m, function(m) {
   // media stream m has apparently failed
});


//
pc.onstreamfailed = function(m) {
   // media steam consent check has failed
}
\end{verbatim}

\begin{itemize}
\item Would a constraint + event combination be better?
\end{itemize}

\end{slide}


\begin{slide}
\heading{Simulated Forking [Westerlund]}

{\centering
\noindent\includegraphics[width=4in]{simulated-forking.png}
}

\end{slide}


\begin{slide}
\heading{Proposed Solutions}

\begin{itemize}
\item Rate limit the number of outstanding ICE connections you respond to?
  \begin{itemize}
  \item Based on unique ufrag/passwords
  \end{itemize}

\item If using DTLS you can rate limit the number of DTLS associations
  \begin{itemize}
  \item Not clear that this is better than ICE
  \end{itemize}
\end{itemize}
\end{slide}



\begin{slide}
\heading{Reminder What are we trying to accomplish with Identity?}

\begin{itemize}
\item Allow Alice and Bob to have a secure call
  \begin{itemize}
  \item Authenticated with their identity providers
  \item On any site
    \begin{itemize}
    \item Even untrusted/partially trusted ones
    \end{itemize}
  \end{itemize}

\item Advantages
  \begin{itemize}
  \item Use one identity on any calling site
  \item Security against active attack by calling site
  \item Support for federated cases
  \end{itemize}
\end{itemize}

\end{slide}

\begin{slide}
\heading{Reminder: Overall Topology}
\vspace{-.4in}

{\center
\hspace{.75in} \includegraphics[width=2.8in]{idp-arch}
}

\end{slide}

\begin{slide}
\heading{What needs to be defined}

\begin{itemize}
\item Information from the signaling message that is authenticated [IETF]
  \begin{itemize}
  \item Minimally: DTLS-SRTP fingerprint
  \item Generic carrier for identity assertion
  \item Depends on signaling protocol
  \end{itemize}

\item Interface from PeerConnection to the IdP [IETF]
  \begin{itemize}
  \item A specific set of messages to exchange
  \item Sent via \verb^postMessage()^ or WebIntents
  \end{itemize}

\item JavaScript calling interfaces to PeerConnection [W3C]
  \begin{itemize}
  \item Specify the IdP
  \item Interrogate the connection identity information
  \end{itemize}
\end{itemize}
\end{slide}


\begin{slide}
\heading{Data to be authenticated}

\begin{itemize}
\item JSON dictionary with one value: fingerprint
\end{itemize}

\begin{verbatim}
       {
          "fingerprint": 
            {
              "algorithm":"SHA-1",
              "digest":"4A:AD:B9:B1:3F:...:E5:7C:AB"
            }
       }
\end{verbatim}

\begin{itemize}
\item Note: this format is trivially the same as the a-line format
\end{itemize}
\end{slide}



\begin{slide}
\heading{Wire format for identity assertions}

\begin{small}
\begin{verbatim}
       "identity":{
            "idp":{     // Standardized
               "domain":"idp.example.org", // Identity domain
               "method":"default"          // Domain-specific method
            },
            "assertion": "..."             // IdP-specific
       }
\end{verbatim}
\end{small}
\end{slide}


\begin{slide}
\heading{Open issue: how is identity assertion carried?}

\begin{itemize}
\item Delivered separately to application
  \begin{itemize}
  \item Requires application to manage the data
  \end{itemize}

\item Along with SDP in \verb^createOffer()/createAnswer()^
  \begin{itemize}
  \item This only works with the ``dictionary'' form
  \item And doesn't guarantee fate-sharing with SDP
  \end{itemize}

\item Best option: put it in an a-line
  \begin{itemize}
  \item Fate-shares with SDP
  \item Can tag to individual a-lines if necessary
  \item Potentially SIP compatible (though not with existing endpoints)
  \end{itemize}
\end{itemize}
\end{slide}


\begin{slide}
\heading{Concrete Proposal: Opaque Value}

\begin{itemize}
\item Just base-64 the data and shove it in an a-line
  \begin{itemize}
  \item e.g., ``identity''
  \item Like ICE candidates can apply to entire SDP or individual m-lines
  \end{itemize}
\end{itemize}

\begin{verbatim}
a = identity: <base-64ed of JSON>
\end{verbatim}

\begin{itemize}
\item Alternative: actually render the identity information
  \begin{itemize}
  \item But we need to potentially escape stuff anyway, so encapsulated is easier
  \end{itemize}
\end{itemize}
\end{slide}



\begin{slide}
\heading{A concrete example}

{\footnotesize
\begin{verbatim}
   v=0
   o=- 1181923068 1181923196 IN IP4 ua1.example.com
   s=example1
   c=IN IP4 ua1.example.com
   a=setup:actpass
   a=fingerprint: SHA-1 \
     4A:AD:B9:B1:3F:82:18:3B:54:02:12:DF:3E:5D:49:6B:19:E5:7C:AB
   a=identity: 6XCJib2JAZXhhbXBsZS5vcmdcIiwKICAgICAgICAgICAgICAg\
               ICAgICAgICAgXCJjb250ZW50c1wiOlwiYWJjZGVmZ2hpamtsb\
               W5vcHFyc3R1dnd5
   t=0 0
   m=audio 6056 RTP/AVP 0
   a=sendrecv
   a=tcap:1 UDP/TLS/RTP/SAVP RTP/AVP
   a=pcfg:1 t=1
\end{verbatim}
}
\end{slide}


\begin{slide}
\heading{Generic Downward Interface\\(Implemented by PeerConnection)}

\begin{itemize}
\item Instantiate ``IdP Proxy'' with JS from IdP
  \begin{itemize}
  \item Probably invisible IFRAME
  \item Maybe a WebIntent
  \item[] {\small \url{https://<idp-domain>/.well-known/idp-proxy/<protocol>}}
  \end{itemize}

\item Send (standardized) messages to IdP proxy via \verb^postMessage()^
  \begin{itemize}
  \item ``\verb^SIGN^'' to get assertion
  \item ``\verb^VERIFY^'' to verify assertion
  \end{itemize}

\item IdP proxy responds 
  \begin{itemize}
  \item ``\verb^SUCCESS^'' with answer
  \item ``\verb^ERROR^'' with error
  \end{itemize}
\end{itemize}
\end{slide}


\begin{slide}
\heading{Signature Messages}

\begin{tiny}
\begin{verbatim}
       PeerConnection -> IdP proxy:
         {
           "type":"SIGN",
            "id":1,
            "contents":
              "{\"fingerprint\":{\"algorithm\":\"SHA-1\",
              \"digest\":\"4A:AD:B9:B1:3F:...:E5:7C:AB\"}}"
         }

       IdPProxy -> PeerConnection:
         {
           "type":"SUCCESS",
           "id":1,
           "message": {
             "idp":{
               "domain": "example.org"
               "protocol": "bogus"
             },
             "assertion": "..." // opaque
           }
         }
\end{verbatim}
\end{tiny}
\end{slide}


\begin{slide}
\heading{Verification Process}

\begin{tiny}
\begin{verbatim}
         PeerConnection -> IdP Proxy:
           {
             "type":"VERIFY",
             "id":2,
             "message": "..." // opaque
           }

         IdP Proxy -> PeerConnection:
           {
            "type":"SUCCESS",
            "id":2,
            "message": {
              "identity" : {
                "name" : "bob@example.org",
                "displayname" : "Bob"
              },
              "contents":
                "{\"fingerprint\":{\"algorithm\":\"SHA-1\",
                \"digest\":\"4A:AD:B9:B1:3F:...:E5:7C:AB\"}}"
              }
           }
\end{verbatim}
\end{tiny}
\end{slide}

\begin{slide}
\heading{Meaning of Successful Verification}

\begin{itemize}
\item IdP has verified assertion
  \begin{itemize}
  \item Identity is given in ``identity''
  \item ``name'' is the actual identity (RFC822 format) [OPEN ISSUE]
  \item ``displayname'' is a human-readable string
  \end{itemize}

\item Contents is the original message the AP passed in
\end{itemize}

\end{slide}


\begin{slide}
\heading{Processing Successful Verifications}

\begin{itemize}
\item Authoritative IdPs
  \begin{itemize}
  \item RHS of \verb^identity.name^ matches IdP domain
  \item No more checks needed
  \end{itemize}

\item Third-party IdPs
  \begin{itemize}
  \item RHS of \verb^identity.name^ does not match IdP domain
  \item IdP MUST be trusted by policy
  \end{itemize}

\item These checks performed by PeerConnection
\end{itemize}
\end{slide}


\begin{slide}
\heading{New API points needed}

\begin{itemize}
\item Set the desired IdP and identity
\item Pre-generate an assertion (performance)
\item Interrogate a validated assertion
\end{itemize}

\end{slide}



\begin{slide}
\heading{Set desired IdP and identity}

\begin{verbatim}
void SetIdentityProvider (
                          DOMString Provider,
                          optional DOMString protocol,
                          optional DOMSTring username
                         );
\end{verbatim}

\vspace{1ex}

\noindent\begin{itemize}
\item[] \verb^provider^ -- domain name of the provider
\item[] \verb^protocol^ -- the protocol name (locally meaningful)
\item[] \verb^username^ -- the desired user name
\end{itemize}

\vspace{1ex}
\begin{itemize}
\item OPEN ISSUE: relationship to user settings (who wins)?
\end{itemize}
\end{slide}


\begin{slide}
\heading{Pre-Generate Assertions}

\begin{verbatim}
void GenerateIdentityAssertion (
                                AssertionSuccessCallback success,
                                PeerConnectionErrorCallback error
                               );
\end{verbatim}

\vspace{1ex}

\begin{itemize}
\item Idea here is to generate identity assertion ahead of time
  \begin{itemize}
  \item Same reason as we want to start ICE gathering early
  \end{itemize}

\item This works because we only need DTLS fingerprint at this time
\item What should be passed to the callback? anything?
\end{itemize}
\end{slide}


\begin{slide}
\heading{Interrogate a valid assertion}

\begin{verbatim}
readonly attribute IdentityAssertion peeridentity;

interface IdentityAssertion {
  attribute DOMString name; // The peer identity (rfc822-style)
  attribute DOMString displayname; // Human-readable
};
\end{verbatim}

\vspace{1ex}

\begin{itemize}
\item If identity was not used, this attribute is null
\item OPEN ISSUE: Do we need an event for identity verification?
\end{itemize}
\end{slide}



\begin{slide}
\heading{Recap of reviews}

\begin{itemize}
\item Reviews from Martin Thomson, Dan Druta
\item Latest drafts attempt to address their issues
\item I probably missed stuff
\end{itemize}
\end{slide}


\begin{slide}
\heading{HTTPS versus HTTP Threat Model}


{\scriptsize
\begin{quote}
``Obviously, the standard class of problems with unsecured HTTP exist, but
within the context of this application, there aren't that many more that
this enables.  The example in S4.1.3 is not unique to this application.
It applies to any user consent that is tied to a particular web origin.

In comparison to possibly visiting and \_using\_ a site operated by a web
attacker, this is not substantially worse, or requiring significantly
more effort to analyze.

Of course, the only safe assumption is that you are talking to a web
attacker when using unsecured HTTP.'' -- Thomson
\end{quote}

New text:
\begin{quote}
Conventionally, we refer to either WEB ATTACKERS, who are able to
induce you to visit their sites but do not control the network, and
NETWORK ATTACKERS, who are able to control your network.  Network
attackers correspond to the [RFC3552] "Internet Threat Model".  Note
that for HTTP traffic, a network attacker is also a Web attacker,
since it can inject traffic as if it were any non-HTTPS Web site.
Thus, when analyzing HTTP connections, we must assume that traffic is
going to the attacker.
\end{quote}
}
\end{slide}


\begin{slide}
\heading{Consent}

{\scriptsize
\begin{quote}
``I think that it has been well-established that consent is required for
access to input devices (e.g., camera/microphone).  The implication from
S4.1 is that this is sufficient as well as necessary.  There is one
crucial piece of the argument that is absent:

\begin{quote}
A site with access to camera or microphone could send media either to
itself or any site that indicates consent (see CORS).  Sending media
over HTTP or thewebsocketprotocol is likely to perform less well than
is ideal, but it would work.'' -- Thomson
\end{quote}
\end{quote}

Some new text here:

\begin{quote}
It's important to understand that consent to access local devices is
largely orthogonal to consent to transmit various kinds of data over
the network (see Section 4.2.  Consent for device access is largely a
matter of protecting the user's privacy from malicious sites.  By
contrast, consent to send network traffic is about preventing the
user's browser from being used to attack its local network.  Thus, we
need to ensure communications consent even if the site is not able to
access the camera and microphone at all (hence WebSockets's consent
mechanism) and similarly we need to be concerned with the site
accessing the user's camera and microphone even if the data is to be
sent back to the site via conventional HTTP-based network mechanisms
such as HTTP POST.
\end{quote}
}
\end{slide}

\begin{slide}
\heading{Calls from non-same-Origin IFRAMEs}

{\footnotesize
\begin{quote}
``I've seen a range of responses from sites.  Some sites vet advertisers
carefully, others could care less as long as the money flows.  Those
that vet have usually been stung once already.

If reputation is important to you, then it is your responsibility to
safeguard your own reputation.  If you rely on others, then you can
use technical measures (checking, etc...), or simply rely on their own
desire to safeguard their reputation.

In summary, I don't see a need for a specific technical solution to
this problem.''
\end{quote}
}

\begin{itemize}
\item Should non-same-origin IFRAMEs notify the enclosing frame
\item Proposed resolution: remove the text
\end{itemize}
\end{slide}



\begin{slide}
\heading{IP Address Hiding (relay)}

\begin{itemize}
\item Current requirements
  \begin{itemize}
  \item API to suppress negotiation until call is answered (can be done with SDP editing)
  \item API to suppress non-relay candidates (now done via constraints)
  \end{itemize}

{\footnotesize
  \begin{quote}
``Section 5.4 stipulates requirements that I don't think are reasonable.
 Including discussion on the subject, including countermeasures, with
a conclusion that there are no requirements on the API would be good.
However, guidance for site implementers would be sensible.'' -- Thomson
\end{quote}
}
\end{itemize}

\begin{itemize}
\item Note sure how to proceed here
\end{itemize}
\end{slide}



\begin{slide}
\heading{Persistent Consent and HTTP}

{\footnotesize
  \begin{quote}
``You already established that - in the presence of a network attacker -
consent to foo.com is equivalent to consent to bar.net.  Therefore, it
seems reasonable to regard the two as being equivalent.  Once you make
that leap, then it is easy to see that you don't have enough
granularity to make any consent meaningful.  So you have to conclude
that providing a (persistent) consent for non-HTTPS sites is
pointless.'' -- Thomson
\end{quote}
}

\begin{itemize}
\item I don't agree they are equivalent
  \begin{itemize}
  \item Web attackers do exist
  \end{itemize}

\item Do we want to forbid persistent consent for HTTP?
\end{itemize}
\end{slide}



\begin{slide}
\heading{Identity and Linkability}

{\footnotesize
  \begin{quote}
``What about identity?  While you are in the business of creating an
identity system, wouldn't it be nice if the site you are using
couldn't identify the people using that site?  Imagine that you are
able to create an assertion that you are
dhouhqed08gslkn209eejit8sfsdo@rtfm.com (that well known IdP), which is
translated (by the IdP validator) to a ekr@rtfm.com only in the
browser chrome...''
\end{quote}}
                        
\begin{itemize}
\item This is compatible with the proposed identity system
\item Though requires that the IdP validator verify the RP's identity
\item I'll add some text
\end{itemize}
\end{slide}


\begin{slide}
\heading{Access to Local Devices}

\vspace{-.25in}
{\footnotesize
\begin{quote}
``Since this describes the threat I think is important to point out
the fact that even when the user provides consent it is difficult
for them to determine the real reason they give the consent to a
site. Take the scenario in which a site obtains consent from the
user for an app that captures a clip and saves it on the user's hard
drive. Later on, if the user gave permanent consent to the site, the
site can obtain access to the camera for the purpose of streaming
without the user knowing that. This can be confusing for users even
if they trust the site.'' -- Druta
\end{quote}
}

\begin{itemize}
\item General limitation of technical consent mechanisms
  \begin{itemize}
  \item Once I let you take my media and send it somewhere it's
    hard to constrain what you do with it
   \end{itemize}

 \item Same thing is generally true with, e.g., image upload, Facebook wall
 \item This is what privacy policies are for
\end{itemize}

\end{slide}

\end{document}
