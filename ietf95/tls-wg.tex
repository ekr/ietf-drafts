\documentclass[helvetica]{seminar} 
\input{xy}
\xyoption{all}
\usepackage{graphicx} 
\usepackage{slidesec} 
\usepackage{url}
\usepackage[framemethod=TikZ]{mdframed}
\usepackage{color}

\long\def\symbolfootnote[#1]#2{\begingroup%
\def\thefootnote{\fnsymbol{footnote}}\footnote[#1]{#2}\endgroup}

% to fix problems making landscape seminar pdfs
% Letter...
\pdfpagewidth=11truein
\pdfpageheight=8.5truein
\pdfhorigin=1truein     % default value(?), but doesn't work without
\pdfvorigin=1truein     % default value(?), but doesn't work without
% A4
%\pdfpagewidth=297truemm % your milage may vary....
%\pdfpageheight=210truemm
%\pdfhorigin=1truein     % default value(?), but doesn't work without
%\pdfvorigin=1truein     % default value(?), but doesn't work without


\renewcommand{\familydefault}{\sfdefault}  
 
\input{seminar.bug} 
\input{seminar.bg2} % See the Seminar bugs list 
 
\slideframe{none} 
 
 
\usepackage{fancyhdr} 
 
% Headers and footers personalization using the `fancyhdr' package 
\fancyhf{} % Clear all fields 
\renewcommand{\headrulewidth}{0mm} 
\renewcommand{\footrulewidth}{0.1mm} 
 
\fancyfoot[L]{\tiny IETF 95} 
\fancyfoot[C]{\tiny TLS}
\fancyfoot[R]{\tiny \theslide} 
 
 
% To center horizontally the headers and footers (see seminar.bug) 
\renewcommand{\headwidth}{\textwidth} 

% To adjust the frame length to the header and footer ones 
\autoslidemarginstrue 
\pagestyle{fancy} 
 

\newcommand{\heading}[1]{% 
  \begin{center} 
    \large\bf 
    #1 
  \end{center} 
  \vspace{.4 in}} 



\begin{document}

\begin{slide}
\begin{center}
\vspace{.5 in}
\LARGE{{\bf}TLS 1.3\\{\small \verb^draft-ietf-tls-tls13-12^}}\\
\vspace{.2in}
\large{
\begin{tabular}{c}
Eric Rescorla\\
Mozilla\\
\url{ekr@rtfm.com}
\end{tabular}
}
\end{center}

\end{slide}

\centerslidesfalse 


\begin{slide}
\heading{Overview}

\begin{itemize}
\item Changes since draft-10
\item Outstanding consensus calls
\item 1-RTT PSK and session tickets
\item Context values
\item Key schedule and key separation
\item 0-RTT details
\item Minor issues
\end{itemize}
\end{slide}

\begin{slide}
\heading{Changes since draft-10}

\begin{itemize}
\item Restructure authentication along uniform lines
\item Restructure 0-RTT content types
\item Reset sequence numbers on key changes
\item Import CFRG Curves
\item Zero-length additional data for AEAD
\item Revised signature algorithm negotiation
\item Define exporters
\item Add anti-downgrade mechanism
\item Add PSK cipher suites
\item Other editorial
\end{itemize}
\end{slide}

\begin{slide}
\heading{Restructuring Authentication}

\begin{itemize}
\item TLS 1.3 has four authentication contexts
  \begin{itemize}
  \item 1-RTT server
  \item 1-RTT client
  \item 0-RTT client\symbolfootnote[2]{Marked for death.}
  \item Post-handshake
  \end{itemize}

\item All were slightly different
\item draft-12 unifies them into one common idiom
\end{itemize}
\end{slide}


\begin{slide}
\heading{TLS 1.3 Authentication Block}

\begin{itemize}
\item Three messages: Certificate*, CertificateVerify*, Finished
\item Inputs
  \begin{itemize}
  \item Handshake Context (generally the handshake hash)
  \item Certificate/signing key
  \item Base key for MAC key
  \end{itemize}

\item CertificateVerify = {\small \verb^digitally-sign(Hash(Handshake Context + Certificate))^\symbolfootnote[1]{Includes disambiguating context string.}}
\item Finished = \\
  {\small \verb^HMAC(finished_key, Handshake Context + Certificate + CertificateVerify)^}
\item Different finished keys for each direction (based on Base Key)
\end{itemize}
\end{slide}


\begin{slide}
\heading{Eye Chart}

{\scriptsize
\begin{verbatim}
   +----------------+-----------------------------------------+--------+
   | Mode           | Handshake Context                       | Base   |
   |                |                                         | Key    |
   +----------------+-----------------------------------------+--------+
   | 0-RTT          | ClientHello + ServerConfiguration +     | xSS    |
   |                | Server Certificate + CertificateRequest |        |
   |                | (where ServerConfiguration, etc. are    |        |
   |                | from the previous handshake)            |        |
   |                |                                         |        |
   | 1-RTT (Server) | ClientHello ... ServerConfiguration     | master |
   |                |                                         | secret |
   |                |                                         |        |
   | 1-RTT (Client) | ClientHello ... ServerFinished          | master |
   |                |                                         | secret |
   |                |                                         |        |
   | Post-Handshake | ClientHello ... ClientFinished +        | master |
   |                | CertificateRequest                      | secret |
   +----------------+-----------------------------------------+--------+
\end{verbatim}
}
\end{slide}


\begin{slide}
\heading{Restructure 0-RTT Record Structure}

\begin{itemize}
\item draft-10 had a somewhat idiosyncratic design
\item draft-12 0-RTT parallels 1-RTT
  \begin{itemize}
  \item \verb^handshake^ for handshake data
  \item \verb^application_data^ for application data
  \item New \verb^end_of_early_data^ (warning) alert for separation
  \item Separate handshake and traffic keys
  \end{itemize}
\end{itemize}

\end{slide}


\end{document}  
                