\documentclass[helvetica]{seminar} 
\input{xy}
\xyoption{all}
\usepackage{graphicx} 
\usepackage{slidesec} 
\usepackage{url}
\usepackage[framemethod=TikZ]{mdframed}

\long\def\symbolfootnote[#1]#2{\begingroup%
\def\thefootnote{\fnsymbol{footnote}}\footnote[#1]{#2}\endgroup}

% to fix problems making landscape seminar pdfs
% Letter...
\pdfpagewidth=11truein
\pdfpageheight=8.5truein
\pdfhorigin=1truein     % default value(?), but doesn't work without
\pdfvorigin=1truein     % default value(?), but doesn't work without
% A4
%\pdfpagewidth=297truemm % your milage may vary....
%\pdfpageheight=210truemm
%\pdfhorigin=1truein     % default value(?), but doesn't work without
%\pdfvorigin=1truein     % default value(?), but doesn't work without



\renewcommand{\familydefault}{\sfdefault}  
 
\input{seminar.bug} 
\input{seminar.bg2} % See the Seminar bugs list 
 
\slideframe{none} 
 
 
\usepackage{fancyhdr} 
 
% Headers and footers personalization using the `fancyhdr' package 
\fancyhf{} % Clear all fields 
\renewcommand{\headrulewidth}{0mm} 
\renewcommand{\footrulewidth}{0.1mm} 
 
\fancyfoot[L]{\tiny IETF 85} 
\fancyfoot[C]{\tiny Random CNAMEs}
\fancyfoot[R]{\tiny \theslide} 
 
 
% To center horizontally the headers and footers (see seminar.bug) 
\renewcommand{\headwidth}{\textwidth} 

% To adjust the frame length to the header and footer ones 
\autoslidemarginstrue 
\pagestyle{fancy} 
 

\newcommand{\heading}[1]{% 
  \begin{center} 
    \large\bf 
    #1 
  \end{center} 
  \vspace{.4 in}} 

\begin{document}
\begin{slide}
\begin{center}
\vspace{1 in}
\LARGE{{\bf}Rationally reconstructing Plan A}\\
\vspace{.2in}
\vspace{3em}
\large{
\begin{tabular}{c}
Eric Rescorla\\
\end{tabular}
}
\end{center}

\end{slide}

\centerslidesfalse 


\begin{slide}
\heading{Desirable Properties (\texttt{draft-jennings-mmusic-media-req-00})}

\begin{itemize}
\item Push as many media flows as possible over one transport 5-tuple
\item Negotiate with both new and old endpoints without multiple O/A exchanges
  \begin{itemize}
  \item Or at least get media flowing with one exchange
  \end{itemize}
\item Be able to independently negotiate media parameters for each flow
  \begin{itemize}
  \item Even when multiplexed
  \end{itemize}
\item Allow for a very large (>> 100) number of media flows
\item Add new tracks without worrying about glare
\item Somehow directly reference individual WebRTC tracks
\end{itemize}
\end{slide}



\begin{slide}
\heading{Bundle solves a lot of this stuff}

\begin{itemize}
\item Push as many media flows as possible over one transport 5-tuple
\item Negotiate with both new and old endpoints without multiple O/A exchanges
  \begin{itemize}
  \item Or at least get media flowing with one exchange
  \end{itemize}
\item Be able to independently negotiate media parameters for each flow
  \begin{itemize}
  \item Even when multiplexed
\end{itemize}
\item[]
\item But....
  \begin{itemize}
  \item How what about large numbers of flows?
  \item 
  \end{itemize}
\end{itemize}
\end{slide}



\end{document}
