\documentclass[helvetica]{seminar} 
\input{xy}
\xyoption{all}
\usepackage{graphicx} 
\usepackage{slidesec} 
\usepackage{url}
\usepackage[normalem]{ulem}
\usepackage[usenames]{color}

\long\def\symbolfootnote[#1]#2{\begingroup%
\def\thefootnote{\fnsymbol{footnote}}\footnote[#1]{#2}\endgroup}

% to fix problems making landscape seminar pdfs
% Letter...
\pdfpagewidth=11truein
\pdfpageheight=8.5truein
\pdfhorigin=1truein     % default value(?), but doesn't work without
\pdfvorigin=1truein     % default value(?), but doesn't work without
% A4
%\pdfpagewidth=297truemm % your milage may vary....
%\pdfpageheight=210truemm
%\pdfhorigin=1truein     % default value(?), but doesn't work without
%\pdfvorigin=1truein     % default value(?), but doesn't work without



\renewcommand{\familydefault}{\sfdefault}  
 
\input{seminar.bug} 
\input{seminar.bg2} % See the Seminar bugs list 
 
\slideframe{none} 
 
 
\usepackage{fancyhdr} 
 
% Headers and footers personalization using the `fancyhdr' package 
\fancyhf{} % Clear all fields 
\renewcommand{\headrulewidth}{0mm} 
\renewcommand{\footrulewidth}{0.1mm} 
 
\fancyfoot[L]{\tiny WebRTC Interim; May 2014} 
\fancyfoot[C]{\tiny Error Reporting}
\fancyfoot[R]{\tiny \theslide} 
 
 
% To center horizontally the headers and footers (see seminar.bug) 
\renewcommand{\headwidth}{\textwidth} 

% To adjust the frame length to the header and footer ones 
\autoslidemarginstrue 
\pagestyle{fancy} 
 

\newcommand{\heading}[1]{% 
  \begin{center} 
    \large\bf 
    #1 
  \end{center} 
  \vspace{.4 in}} 

\begin{document}
\begin{slide}
\begin{center}
\vspace{1 in}
\LARGE{{\bf}WebRTC Error Reporting}\\
\vspace{.2in}
\large{{Interim Meeting: May 2014}} \\
\vspace{3em}
\large{
\begin{tabular}{c}
Eric Rescorla \\
\emph{Mozilla} \\
\url{ekr@rtfm.com}
\end{tabular}
}
\end{center}

\end{slide}


\centerslidesfalse

\begin{slide}
\heading{General Principles}

\begin{itemize}
\item Error types
  \begin{itemize}
  \item Programming errors get exceptions
  \item Other errors get callbacks
  \end{itemize}

\item WebIDL does type checking
\item Try not to fail silently
\end{itemize}
\end{slide}


\begin{slide}
\heading{Calls after \texttt{pc.close()}}

\begin{itemize}
\item Some methods generate \verb^InvalidStateError^ (e.g., \verb^addStream()^) and some don't (e.g., \verb^createAnswer()^)
\item This is confusing for programmers
\item Proposed behavior
  \begin{itemize}
  \item Any call other than \verb^close()^ generates \verb^InvalidStateError^
  \item Asynchronous calls followed by \verb^close()^ simply never happen
    \begin{itemize}
    \item Some of the descriptions have them happen but then not generate callbacks
    \end{itemize}
  \item \verb^close()^ fails silently
  \end{itemize}

\item This should all be written in one place  
\end{itemize}
\end{slide}


\begin{slide}
\heading{Multiple calls to \texttt{addStream()} with same stream}

\begin{itemize}
\item Currently this is ignored
\item Proposed behavor
  \begin{itemize}
  \item Throw \verb^ResourceInUse^
  \end{itemize}
\end{itemize}
\end{slide}

\begin{slide}
\heading{Trying to send on a closed DataChannel (\S\  5.2.)}

\begin{itemize}
\item Currently behavior
  \begin{itemize}
  \item Increment \verb^bufferedAmount^ (step 3)
  \item Then silently abort
  \end{itemize}


\item Proposed behavior
  \begin{itemize}
  \item Throw \verb^InvalidStateError^
  \end{itemize}
\end{itemize}
\end{slide}


\begin{slide}
\heading{Create RTCDTMFSender on non-audio track}

\begin{itemize}
\item Unclear what current behavior is
  \begin{itemize}
  \item Probably create it with \verb^.canInsertDTMF === false^
  \end{itemize}

\item Proposed behavior
  \begin{itemize}
  \item Generate an \verb^InvalidParameter^ exception
  \end{itemize}
  
\end{itemize}
\end{slide}


\begin{slide}
\heading{Call \texttt{insertDTMF()} when \texttt{.canInsertDTMF} is false}

\begin{itemize}
\item Current behavior
  \begin{itemize}
  \item Silently ignore
  \end{itemize}

\item Proposed behavior
  \begin{itemize}
  \item Generate \verb^InvalidStateError^
  \end{itemize}
\end{itemize}
\end{slide}


\begin{slide}
\heading{Insert DTMF with bogus values (Guduru)}

\begin{itemize}
\item Generate \verb^InvalidParameter^ exception
\end{itemize}

\end{slide}

\begin{slide}
\heading{Unused errors (\S\  4.6.2.1)}

\begin{itemize}
\item \verb^IncompatibleConstraintsError^
\item \verb^incompatibleMediaStreamTrackError^
\item
\item Propose removing these
\end{itemize}
\end{slide}


\begin{slide}
\heading{\texttt{RTCSdpError} (\S\ 4.6.2}

\begin{itemize}
\item This is defined but never used
  \begin{itemize}
  \item Even though it's semi-referenced
  \end{itemize}

\item Proposed resolution
  \begin{itemize}
  \item Edit \verb^set{Local,Remote}Description^ sections to make clear it is passed to error callbacks
  \end{itemize}
\end{itemize}
\end{slide}


\begin{slide}
\heading{\texttt{addIceCandidate()} in wrong state}

\begin{itemize}
\item Claim 1: this should be queued
\item Claim 2: Any attempt to \verb^addIceCandidate()^ in wrong state should get \verb^InvalidSteError^ (just like everything else)
\end{itemize}
\end{slide}


\begin{slide}
\heading{\texttt{addIceCandidate()} bogus data}

\vspace{-3ex}
\small{
\begin{quote}
``If the candidate could not be successfully applied, the user agent
must queue a task to invoke failureCallback with a DOMError object
whose name attribute has the value TBD (TODO InvalidCandidate and
InvalidMidIndex).''

NOTE What errors do we need here? Should we reuse the
*SessionDescriptionError names or invent new ones for
candidates? Should this method be queued?
\end{quote}
}

\begin{itemize}
\item The session description errors just say ``it's busted'' and give line number
  \begin{itemize}
  \item Assume we don't want to change that
  \item And even if they did, it would just say ``ICE candidate is busted''
  \end{itemize}

\item Proposed resolution
  \begin{itemize}
  \item \verb^InvalidCandidate^, \verb^InvalidMid^ and \verb^InvalidMLineIndex^ with obvious meanings
  \end{itemize}
\end{itemize}
\end{slide}


\begin{slide}
\heading{failure callbacks for \texttt{createOffer()} and \texttt{createAnswer()}}

\begin{quote}
If the SDP generation process failed for any reason, the user agent
must queue a task to invoke failureCallback with an DOMError object of
type TBD as its argument.
\end{quote}

\begin{itemize}
\item Can anything go wrong here?
\item ... other than \verb^InternalError^?
\item Can pretty much be called in any state
\end{itemize}
\end{slide}


\begin{slide}
\heading{failure callbacks for \texttt{setLocalDescription()} and \texttt{setRemoteDescription()}}

\vspace{-4ex}

\begin{itemize}
\item Currently the following errors
  \begin{itemize}
  \item \verb^InvalidSessionDescriptionError^ --- SDP is bogus; need to specify line number
  \item \verb^IncompatibleSessionDescriptionError^ --- can't apply SDP; freeform text to explain the problem?
  \end{itemize}

\item Also add
\begin{itemize}
\item \verb^InvalidStateError^ --- e.g., \verb^setLocalDescription(answer)^ 
\item \verb^InvalidPeerIdentityError^ --- e.g., if the target identity doesn't match the remote description
\item What should we use for bogus changes to SDP? \verb^InvalidSessionDescriptionError^?
\end{itemize}

\end{itemize}
\end{slide}

\begin{slide}
\heading{Spontaneous Errors}

\begin{itemize}
\item Some kinds of errors just happen
  \begin{itemize}
  \item TURN errors
  \item DTLS connection error
  \item ``Internal errors''
  \end{itemize}

\item These aren't tied to any API call
\end{itemize}
\end{slide}


\begin{slide}
\heading{\texttt{.onerror} or something}

\begin{itemize}
\item Throws some typed object
\end{itemize}

\begin{verbatim}
interface RTCRuntimeError : DOMError {
    readonly    attribute explanation;
};
\end{verbatim}

\begin{itemize}
\item The \verb^.name^ property should tell us what happened (machine readable)
\item The \verb^.explanation^ should be freeform
\item Individual errors could have their own attributes
  \begin{itemize}
  \item E.g., TURN server URL for TURN errors
  \item Idea is that it should be processable by app
  \end{itemize}
\end{itemize}
\end{slide}


\begin{slide}
\heading{What about state?}

\begin{itemize}
\item How do I distinguish fatal from nonfatal
\item State needs to change \emph{first}
\item So that the event handler can interrogate it
\end{itemize}

\end{slide}


\begin{slide}
\heading{Are there enough of these?}

\begin{itemize}
\item TURN errors
\item DTLS connection error
\item ``Internal errors''
\item
\item Are there really enough of these to design a general facility?
\end{itemize}

\end{slide}

\end{document}


