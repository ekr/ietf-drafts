\documentclass[helvetica]{seminar} 
\input{xy}
\xyoption{all}
\usepackage{graphicx} 
\usepackage{slidesec} 
\usepackage{url}

\long\def\symbolfootnote[#1]#2{\begingroup%
\def\thefootnote{\fnsymbol{footnote}}\footnote[#1]{#2}\endgroup}

% to fix problems making landscape seminar pdfs
% Letter...
\pdfpagewidth=11truein
\pdfpageheight=8.5truein
\pdfhorigin=1truein     % default value(?), but doesn't work without
\pdfvorigin=1truein     % default value(?), but doesn't work without
% A4
%\pdfpagewidth=297truemm % your milage may vary....
%\pdfpageheight=210truemm
%\pdfhorigin=1truein     % default value(?), but doesn't work without
%\pdfvorigin=1truein     % default value(?), but doesn't work without



\renewcommand{\familydefault}{\sfdefault}  
 
\input{seminar.bug} 
\input{seminar.bg2} % See the Seminar bugs list 
 
\slideframe{none} 
 
 
\usepackage{fancyhdr} 
 
% Headers and footers personalization using the `fancyhdr' package 
\fancyhf{} % Clear all fields 
\renewcommand{\headrulewidth}{0mm} 
\renewcommand{\footrulewidth}{0.1mm} 
 
\fancyfoot[L]{\tiny IETF 80} 
\fancyfoot[C]{\tiny P2PSIP}
\fancyfoot[R]{\tiny \theslide} 
 
 
% To center horizontally the headers and footers (see seminar.bug) 
\renewcommand{\headwidth}{\textwidth} 

% To adjust the frame length to the header and footer ones 
\autoslidemarginstrue 
\pagestyle{fancy} 
 

\newcommand{\heading}[1]{% 
  \begin{center} 
    \large\bf 
    #1 
  \end{center} 
  \vspace{.4 in}} 

\begin{document}
\begin{slide}
\begin{center}
\vspace{-.5in}
\LARGE{{\bf}RELOAD Base Update}
\large{\url{draft-ietf-p2psip-base-13}}\\
\large{{IETF 80}} \\
\vspace{.2in}
\large{
Rev. Dr. Cullen Jennings, Ph.D. \\
Distinguished Engineer\\
Office of the CTO, Cisco VTG\\
UBC Young Alumnus of the Year 2005\\
\url{fluffy@cisco.com}\\
\vspace{1em}
NASDAQ: CSCO
}
\end{center}

\end{slide}


\centerslidesfalse 





\begin{slide}
\heading{DRR}

\vspace{-.4in}
\begin{itemize}
\item Specified in a separate draft
  \begin{itemize}
  \item Need minimal support in base draft for forward compatibility
  \end{itemize}

\item Proposal
  \begin{itemize}
  \item Exception in \S\ 4.1.2 to allow extensions to adjust state keeping rules:
    \small{
    \begin{quote}
      Whatever algorithm is used, unless a FORWARD\_CRITICAL forwarding
      option or overlay configuration option explicitly indicates this
      state is not needed, the state MUST be maintained for at least the
      value of the overlay reliability timer (3 seconds) and MAY be kept
      longer.  Future extension, such as [I-D.jiang-p2psip-relay], may
      define mechanisms for determining when this state does not need to be
      retained.
 \end{quote}
}
  \item \verb^IGNORE-STATE-KEEPING^ forwarding extension in \verb^draft-jiang-p2psip-relay-05^ to use via list
  \item Agreement from Jennings, Lowekamp, Even, Rescorla
  \end{itemize}
\end{itemize}
                

\end{slide}



\begin{slide}
\heading{Configuration Refresh}

\begin{itemize}
\item Previously, just expiration time
\item Now: retrieve configuration at a randomly selected time in the future
\end{itemize}
\end{slide}



\begin{slide}
\heading{Configuration Defaults}

\begin{itemize}
\item Many settings in which there were no defaults now have them
  \begin{itemize}
  \item Topology plugin (\verb^CHORD_RELOAD^)
  \item Port (\verb^6084^)
  \item Clients-permitted (\verb^chord-reactive = true^)
   \item ...
  \end{itemize}
\end{itemize}
\end{slide}


\begin{slide}
\heading{Resolving Simultaneous Connects}

\begin{itemize}
\item What happens if two nodes try to simultaneously connect to each other
  \begin{itemize}
  \item End up with two connections
  \end{itemize}

\item Tie-breaker algorithm [Chen]
  \begin{itemize}
  \item Comparison of Node-Id $A$, $B$ as unsigned integers 
  \item Keep $A \rightarrow B$ connection iff $A > B$
  \item Larger node-id sends a \verb^Error_In_Progress^ error (new)
  \end{itemize}
\end{itemize}

\end{slide}



\begin{slide}
\heading{Replica/Topology Shift Stores and Lifetime}

\begin{itemize}
\item At time $t_0$ data is stored at $A$ with lifetime $T$ 
\item At time $t_1$, $A$ stores to $B$ and needs to send lifetime $T'$
  \begin{itemize}
  \item Draft didn't define the adjustment algorithm
  \end{itemize}
\item \S\ 6.4.1.1 now has an explicit algorithm:
  \begin{itemize}
    \item $T' = T - (t_1 - t_0)$
  \end{itemize}
\end{itemize}
\end{slide}



\begin{slide}
\heading{Triggered ConfigUpdates}

\begin{itemize}
\item $A$ tries to store item of kind $k$ to $B$ but $B$ doesn't know about $k$
  \begin{itemize}
  \item $B$ sends back \verb^Error_Unknown_Kind^
  \end{itemize}

\item This is a signal that $B$'s config is likely out of date
\item New requirement (\S\ 6.4.1.2) for $A$ to generate a \verb^ConfigUpdate^
\end{itemize}
\end{slide}


\begin{slide}
\heading{Resource-ID Computation Algorithm}

\begin{itemize}
\item Used to be in the configuration document but now it's defined by the overlay algorithm
  \begin{itemize}
  \item Proposal: SHA-1 for Chord
  \end{itemize}
\item Bug in the current document: \S\ 9.2 specifies truncation to 128 but node-ids vary between 128 and 160 bits (\S\ 5.3.1.1)
  \begin{itemize}
  \item This is obviously bad
  \item Proposal: SHA-1 truncated to node-id length
  \end{itemize}
\end{itemize}

\end{slide}






\begin{slide}
\heading{Clarify Congestion Control Considerations}

\begin{itemize}
\item Wasn't totally clear what the requirements were for flow control algorithms.
Reworded in \S\ 5.6.3.1
\end{itemize}

\small{
\begin{quote}
   ``Because the receiver's role is limited to providing packet
   acknowledgements, a wide variety of congestion control algorithms can
   be implemented on the sender side while using the same basic wire
   protocol.  In general, senders MAY implement any rate control scheme
   of their choice, provided that it is REQUIRED to be no more
   aggressive then TFRC[RFC5348].

   The following section describes a simple, inefficient, scheme that
   complys with this requirement.  Another alternative would be TFRC-SP
   [RFC4828] and use the received bitmask to allow the sender to compute
   packet loss event rates.''
\end{quote}
}

\end{slide}

\begin{slide}
\heading{Extensive Uncontroversial Technical and Editorial Changes}

Special thanks to Marc Petit-Huegenin for an amazingly thorough reviews.
\end{slide}


\begin{slide}
\heading{Open Issue: How to get a certificate with multiple Node-Ids?}

\begin{itemize}
\item Nothing in the draft here.
\item Consensus seems to be \verb^POST^ argument
\item Concrete Proposal: \verb^nodeids=X^ query parameter
\end{itemize}

\end{slide}



\begin{slide}
\heading{Open Issue: Signing With Multiple Node-Ids}

\vspace{-.4in}

\begin{itemize}
\item Signatures are currently tied to certificates (\verb^SignerIdentity^)
  \begin{itemize}
  \item But what about certificates with multiple Node-Ids?
  \end{itemize}

\item Proposal: hash the Node-Id into the hash of the certificate
\end{itemize}

{\scriptsize
\begin{verbatim}
    struct {
      select (identity_type) {
        case cert_hash:
          HashAlgorithm      hash_alg;              // From TLS
          opaque             certificate_hash<0..2^8-1>;

        case cert_hash_node_id:
          HashAlgorithm      hash_alg;              // From TLS
          opaque             certificate_node_id_hash<0..2^8-1>;

        /* This structure may be extended with new types if necessary*/
      };
    } SignerIdentityValue;

    certificate_node_id_hash = H(NodeId || certificate)
\end{verbatim}
}

\begin{itemize}
  \item Cost = $n * m$
\end{itemize}

\end{slide}


\begin{slide}
\heading{Identifying Connection Endpoints with muliple Node-Ids}

\begin{itemize}
\item How do you know who is at the other end of the connection?
\end{itemize}
\begin{itemize}
\item Proposal from Bruce Lowekamp: use \verb^Attach^ for all but bootstrap
  \begin{itemize}
  \item This includes client connections
  \item This allows determination of Node-Id
  \item Current status of \S\ 11.
  \end{itemize}

\item Marc Petit-Huegenin argues that this doesn't work for bootstrap
  \begin{itemize}
  \item Suggests using a \verb^Ping^ (possibly optional) instead.
  \end{itemize}

\item Discuss?
\end{itemize}
\end{slide}



\begin{slide}
\heading{Open Issue: Who Signs Stores?}

\begin{itemize}
\item \S\ 6.4.1.1 (\verb^StoreReq^ processing) requires verifying the stored value signatures
\item \S\ 6.3.x (access control policies) is phrased in terms of the signed request

\item Both signatures matter, but this is wrong
\item Proposal:
  \begin{itemize}
  \item Check stored value signatures against access control policies (\S\ 6.3.1)
  \item Check the request signature against either access control policies (original store) or plausible responsible node (replica store)
  \end{itemize}
\end{itemize}
\end{slide}



\begin{slide}
\heading{Open Issue: Refreshing Finger Table}

\begin{itemize}
\item Formula is still wrong
\item Proposal: Dr. Fluffy to fix
\end{itemize}



\end{slide}


\begin{slide}
\heading{Way Forward}

\begin{itemize}

\item Resolve these issues
\item Issue a new draft by 4/30
\item ???
\item Profit
\end{itemize}

\end{slide}


% Who signs synthesized item
% What certificate used to sign configuration element in XML file?
  -- 



\end{document}