\documentclass[helvetica]{seminar} 
\input{xy}
\xyoption{all}
\usepackage{graphicx} 
\usepackage{slidesec} 
\usepackage{url}

\long\def\symbolfootnote[#1]#2{\begingroup%
\def\thefootnote{\fnsymbol{footnote}}\footnote[#1]{#2}\endgroup}

% to fix problems making landscape seminar pdfs
% Letter...
\pdfpagewidth=11truein
\pdfpageheight=8.5truein
\pdfhorigin=1truein     % default value(?), but doesn't work without
\pdfvorigin=1truein     % default value(?), but doesn't work without
% A4
%\pdfpagewidth=297truemm % your milage may vary....
%\pdfpageheight=210truemm
%\pdfhorigin=1truein     % default value(?), but doesn't work without
%\pdfvorigin=1truein     % default value(?), but doesn't work without



\renewcommand{\familydefault}{\sfdefault}  
 
\input{seminar.bug} 
\input{seminar.bg2} % See the Seminar bugs list 
 
\slideframe{none} 
 
 
\usepackage{fancyhdr} 
 
% Headers and footers personalization using the `fancyhdr' package 
\fancyhf{} % Clear all fields 
\renewcommand{\headrulewidth}{0mm} 
\renewcommand{\footrulewidth}{0.1mm} 
 
\fancyfoot[L]{\tiny IETF 80} 
\fancyfoot[C]{\tiny RTC-Web Security Issues}
\fancyfoot[R]{\tiny \theslide} 
 
 
% To center horizontally the headers and footers (see seminar.bug) 
\renewcommand{\headwidth}{\textwidth} 

% To adjust the frame length to the header and footer ones 
\autoslidemarginstrue 
\pagestyle{fancy} 
 

\newcommand{\heading}[1]{% 
  \begin{center} 
    \large\bf 
    #1 
  \end{center} 
  \vspace{.4 in}} 

\begin{document}
\begin{slide}
\begin{center}
\vspace{1 in}
\LARGE{{\bf}RTC-Web Security Considerations}\\
\vspace{.2in}
\large{{IETF 80}} \\
\vspace{3em}
\large{
\begin{tabular}{c}
Eric Rescorla \\
\url{ekr@rtfm.com}
\end{tabular}
}
\end{center}

\end{slide}


\centerslidesfalse 


\begin{slide}
\heading{The Browser Threat Model}

\textbf{Core Web Security Guarantee}: ``users can safely visit arbitrary web sites
and execute scripts provided by those sites.''\cite{huang-cache}

\vspace{1em}

\begin{itemize}
\item This includes sites which are hosting malicious scripts!
\item Basic Web security technique is isolation/sandboxing
  \begin{itemize}
  \item Protect your computer from malicious scripts
  \item Protect content from site A from content hosted at site B
  \item Protect site A from content hosted at site B
  \end{itemize}
\item In this case we're primarily concerned with JavaScript running in the browser
\end{itemize}

\textbf{The browser acts as a trusted computing base for the site}

\end{slide}



\begin{slide}
\heading{List of Issues to Consider}

\begin{itemize}
\item Consent to communications
\item Access to local devices
\item Communications security
\end{itemize}
              
\end{slide}


\begin{slide}
\heading{In an alternate universe: Cross-Site Requests}

\vspace{-.65in}
$$
\xymatrix@R=.2in@C=1.3in{
  \txt{\textbf{Victim}} & \txt{\textbf{Gmail}} & \txt{\textbf{Attacker}} \\
  \ar[r]^{\txt{Login w/ Password}} & & \\
  & \ar[l]_{\txt{Cookie=XXX}} & \\
  & \txt{\textbf{...}} & \\
  \ar[rr]^{\txt{\texttt{GET /malicious.js}}} & &\\
  & & \ar[ll]_{\txt{\texttt{<script>XmlHttpRequest("https://gmail.com/")...}}} \\
  \ar[r]^{\txt{\texttt{GET} w/ XXX}} & & \\
  & \ar[l]_{\txt{Mail data}} & \\
  \ar[rr]^{\txt{Mail data}} & & \\
}
$$

\center{\textbf{Obviously this is bad... and it's a problem even w/o cookies}}

\end{slide}



\begin{slide}
\heading{The Same Origin Policy (SOP)}

\begin{itemize}
\item A page's security properties are determined by its \emph{origin}
  \begin{itemize}
  \item This includes: protocol (HTTP or HTTPS), host, and port
  \item All these must match for two pages to be from the same origin
  \end{itemize}

\item Each origin is associated with its own security contet
  \begin{itemize}
  \item Scripts in origin A have only very limited access to resources in origin B
  \end{itemize}


\item \emph{Important:} the origin is associated with the page, \emph{not} where the script came from
  \begin{itemize}
  \item Scripts loaded via \verb^<script src="">^ tags are associated with the origin of the page, not the URL for the script!

  \end{itemize}
\end{itemize}
\end{slide}




\begin{slide}
\heading{The Same Origin Policy for Page Data}
\begin{itemize}
\item Scripts can only access page data from their own origin
  \begin{itemize}
  \item Contents of the DOM
  \item JavaScript variables
  \item Cookies
  \item Important exception: JavaScript pointer leakage~\cite{barth-weinberger-song-cross-origin}
  \end{itemize}

\item Scripts can access any other page data from their origin
  \begin{itemize}
  \item Includes other windows and IFRAMEs
  \end{itemize}

\item Frame can navigate their own children
  \begin{itemize}
  \item This is used for cross-site communication (e.g., FaceBook Connect)
  \end{itemize}
\end{itemize}
\end{slide}



\begin{slide}
\heading{The Same Origin Policy for HTTP Requests}

\begin{itemize}
\item JavaScript can be used to make fairly controllable HTTP requests with \verb^XmlHttpRequest()^ API
  \begin{itemize}
  \item But only to the same origin 
  \end{itemize}

\item Origin A can make partly controllable requests to origin B via HTML forms
  \begin{itemize}
  \item But cannot read the response
  \item \emph{Cross-Site Request Forgery} (CSRF) defenses depend on this
  \end{itemize}

\item Origin A can read scripts from origin B
  \begin{itemize}
  \item But they run in A's context
  \item This is done all the time (e.g., Google analytics)
  \end{itemize}
\end{itemize}
\end{slide}



\begin{slide}
\heading{What does all this mean for RTC?}

\end{slide}


\begin{slide}
\heading{Consent for real-time peer-to-peer communication}

\begin{itemize}
\item Need to able to send data between two browsers
  \begin{itemize}
  \item Unless you want to relay everything
  \end{itemize}

\item But this is unsafe (and violates SOP)
  \begin{itemize}
  \item Not OK to let browsers send TCP and UDP to arbitrary locations
  \end{itemize}

\item General principle: verify consent
  \begin{itemize}
  \item Before sending traffic from a script to recipient, verify recipient
    wants to receive it from the sender
  \item Familiar paradigm from CORS~\cite{cors} and WebSockets\cite{hybi}
  \end{itemize}
\end{itemize}

\end{slide}

\begin{slide}
\heading{How to verify consent for RTC-Web}

\vspace{-.2in}
\begin{itemize}
\item Can't trust the server (see above)
  \begin{itemize}
  \item Needs to be enforced by the browser
  \end{itemize}

\item Browser does a handshake with target peer to verify connectivity
\end{itemize}

\vspace{-.25in}
$$
\xymatrix@R=.2in@C=1.3in{
  \txt{\textbf{Alice}} & \txt{\textbf{Server}} & \txt{\textbf{Bob}} \\
  & \ar[l]_{\txt{Connect to Bob}}  \ar[r]^{\txt{Connect to Alice}} &\\
  \ar@{<->}[rr]^{\txt{Handshake}} & & \\
  \ar@{<->}[rr]^{\txt{Media traffic}} & &\\
}`
$$

\begin{itemize}
\item This should look familiar from ICE~\cite{rfc5245}
\item Restricts communication with that endpoint until handshake complete (\textbf{new})
\end{itemize}
\end{slide}



\begin{slide}
\heading{Access to Local Devices}

\vspace{-.2in}

\begin{itemize}
\item Making phone (and video) calls requires that your voice be transmitted to other side
  \begin{itemize}
  \item But the \emph{other side} is controlled by some site you visit
  \item What if you visit \url{http://bugmyphone.example.com}?
  \end{itemize}

\item Somehow we need to get the user's consent
  \begin{itemize}
  \item But to what?
  \item And when?
  \item Users routinely click through warning dialogs when presenting ``in-flow''
  \end{itemize}

\item What is the scope of consent?
  \begin{itemize}
  \item By origin?
  \item What about mash-ups?
  \end{itemize}
  

\end{itemize}
\end{slide}


\begin{slide}
\heading{What about communications security?}

\begin{itemize}
\item We've already addressed this in the context of SIP
  \begin{itemize}
  \item Things aren't that different here--all the usual protocols work
  \end{itemize}

\item Open question: where is the keying material stored?
  \begin{itemize}
  \item On the server?
  \item In localstorage?
  \item In the browser but isolated from the JavaScript? (probably best)
  \end{itemize}
  
\end{itemize}

\end{slide}




\begin{slide}
\scriptsize{
\bibliographystyle{alpha}
\bibliography{ws}
}
\end{slide}






\end{document}