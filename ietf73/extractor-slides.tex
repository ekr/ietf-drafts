\documentclass[helvetica]{seminar} 
\centerslidesfalse 
\input{xy}
\xyoption{all}
\usepackage{graphicx} 
\usepackage{slidesec} 
\usepackage{url} 

\long\def\symbolfootnote[#1]#2{\begingroup%
\def\thefootnote{\fnsymbol{footnote}}\footnote[#1]{#2}\endgroup}

% to fix problems making landscape seminar pdfs
% Letter...
\pdfpagewidth=11truein
\pdfpageheight=8.5truein
\pdfhorigin=1truein     % default value(?), but doesn't work without
\pdfvorigin=1truein     % default value(?), but doesn't work without
% A4
%\pdfpagewidth=297truemm % your milage may vary....
%\pdfpageheight=210truemm
%\pdfhorigin=1truein     % default value(?), but doesn't work without
%\pdfvorigin=1truein     % default value(?), but doesn't work without



\renewcommand{\familydefault}{\sfdefault}  
 
\input{seminar.bug} 
\input{seminar.bg2} % See the Seminar bugs list 
 
\slideframe{none} 
 
 
\usepackage{fancyhdr} 
 
% Headers and footers personalization using the `fancyhdr' package 
\fancyhf{} % Clear all fields 
\renewcommand{\headrulewidth}{0mm} 
\renewcommand{\footrulewidth}{0.1mm} 
 
\fancyfoot[L]{\tiny IETF 73} 
\fancyfoot[C]{\tiny TLS Extractors}
\fancyfoot[R]{\tiny \theslide} 
 
 
% To center horizontally the headers and footers (see seminar.bug) 
\renewcommand{\headwidth}{\textwidth} 

% don't center vertically
\centerslidesfalse 
% To adjust the frame length to the header and footer ones 
\autoslidemarginstrue 
\pagestyle{fancy} 
 

\newcommand{\heading}[1]{% 
  \begin{center} 
    \large\bf 
    #1 
  \end{center} 
  \vspace{.4 in}} 
\begin{document}


        
\begin{slide}
\begin{center}
\LARGE{{\bf}TLS Extractor Status}\\

\vspace{.3 in}
\large{Eric Rescorla}\\
\large{RTFM, Inc.}\\
\large{\texttt{ekr@networkresonance.com}}
\end{center}
\end{slide}



\begin{slide}
\heading{Document Currently in Last Call}

\begin{itemize}
\item Last call ends Dec 4, 2008
\item Comments from Pasi Eronen, Alfred H\"{i}nes, Hugo Krawczyk
\item Mostly editorial
\item One typo in the definition of \textsf{context\_value}
\item Terminology issue from Hugo
\end{itemize}
\end{slide}


\begin{slide}
\heading{Terminology Issue}


\footnotesize{
\begin{verbatim}
I have a single "objection" to this document, namely, the use of the word
"extractor". Let me explain.

In the context of key derivation functions the notion of extraction refers
to a first phase where one starts with a somewhat weak source of randomness
(such as an imperfect RNG, a Diffie-Hellman value, etc) and extracts a first
cryptographically strong key K.

In a second phase, often called key expansion, one derives multiple keys out
of this K using a PRF exactly as the current document specifies. Since
master_secret is assumed to already be a cryptographically strong key, then
this specification is sound and correct (especially that it includes the
essential context information).
\end{verbatim}
}
\end{slide}


\begin{slide}
\heading{Proposed New Names}

\begin{itemize}
\item Exporter
\item Deriver
\item Your name here
\end{itemize}


\end{slide}




\end{document}